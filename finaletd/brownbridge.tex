\section{Cyclic Graph}
The attractive property of reference counting algorithm is localized computation and incremental approach. Mark Sweep algorithm innately requires synchronization and computing the garbage nodes globally. Although reference counting had attractive benefits, reference counting  suffered from fundamental problem of inability to detect all types of garbage especially cyclic garbage. Cyclic garbage are collection of nodes that creates a cycle. By reference counting principle, all the nodes in the graph contain non-zero reference count and hence the algorithm does not recognize the garbage. These cyclic garbage are very common. For a garbage collector to be complete, it is not enough to guarantee the safety property. Liveness property is crucial property to ensure the functional requirements of garbage collector. Brownbridge invented an idea to classify edges in the graph into two types namely : strong and weak. 
\section{Cycle detection using Strong-Weak}
Brownbridge classifies the edges in the graph into strong and weak. The classification is based on the invariant that there is no cycle of strong edges. So every cycle must contain at least one weak edge. The notion of strong edge is connected to liveness of a node. So  the existence of strong edge indicates that the node is live. 

Strong edges form a connected, acyclic, directed graph in which every node is reachable from R. The rest of the edges are classified as weak. This classification is not simple to compute as themselves might requires complete scanning of the graph. To avoid complete scanning, some heurisitic approaches are required. The identification of cycle creating edges are the bottleneck of the classification problem. The edges are  labeled weak if it is created to a node that already exists in the memory. this heuristic guarantee that all cycle contains at least a weak edge. But a node with weak incoming edge does not mean it is part of a cycle. The heuristic helps the mutator to save time in classifying the edges. But the collectors require more information about the topology of the subgraph to identify if the subgraph is indeed garbage to delete.
\section{Brownbridge Garbage Collector}
\section{Pitfalls of Brownbridge and successor's technique}
