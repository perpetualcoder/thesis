\section{Introduction}

Garbage-collected languages are widely used in distributed systems, including big-data applications in the cloud~\cite{maas2015trash,gog2015broom}.  Languages  in this space include
Java, Scala, Python, C\#, PHP, etc., and platforms include Hadoop, Spark,
Zookeeper, DryadLINQ, etc. In addition, cloud services such as Microsoft Azure and
Google AppEngine, and companies such as Twitter and Facebook all make significant
use of garbage-collected languages in their code
infrastructure~\cite{maas2015trash}.  Garbage collection is seen as a
significant benefit by the developers of these applications and platforms
because it eliminates a large class of programming errors, which translates
into higher developer productivity.
%Unfortunately, this cost can become high
%when object references are allowed to point outside of or between single
%shared memory systems.

Garbage collection is also an issue in networked object
stores, which share many properties with distributed systems. Like
such systems, they cannot use algorithms that require scanning the entire heap.
In any situation in which traces can go across storage boundaries, i.e. from
node to node, node to disk, etc., garbage collectors that need to trace the heap
becomes impractical.  What is needed is something distributed, decentralized, scalable,
that can run without ``stopping
the world,'' and has good time and space complexity.

%We note that time complexity may be of particular interest in increasing productivity.
%Many garbage-collection system are not able to reliably inform the programmer when
%objects are ready for reclamation. Most garbage collected languages make little effort
%to call destructors, or finalizers, and so this important feature is often unusable.

%In this present work we address all concerns except the last. We note, however,
%that as machines have scaled up they have also become more reliable and failures
%are not seen as frequently as many have feared.

%Distributed systems access objects stored at remote sites. Programming
%languages access objects allocated in the heap using stack variables and
%global variable.  They are often referred as roots. In the distributed
%systems, the heaps are distributed across sites and inter-site heap references
%are allowed. Any object is considered live only when it is reachable from any
%root.  So all objects that are not reachable will never be used and called
%garbage.  Garbage collectors identify the objects that are garabge and reclaim
%them.
There are two main types of garbage collectors, namely Tracing and Reference
Counting (RC). Tracing collectors track all the reachable objects from the root (in the reference graph)
and delete all the unreachable objects.
RC collectors
count the number of objects pointing to a given block of data at any point in
time.  When a reference count is decremented and becomes zero, its object is
garbage, otherwise it {\em might} be
garbage and some tracing is necessary (note that pure RC collectors do not trace and will not collect cycles ~\cite{rcfail}).  Unfortunately, cycles among distributed objects are
frequent~\cite{richer}.
%Our algorithm belongs to the family of hybrid RC collectors, which is RC with partial tracing. It advances the use of a
%three reference count collector designed for a single machine~\cite{Brandt2014}
%based on the Brownbridge~\cite{Brownbridge1985} algorithm. The advantage for such
%systems is that they can frequently determine that a reference count decrement
%does not create garbage without the need for tracing.
Previous attempts at distributed garbage collection, e.g. \cite{Hudak:1982,Ladin,LeFessant,liskov97,liskov95,Veiga}, suffer from the need for centralization,
%cooperation among all sites,
global barriers,
the migration of objects, or have inefficient time and space complexity guarantees.
%Our algorithm can be used in a system where the migration of the objects are
%not acceptable.
%In the work that follows we first describe the operation of a serial version
%of the collector, then proceed to discussion of the parallel version.



%%Objects become unreachable. The reclaimation of these unreachable objects are
%necessary for better performance. So distributed garbage collectors are
%inevitable in distributed systems.
%Although the distributed garbage collection is an active area of research, most
%of the interesting algorithms published are not practical. They are either
%centralized, require synchronization among all sites (a global barrier), or
%require migration of all objects in one cycle to a common machine.
%an accurate summary?}

%To identify all the nodes that are dead in graph G, there are multiple
%alternatives already available. The two abstract approach can be classified into
%two ideas of same coin. The first idea is identifying all the live nodes and
%then detecting the dead nodes. This approach is generally called as tracing
%approach. Tracing approaches uses an attribute called mark which will be set if
%there collectors reaches the node from root set R transitively. Then all the
%nodes where mark is not set in the garbage detection process will be considered
%garbage and deleted. This approach makes it clear that the node will not be able
%to detect if it is a garbage. A centralized component is required to find all
%the nodes that are live and then detect dead nodes. These systems are clearly
%not scalable as they require synchronization among all collection processes. As
%the model does not allow the nodes to know the $\inneighbors$ of a node,
%traversing all the nodes backwards to detect if the node has a path from any
%root is not considered. The model reflects the practical model used by real-time
%systems, that back edges for all $\inneighbors$ creates a additional memory
%overhead for each node. The additional memory overhead to save back edges can
%bounded to O(V).

%The second technique is identifying all the dead nodes by itself. This technique
%is widely called reference counted system. The nodes count the number of the
%incoming edges. This approach can detect a set of dead nodes trivially. When
%there are no incoming edges to a node, it is clear that the node is dead. The
%challenge in the reference counted system is node can be dead even if the node
%$x$ has incoming edges. If all dead nodes has at least an incoming edge from
%other dead node, garbage goes undetectad by simple reference counter. A simple
%example would be a cycle created by dead nodes and no node in the cycle has a
%path from root to it. Although simple reference counting suffers from complete
%garbage collection, they gained popularity because of the promptness and
%traversal cost to detect garbage. An garbage collection algorithm can be called
%prompt if it can detect dead node immediately and avoid dead nodes floating
%around. If reference counting technique can collect a cycle garbage and employ
%decentralized decision using locality, these properties make it suitable for the
%distributed system as they are easily scalable and prompt. A vast literature of
%cycle collecting reference counting asynchronous shared memory garbage
%collectors are available. All cycle collecting reference counted garabge
%collectors are hybrid approach. In fair execution, to detect dead nodes, the
%system employs a traversal that visits only a subset of nodes in graph. All
%asynchronous shared memory cycle collecting reference counted garbage collectors
%require some centralized data structure to detect dead nodes in dynamic graph
%with multiple garbage detection process running simultaneously. A centralized
%data structure based distributed algorithm will have bottlenecks and cannot be
%self stabilizing. To avoid all the bottlenecks and make the distributed garbage
%collection truely decentralized, our algorithm elminates use of centralized
%approach among all garbage detection processes in the distributed system and a
%node can make a decision about its liveness in finite steps once the node is
%dead.

\paragraph{Contributions:}
We present a hybrid reference counting algorithm for garbage collection in
distributed systems that works on the asynchronous model of communication with
reliable FIFO channel. Our algorithm collects both
non-cyclic and cyclic garbage in distributed systems of arbitrary topology by advancing on the
three reference count collector which only works for a single machine~\cite{Brandt2014}
based on the Brownbridge system~\cite{Brownbridge1985} \footnote{
Note, the original algorithm of Brownbridge suffered from premature collection, and subsequent attempts to remedy this problem needed exponential time in certain scenarios \cite{Salkild1987,Pepels1988}. We addressed those limitations in \cite{Brandt2014}.
}. The advantage for such
systems is that they can frequently determine that a reference count decrement
does not create garbage without the need for tracing. 
%However, our technique in \cite{Brandt2014} does not extend directly to the distributed garbage collection scenario involving multiple sites.

Our proposed algorithm is scalable
because it collects garbage in $O(E)$ time using only $O(\log n)$ bits memory per
node, where $E$ is the number of edges in the affected subgraph (of the reference
graph) and $n$ is the number of nodes of the reference graph. Moreover, in
contrast to previous work, our algorithm does not need centralization,
global barriers, or the migration of objects. Apart from the benefits
mentioned above, our algorithm handles concurrent mutation (addition and deletion of edges and nodes in the reference graph) and provides liveness
and safety guarantees by maintaining a property called \emph{isolation}.
Theorems \ref{pro:livenesss} and \ref{pro:safetys} prove that when a
collection process works alone (i.e. in isolation), it is guaranteed to collect
garbage and not to prematurely delete. Theorem \ref{thm:alliso} proves that
when multiple collection processes interact, our synchronization mechanisms
allow each of them to act as if they were working alone, and Theorems
\ref{liveness} and \ref{safety} prove the correctness in this setting.
To the best of our knowledge, this is the first algorithm for garbage
collection in distributed systems that simultaneously achieves such guarantees.

This algorithm falls into the category of self-stabilization algorithms, because
it maintains the invariants (1) all nodes strongly connected, (2) no strong
cycles, and (3) no floating garbage.

\paragraph{Related Work:}
%All existing garbage collectors designs fall into to one of these categories:
%Tracing, Reference Count and Hybrid.
Prior work on distributed garbage collection is vast; we discuss here the papers that are closely related to our work.  %Many of these collectors
%suffer from the need for centralization, global barriers,
%the migration of objects, or have inefficient time and space complexity.
The marking algorithm proposed by Hudak~\cite{Hudak:1982} requires a global barrier.
All local garbage
collectors coordinate to start the marking phase. Once the marking phase is over
in all the sites, then the sweeping phase continues. Along with the marking and sweeping overhead,
there are consistency issues in tracing based collectors~\cite{shapiro95}.

Ladin and Liskov~\cite{Ladin} compute reachability of objects in a highly
available centralized service. This algorithm is logically centralized but
physically replicated, hence achieving high availability and fault-tolerance.
All objects and tables are backed up in stable storage. Clocks are synchronized
and message delivery delay is bounded. These assumptions enable the centralized
service to build a consistent view of the distributed system. The centralized
service registers all the inter-space references and detects garbage using a standard
tracing algorithm. This algorithm has scalability issues due to the centralized
service. A heuristic-based algorithm by Le Fessant ~\cite{LeFessant} uses the minimal number
of inter-node references from a root to identify ``garbage suspects'' and then
verifies the suspects. The algorithm propagates marks from the references to all the
reachable objects. A cycle is detected when a remote object receives only its
own mark. The algorithm needs tight coupling between local
collectors and time complexity for the collection of cycles is not analyzed.

Collecting distributed garbage cycles by backtracking is proposed by Maheshwari
and Liskov~\cite{liskov97}. This algorithm first hypothesizes that objects are dead and
then tries to backtrack and prove whether that is true. The algorithm
needs to use an additional datastructure to store all inter-site references.
An alternative approach of distributed garbage collection by
migrating objects have been proposed by Maheshwari and Liskov~\cite{liskov95}. Both
of the algorithms use heuristics to detect garbage. The former one uses more
memory, the latter one increases the overhead by moving the objects between the
sites. Recently, Veiga et al.~\cite{Veiga} proposed an algorithm that uses
asynchronous local snapshots to identify global garbage. Cycle detection
messages (CDM) traverse the reference graph and gain information about the
graph. A drawback of this approach is that the algorithm doesn't work with the $\mathcal{CONGEST}$ model. %the growth of the messages is limited
%only by the size of the distributed system.
Any change to the snapshots
has to be updated by local mutators, forcing current global garbage collection
to quit. For a thorough understanding of the literature, we recommend reading~\cite{shapiro95, Abu}.


%\subsection{Distributed Garabge Collection = GC + Distributed Termination
%Detection}
%necessary in decentralized algorithm}

%Distributed termination detection(DTD) is a fundamental to distributed
%computation. The solution can be applied to any computation that tries to
%evaluate some stable property.
%Examples of such properties are
%deadlock detection, garbage detection etc. Once a system eventually enters into
%deadlock, it remains deadlocked. Once an object becomes garbage, it remains so.
%Detecting whether the stable property is reached is equivalent to termination in
%the DTD. The DTD can be formally described as a collection of processes
%communicating by messages. A process can be either passive or active. Active
%processes may send message while passive process will not send message. A passive
%process will be become an active when a message is received. When all the
%distributed processes are passive, the stable property is achieved. Stability
%here means no messages are in flight. The popular termination detection algorithms
%are ~\cite{dijk83,Tel88,dijk80,Misra82,Chandra90,nir}. Dijkstra and Scholten
%~\cite{dijk80} create spanning trees of the graph for detecting the termination
%detection. Our algorithm uses their solution to control the traversal and detect
%the termination.

%Tel et al showed that Distributed Termiantion Detection Algorithm (DTDA) can be
%derived from Garbage Collection Schemes.~\cite{Tel:1993}. Following Tel et al,
%Blackburn et al describes that reversing mapping is also
%possible~\cite{Blackburn:2001}. They also explained
%that any known DTDA with a centralized garbage collection will produce a
%distrbuted garbage collector (DGC). Although the literature clearly
%demonstrates the relationship between the fundamental problems of DTDA and DGC, cyclic
%reference counting-based DGC algorithm that contains this relation have never
%been explored. To our knowledge this is the first article that to do so.

%Most modern programming languages use heap and stack memory to run application.
%Objects are allocated in heap memory to increase the lifetime of the object.
%Managed runtime systems help developers to ignore the efficient deallocation of
%the objects. These managed runtime systems identify the heap allocated objects
%that are garbage and deallocate them. The relationship among various allocated
%objects in memory can be modeled as directed graph. Pointers that are in stack
%can hold reference to object in the heap. These pointers are generally referred
%as roots. Also, global pointers can also be considered as roots. In abstract,
%any pointer that does not reside in heap memory but holds reference to heap
%objects are called roots. Any objects in the heap are called nodes. These nodes
%are synonymous to regular graph nodes in graph theory. Any node can hold
%reference to any other node. These references are unidirectional link incident
%on two nodes. So they form a directed graph. The distinction between roots and
%nodes are any root can hold reference to nodes but nodes can only reference to
%other nodes. To map this into graph, we have two kind of vertices. Roots are
%special vertices that does not have any incoming link but an outgoing link.
%Nodes are regular vertex that can have incoming and outgoing links. A node is
%considered reachable if the node has a path from one of the root node to it.
%Unreachable nodes are called garbage and needs to be deleted for efficiency. The
%problem of garbage collection can be described as problem of finding all the
%garbage nodes in the reference graph and reclaim/delete them.

%It is shown in ~\cite{Tel:1993,Blackburn:2001} that there is
%strong relation between DTD and distributed garabge collection. Tel et al shown
%that a distributed termination detection algorithm can be derived from any
%distributed garabge collector~\cite{Tel:1993}. Blackburn et al, described that a
%distributed garbage collector is composite of distributed termination detection
%and a garbage collector ~\cite{Blackburn:2001}. Our algorithm indeed has a
%distributed termination detection algorithm in it along with a garbage
%collection algorithm.

%\subsection{Contributions}

\paragraph{Paper Outline:}
%The rest of the paper is organized as follows.
Section \ref{model} describes the garbage collection problem, constraints involved in the problem, and the model for which the algorithm is presented. Section \ref{single} explains the single collector version of the algorithm and provides correctness, time, and space guarantees.
Section \ref{multi} extends the results of Section \ref{single} to the multiple collector scenario. Finally, Section \ref{section:conclusions} concludes the paper with a short discussion. Some of the proofs and the pseudocode for the algorithms may be found in Appendix \ref{section:appendix}.
