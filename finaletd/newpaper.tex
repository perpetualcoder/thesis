\begin{comment}
\documentclass[11pt]{article}
\usepackage[text={6.6in,9.3in},centering]{geometry}
\usepackage{times}
\usepackage{amsthm}
%\usepackage[ruled,vlined,linesnumbered]{algorithm2e}
\usepackage{amssymb, amsmath}
\usepackage{color}
\usepackage{graphicx}
%\usepackage{subfig}
\usepackage{lipsum}
\usepackage{wrapfig}
\usepackage{url}
\usepackage{hyperref}
\usepackage{verbatim}
\usepackage{xcolor}
\usepackage{algorithm}
\usepackage{amsmath}
\usepackage{multicol}
\usepackage{algpseudocode}
\usepackage{subcaption}
\usepackage{tikz}
\usetikzlibrary{arrows}
\usepackage{cleveref}
%\usepackage{algcompatible}

 \usepackage{pstricks}
% The following commands are not supported in PSTricks at present
% We define them conditionally, so when they are implemented,
% this pstricks file will use them.
\ifx\setlinejoinmode\undefined
  \newcommand{\setlinejoinmode}[1]{}
\fi
\ifx\setlinecaps\undefined
  \newcommand{\setlinecaps}[1]{}
\fi
% This way define your own fonts mapping (for example with ifthen)
\ifx\setfont\undefined
  \newcommand{\setfont}[2]{}
\fi

\algtext*{EndIf}% Remove "end if" text

%\newtheorem{theorem}{Theorem}[section]

\newtheorem{theorem}{Theorem}
\newtheorem{lemmas}{Lemma}
\newtheorem{axioms}{Axiom}
\newtheorem{lemma}[lemmas]{Lemma}
\newtheorem{axiom}[axioms]{Axiom}
\newtheorem{claim}[theorem]{Claim}
\newtheorem{fact}[theorem]{Fact}
\newtheorem{corollary}[theorem]{Corollary}
\newtheorem{definition}{Definition}
\newtheorem{observation}[lemmas]{Corollary}
\newtheorem{invariant}{Invariant}

\newtheorem*{rep@theorem}{\rep@title}
\newcommand{\newreptheorem}[2]{%
\newenvironment{rep#1}[1]{%
 \def\rep@title{#2 \ref{##1}}%
 \begin{rep@theorem}}%
 {\end{rep@theorem}}}


\newreptheorem{theorem}{Theorem}
%\newtheorem{lemma}{Lemma}
\newreptheorem{lemma}{Lemma}



\newcommand{\sq}{\hbox{\rlap{$\sqcap$}$\sqcup$}}
%\newcommand{\qed}{\hspace*{\fill}\sq}
%\newenvironment{proof}{\noindent {\bf Proof.}\ }{\qed\par\vskip %4mm\par}
\newenvironment{proofs}{\noindent {\bf Proof (sketch).}\ }{\qed\par\vskip
0mm\par}

\newcommand{\todo}[1]{\textcolor{red}{\textsf{[#1]}}}
\newcommand{\srb}[1]{\textcolor{blue}{\textsf{Steve: [#1]}}}
\newcommand{{\rh}}{{\widehat r}}
\newcommand{{\Rh}}{{\widehat R}}
\newcommand{\E}{{\mathcal E}}
\newcommand{\T}{{\mathcal T}}
\newcommand{\ww}{{\mathfrak{w}}}
\newcommand{\diam}{{\rm diam}}
\newcommand{\dist}{{\rm dist}}
\newcommand{\length}{{\rm length}}
\newcommand{\publish}{{\sl publish}}
\newcommand{\move}{{\sl move}}
\newcommand{\lookup}{{\sl lookup}}
\newcommand{\Read}{{\rm Read}}
\newcommand{\Write}{{\rm Write}}
\newcommand{\mysf}[1]{\textup{\sffamily #1}}
\newcommand{\CONGEST}{\mysf{CONGEST}}

\newcommand{\note}[1]{[[[{\bf #1}]]]}

\newcommand\outneighbors{\mathop{\mbox{$out$-$\mathit{neighbors}$}}}
\newcommand\inneighbors{\mathop{\mbox{$in$-$\mathit{neighbors}$}}}
\newcommand\maxweight{\mathop{\mbox{$max$-$\mathit{weight}$}}}
\newcommand\nonstrong{\mathop{\mbox{$non$-$\mathit{strong}$}}}


\begin{document}
\sloppy
\begin{titlepage}

\title{Distributed Garbage Collection for General Graphs}

\author{
Hari Krishnan\thanks{Corresponding author.} \thanks{Center for Computation and Technology, and Computer
Science and Engineering Division, Louisiana State University, LA 70803.}\\
{\em  Louisiana State University}\\
\url{hkrish4@tigers.lsu.edu}
\and
Steven R Brandt\thanks{Center for Computation and Technology, and Computer
Science and Engineering Division, Louisiana State University, LA 70803.} \\
{\em  Louisiana State University}\\
\url{sbrandt@cct.lsu.edu}
\and
Costas Busch\thanks{Computer Science and Engineering Division, Louisiana State
University, LA 70803.}\\
{\em Louisiana State University}\\
\url{busch@csc.lsu.edu}
\and
Gokarna Sharma\thanks{Department of Computer Science, Kent
State University, OH 44242}\\
{\em Kent State University}\\
\url{sharma@cs.kent.edu}
}

\date{}
\maketitle


\begin{abstract}

We propose a scalable, cycle-collecting, decentralized, reference counting
garbage collector with partial tracing.  The algorithm is based on the
Brownbridge system but uses four different types of references to label edges
in the reference graph. Memory usage is $O(\log n)$ bits per node, where $n$ is the
number of nodes in the graph.  The algorithm assumes an asynchronous network
model with a reliable FIFO channel. It collects garbage in
$O(E)$ time, where $E$ is the number of edges in the induced subgraph of the reference graph. The
algorithm uses termination detection to manage the
distributed computation, a unique identifier to break the symmetry among
multiple collectors, and a transaction-based approach 
when multiple collectors conflict. Unlike existing
algorithms, ours is not centralized, does not require barriers,
does not require
migration of nodes, does not require back-pointers on every edge,
and is stable against concurrent mutation. %, even with multiple collection
%operations overlapping the induced subgraph.

%\keywords{Distributed Garbage Collection, Distributed Termination Detection,
%Strong Weak Phantom, Reference Count}
\end{abstract} 

\centerline{}
\centerline{}
\centerline{{\bf Regular Paper. Consider also for Brief Announcement.}}
\centerline{{\bf Eligible for best student paper award: Hari Krishnan is a full
time student.}}

\thispagestyle{empty}


\end{titlepage}
\end{comment}

\section{Introduction}

Garbage-collected languages are widely used in distributed systems, including big-data applications in the cloud~\cite{maas2015trash,gog2015broom}.  Languages  in this space include
Java, Scala, Python, C\#, PHP, etc., and platforms include Hadoop, Spark,
Zookeeper, DryadLINQ, etc. In addition, cloud services such as Microsoft Azure and
Google AppEngine, and companies such as Twitter and Facebook all make significant
use of garbage-collected languages in their code
infrastructure~\cite{maas2015trash}.  Garbage collection is seen as a
significant benefit by the developers of these applications and platforms
because it eliminates a large class of programming errors, which translates
into higher developer productivity.
%Unfortunately, this cost can become high
%when object references are allowed to point outside of or between single
%shared memory systems.

Garbage collection is also an issue in networked object
stores, which share many properties with distributed systems. Like
such systems, they cannot use algorithms that require scanning the entire heap.
In any situation in which traces can go across storage boundaries, i.e. from
node to node, node to disk, etc., garbage collectors that need to trace the heap
becomes impractical.  What is needed is something distributed, decentralized, scalable,
that can run without ``stopping
the world,'' and has good time and space complexity.

%We note that time complexity may be of particular interest in increasing productivity.
%Many garbage-collection system are not able to reliably inform the programmer when
%objects are ready for reclamation. Most garbage collected languages make little effort
%to call destructors, or finalizers, and so this important feature is often unusable.

%In this present work we address all concerns except the last. We note, however,
%that as machines have scaled up they have also become more reliable and failures
%are not seen as frequently as many have feared.

%Distributed systems access objects stored at remote sites. Programming
%languages access objects allocated in the heap using stack variables and
%global variable.  They are often referred as roots. In the distributed
%systems, the heaps are distributed across sites and inter-site heap references
%are allowed. Any object is considered live only when it is reachable from any
%root.  So all objects that are not reachable will never be used and called
%garbage.  Garbage collectors identify the objects that are garabge and reclaim
%them.
There are two main types of garbage collectors, namely Tracing and Reference
Counting (RC). Tracing collectors track all the reachable objects from the root (in the reference graph)
and delete all the unreachable objects.
RC collectors
count the number of objects pointing to a given block of data at any point in
time.  When a reference count is decremented and becomes zero, its object is
garbage, otherwise it {\em might} be
garbage and some tracing is necessary (note that pure RC collectors do not trace and will not collect cycles ~\cite{rcfail}).  Unfortunately, cycles among distributed objects are
frequent~\cite{richer}.
%Our algorithm belongs to the family of hybrid RC collectors, which is RC with partial tracing. It advances the use of a
%three reference count collector designed for a single machine~\cite{Brandt2014}
%based on the Brownbridge~\cite{Brownbridge1985} algorithm. The advantage for such
%systems is that they can frequently determine that a reference count decrement
%does not create garbage without the need for tracing.
Previous attempts at distributed garbage collection, e.g. \cite{Hudak:1982,Ladin,LeFessant,liskov97,liskov95,Veiga}, suffer from the need for centralization,
%cooperation among all sites,
global barriers,
the migration of objects, or have inefficient time and space complexity guarantees.
%Our algorithm can be used in a system where the migration of the objects are
%not acceptable.
%In the work that follows we first describe the operation of a serial version
%of the collector, then proceed to discussion of the parallel version.



%%Objects become unreachable. The reclaimation of these unreachable objects are
%necessary for better performance. So distributed garbage collectors are
%inevitable in distributed systems.
%Although the distributed garbage collection is an active area of research, most
%of the interesting algorithms published are not practical. They are either
%centralized, require synchronization among all sites (a global barrier), or
%require migration of all objects in one cycle to a common machine.
%an accurate summary?}

%To identify all the nodes that are dead in graph G, there are multiple
%alternatives already available. The two abstract approach can be classified into
%two ideas of same coin. The first idea is identifying all the live nodes and
%then detecting the dead nodes. This approach is generally called as tracing
%approach. Tracing approaches uses an attribute called mark which will be set if
%there collectors reaches the node from root set R transitively. Then all the
%nodes where mark is not set in the garbage detection process will be considered
%garbage and deleted. This approach makes it clear that the node will not be able
%to detect if it is a garbage. A centralized component is required to find all
%the nodes that are live and then detect dead nodes. These systems are clearly
%not scalable as they require synchronization among all collection processes. As
%the model does not allow the nodes to know the $\inneighbors$ of a node,
%traversing all the nodes backwards to detect if the node has a path from any
%root is not considered. The model reflects the practical model used by real-time
%systems, that back edges for all $\inneighbors$ creates a additional memory
%overhead for each node. The additional memory overhead to save back edges can
%bounded to O(V).

%The second technique is identifying all the dead nodes by itself. This technique
%is widely called reference counted system. The nodes count the number of the
%incoming edges. This approach can detect a set of dead nodes trivially. When
%there are no incoming edges to a node, it is clear that the node is dead. The
%challenge in the reference counted system is node can be dead even if the node
%$x$ has incoming edges. If all dead nodes has at least an incoming edge from
%other dead node, garbage goes undetectad by simple reference counter. A simple
%example would be a cycle created by dead nodes and no node in the cycle has a
%path from root to it. Although simple reference counting suffers from complete
%garbage collection, they gained popularity because of the promptness and
%traversal cost to detect garbage. An garbage collection algorithm can be called
%prompt if it can detect dead node immediately and avoid dead nodes floating
%around. If reference counting technique can collect a cycle garbage and employ
%decentralized decision using locality, these properties make it suitable for the
%distributed system as they are easily scalable and prompt. A vast literature of
%cycle collecting reference counting asynchronous shared memory garbage
%collectors are available. All cycle collecting reference counted garabge
%collectors are hybrid approach. In fair execution, to detect dead nodes, the
%system employs a traversal that visits only a subset of nodes in graph. All
%asynchronous shared memory cycle collecting reference counted garbage collectors
%require some centralized data structure to detect dead nodes in dynamic graph
%with multiple garbage detection process running simultaneously. A centralized
%data structure based distributed algorithm will have bottlenecks and cannot be
%self stabilizing. To avoid all the bottlenecks and make the distributed garbage
%collection truely decentralized, our algorithm elminates use of centralized
%approach among all garbage detection processes in the distributed system and a
%node can make a decision about its liveness in finite steps once the node is
%dead.

\paragraph{Contributions:}
We present a hybrid reference counting algorithm for garbage collection in
distributed systems that works on the asynchronous model of communication with
reliable FIFO channel. Our algorithm collects both
non-cyclic and cyclic garbage in distributed systems of arbitrary topology by advancing on the
three reference count collector which only works for a single machine~\cite{Brandt2014}
based on the Brownbridge system~\cite{Brownbridge1985} \footnote{
Note, the original algorithm of Brownbridge suffered from premature collection, and subsequent attempts to remedy this problem needed exponential time in certain scenarios \cite{Salkild1987,Pepels1988}. We addressed those limitations in \cite{Brandt2014}.
}. The advantage for such
systems is that they can frequently determine that a reference count decrement
does not create garbage without the need for tracing. 
%However, our technique in \cite{Brandt2014} does not extend directly to the distributed garbage collection scenario involving multiple sites.

Our proposed algorithm is scalable
because it collects garbage in $O(E)$ time using only $O(\log n)$ bits memory per
node, where $E$ is the number of edges in the affected subgraph (of the reference
graph) and $n$ is the number of nodes of the reference graph. Moreover, in
contrast to previous work, our algorithm does not need centralization,
global barriers, or the migration of objects. Apart from the benefits
mentioned above, our algorithm handles concurrent mutation (addition and deletion of edges and nodes in the reference graph) and provides liveness
and safety guarantees by maintaining a property called \emph{isolation}.
Theorems \ref{pro:livenesss} and \ref{pro:safetys} prove that when a
collection process works alone (i.e. in isolation), it is guaranteed to collect
garbage and not to prematurely delete. Theorem \ref{thm:alliso} proves that
when multiple collection processes interact, our synchronization mechanisms
allow each of them to act as if they were working alone, and Theorems
\ref{liveness} and \ref{safety} prove the correctness in this setting.
To the best of our knowledge, this is the first algorithm for garbage
collection in distributed systems that simultaneously achieves such guarantees.

This algorithm falls into the category of self-stabilization algorithms, because
it maintains the invariants (1) all nodes strongly connected, (2) no strong
cycles, and (3) no floating garbage.

\paragraph{Related Work:}
%All existing garbage collectors designs fall into to one of these categories:
%Tracing, Reference Count and Hybrid.
Prior work on distributed garbage collection is vast; we discuss here the papers that are closely related to our work.  %Many of these collectors
%suffer from the need for centralization, global barriers,
%the migration of objects, or have inefficient time and space complexity.
The marking algorithm proposed by Hudak~\cite{Hudak:1982} requires a global barrier.
All local garbage
collectors coordinate to start the marking phase. Once the marking phase is over
in all the sites, then the sweeping phase continues. Along with the marking and sweeping overhead,
there are consistency issues in tracing based collectors~\cite{shapiro95}.

Ladin and Liskov~\cite{Ladin} compute reachability of objects in a highly
available centralized service. This algorithm is logically centralized but
physically replicated, hence achieving high availability and fault-tolerance.
All objects and tables are backed up in stable storage. Clocks are synchronized
and message delivery delay is bounded. These assumptions enable the centralized
service to build a consistent view of the distributed system. The centralized
service registers all the inter-space references and detects garbage using a standard
tracing algorithm. This algorithm has scalability issues due to the centralized
service. A heuristic-based algorithm by Le Fessant ~\cite{LeFessant} uses the minimal number
of inter-node references from a root to identify ``garbage suspects'' and then
verifies the suspects. The algorithm propagates marks from the references to all the
reachable objects. A cycle is detected when a remote object receives only its
own mark. The algorithm needs tight coupling between local
collectors and time complexity for the collection of cycles is not analyzed.

Collecting distributed garbage cycles by backtracking is proposed by Maheshwari
and Liskov~\cite{liskov97}. This algorithm first hypothesizes that objects are dead and
then tries to backtrack and prove whether that is true. The algorithm
needs to use an additional datastructure to store all inter-site references.
An alternative approach of distributed garbage collection by
migrating objects have been proposed by Maheshwari and Liskov~\cite{liskov95}. Both
of the algorithms use heuristics to detect garbage. The former one uses more
memory, the latter one increases the overhead by moving the objects between the
sites. Recently, Veiga et al.~\cite{Veiga} proposed an algorithm that uses
asynchronous local snapshots to identify global garbage. Cycle detection
messages (CDM) traverse the reference graph and gain information about the
graph. A drawback of this approach is that the algorithm doesn't work with the $\mathcal{CONGEST}$ model. %the growth of the messages is limited
%only by the size of the distributed system.
Any change to the snapshots
has to be updated by local mutators, forcing current global garbage collection
to quit. For a thorough understanding of the literature, we recommend reading~\cite{shapiro95, Abu}.


%\subsection{Distributed Garabge Collection = GC + Distributed Termination
%Detection}
%necessary in decentralized algorithm}

%Distributed termination detection(DTD) is a fundamental to distributed
%computation. The solution can be applied to any computation that tries to
%evaluate some stable property.
%Examples of such properties are
%deadlock detection, garbage detection etc. Once a system eventually enters into
%deadlock, it remains deadlocked. Once an object becomes garbage, it remains so.
%Detecting whether the stable property is reached is equivalent to termination in
%the DTD. The DTD can be formally described as a collection of processes
%communicating by messages. A process can be either passive or active. Active
%processes may send message while passive process will not send message. A passive
%process will be become an active when a message is received. When all the
%distributed processes are passive, the stable property is achieved. Stability
%here means no messages are in flight. The popular termination detection algorithms
%are ~\cite{dijk83,Tel88,dijk80,Misra82,Chandra90,nir}. Dijkstra and Scholten
%~\cite{dijk80} create spanning trees of the graph for detecting the termination
%detection. Our algorithm uses their solution to control the traversal and detect
%the termination.

%Tel et al showed that Distributed Termiantion Detection Algorithm (DTDA) can be
%derived from Garbage Collection Schemes.~\cite{Tel:1993}. Following Tel et al,
%Blackburn et al describes that reversing mapping is also
%possible~\cite{Blackburn:2001}. They also explained
%that any known DTDA with a centralized garbage collection will produce a
%distrbuted garbage collector (DGC). Although the literature clearly
%demonstrates the relationship between the fundamental problems of DTDA and DGC, cyclic
%reference counting-based DGC algorithm that contains this relation have never
%been explored. To our knowledge this is the first article that to do so.

%Most modern programming languages use heap and stack memory to run application.
%Objects are allocated in heap memory to increase the lifetime of the object.
%Managed runtime systems help developers to ignore the efficient deallocation of
%the objects. These managed runtime systems identify the heap allocated objects
%that are garbage and deallocate them. The relationship among various allocated
%objects in memory can be modeled as directed graph. Pointers that are in stack
%can hold reference to object in the heap. These pointers are generally referred
%as roots. Also, global pointers can also be considered as roots. In abstract,
%any pointer that does not reside in heap memory but holds reference to heap
%objects are called roots. Any objects in the heap are called nodes. These nodes
%are synonymous to regular graph nodes in graph theory. Any node can hold
%reference to any other node. These references are unidirectional link incident
%on two nodes. So they form a directed graph. The distinction between roots and
%nodes are any root can hold reference to nodes but nodes can only reference to
%other nodes. To map this into graph, we have two kind of vertices. Roots are
%special vertices that does not have any incoming link but an outgoing link.
%Nodes are regular vertex that can have incoming and outgoing links. A node is
%considered reachable if the node has a path from one of the root node to it.
%Unreachable nodes are called garbage and needs to be deleted for efficiency. The
%problem of garbage collection can be described as problem of finding all the
%garbage nodes in the reference graph and reclaim/delete them.

%It is shown in ~\cite{Tel:1993,Blackburn:2001} that there is
%strong relation between DTD and distributed garabge collection. Tel et al shown
%that a distributed termination detection algorithm can be derived from any
%distributed garabge collector~\cite{Tel:1993}. Blackburn et al, described that a
%distributed garbage collector is composite of distributed termination detection
%and a garbage collector ~\cite{Blackburn:2001}. Our algorithm indeed has a
%distributed termination detection algorithm in it along with a garbage
%collection algorithm.

%\subsection{Contributions}

\paragraph{Paper Outline:}
%The rest of the paper is organized as follows.
Section \ref{model} describes the garbage collection problem, constraints involved in the problem, and the model for which the algorithm is presented. Section \ref{single} explains the single collector version of the algorithm and provides correctness, time, and space guarantees.
Section \ref{multi} extends the results of Section \ref{single} to the multiple collector scenario. Finally, Section \ref{section:conclusions} concludes the paper with a short discussion. Some of the proofs and the pseudocode for the algorithms may be found in Appendix \ref{section:appendix}.



\section{Model and Preliminaries}
\label{model}
\paragraph{Distributed System:}
We consider a distributed system of nodes
%A distributed algorithm is one where
%self-stabilization problem where nodes
%the dead nodes identify that they are dead and
%delete themselves. Identifying a dead node by itself is not trivial in most
%cases. Each node can be assumed to
where each node operates independently and communicates with other nodes through
message passing. The underlying topology of the system is assumed to be arbitrary but connected. Each node has a queue to receive messages, and in response to a message a node can only read and write its own state and send additional messages. %A node can only send a
%message to its out-neighbors and parent of the node. Parent, a chosen node in
%$\Gamma_{in}$ from communication happened by our algorithm, .
We further assume that the nodes can communicate over
a reliable FIFO channel and that messages
are never lost, duplicated, or delivered out of order.
These properties make our system compatible with the $\mathcal{CONGEST}$ asynchronous network model with no failures ~\cite{congest}.
%Each site has memory where the objects reside. \todo{need precise definition}

\paragraph{Basic Reference Graph:}
We model the relationship among various objects and pointers in memory through a directed graph $G = (V, E)$, which we call a {\it reference graph}.
The graph $G$ has a special node $R$, which we call the {\it root}. Node $R$
represents
global and stack pointer variables, and
thus does not have any incoming edges.
Each node in $G$ is assumed to contain a unique ID. % to distinguish it within
%$G$. (For example, when each node of $G$ maps to a unique site in $H$, then ID of the site in $H$ can be used for the ID of the node in $G$.)
All adjacent nodes to a given node in $G$ are called $neighbors$, and denoted by $\Gamma$. The
$\inneighbors$ of a node $x\in G$ can be defined as the set of nodes whose outgoing edges
are incident on $x$, represented by $\Gamma_{in}(x)$. The $\outneighbors$ of $x$ can be defined as the set
of nodes whose incoming edges originate on $x$, represented by $\Gamma_{out}(x)$.
Note that each node $x\in G$ does not know $\Gamma_{in}(x)$ at any point in time.
%This constraint
%reduces memory complexity, thereby making
%the algorithm more practical and scalable. \todo{remove preceding sentence?}

\paragraph{Distributed Garbage Collection Problem:}
All nodes in $G$ can be classified as either {\em live} (i.e., not garbage) or {\em dead} (i.e., garbage) based on a property called
$reachability$. Live and dead nodes can be defined as below:


%$Live(x)= \exists y \in V \mid Reachable(y, x)$

%$Reachable(y, x) = x \in R \vee (Live(y) \wedge x \in \Gamma_{out}(y))$

$Reachable(y,x) = x \in \Gamma_{\rm out}(y) \lor (x \in \Gamma_{\rm out}(z) \mid Reachable(y,z))$

$Live(x) = Reachable(R,x)$

$Dead(x) = \neg Live(x)$

%From above definition, it is clearly that all root nodes are always reachable / live.
We allow the live portion
of $G$, denoted as $G'$, to be mutated while the algorithm is running, and we refer
to the source of these mutations as the {\em Adversary.}
The Adversary can create nodes and attach them to $G'$, create new edges between existing nodes of $G'$, or delete edges from $G'$. Moreover, the Adversary can perform multiple events (creation and deletion of edges)
simultaneously. The Adversary, however, can never mutate the dead portion of the graph $G'' = G\backslash G'$.
%Creation event represents creating directed
%edges. An edge can be only created between two live nodes in $G$. To create a
%node, an edge is created between a live node and non-existing node. During link
%creation, if the node is not available, the requested node is created and a
%directed edge is incident on newly created node. So the newly
%created node is always reachable. Deletion happens only for edges. Adversary can
%delete outgoing edge of a node $x$ only if $x$ is live . So $A$ cannot perform
%any mutation on dead nodes. These constraints create the following principles /
%axioms in the model.

\begin{axiom}[Immutable Dead Node]
 The Adversary cannot mutate a dead node.
 \label{ax:immut}
\end{axiom}

\begin{axiom}[Node Creation]
 All nodes are live when they are created by the Adversary.
\end{axiom}

From Axiom \ref{ax:immut}, it follows that a node that becomes dead will never
become live again.
%This property allows the problem to be solved in
%finite time. \todo{remove preceding sentence}
%The problem of distributed garbage collection can be modeled as
%detecting all dead nodes in dynamic directed graph $G$.

Each node experiencing deletion of an incoming edge has to determine whether it is still
live. If a node detects that it is dead then it must delete itself from the
graph $G$. %The problem of garbage collection is the process of
%identifying dead nodes in $G$ and deleting them.

\begin{definition}[Distributed Garbage Collection Problem]
%The problem of garbage collection is the process of
Identify the dead nodes in the reference graph $G$ and delete them.
\end{definition}

%We consider a distributed system where
%%A distributed algorithm is one where
%%self-stabilization problem where nodes
%%the dead nodes identify that they are dead and
%%delete themselves. Identifying a dead node by itself is not trivial in most
%%cases. Each node can be assumed to
%each node communicates with other nodes through
%messages. Each node has a queue to receive messages, and in response to a message a node can only read and write its own state and send additional messages. %A node can only send a
%%message to its out-neighbors and parent of the node. Parent, a chosen node in
%%$\Gamma_{in}$ from communication happened by our algorithm, .
%We further assume that the algorithm communicates over
%a reliable FIFO channel and that messages
%are never lost, duplicated, or delivered out of order.
%These properties make this algorithm compatible with the $\mathcal{CONGEST}$~\cite{congest} asynchronous network model with no failures.  \todo{talk also about fair execution here which is currently missing}
%%The $\mathcal{CONGEST}$ model adds the additional requirement that %messages are limited in size to $O (\log N)$. This limit is %already reached if unique object ID's are present in a message, as %they are of size $O (\log N)$.

%\subsection{Preliminaries}

\begin{figure}
\centering
%\begin{center}
\scalebox{0.6}[0.6]{% Graphic for TeX using PGF
% Title: /home/hkrish/podcpaper/distgc/completeabstract.dia
% Creator: Dia v0.97.2
% CreationDate: Wed Apr 27 12:51:38 2016
% For: hkrish
% \usepackage{tikz}
% The following commands are not supported in PSTricks at present
% We define them conditionally, so when they are implemented,
% this pgf file will use them.
\ifx\du\undefined
  \newlength{\du}
\fi
\setlength{\du}{15\unitlength}
\begin{tikzpicture}
\pgftransformxscale{1.000000}
\pgftransformyscale{-1.000000}
\definecolor{dialinecolor}{rgb}{0.000000, 0.000000, 0.000000}
\pgfsetstrokecolor{dialinecolor}
\definecolor{dialinecolor}{rgb}{1.000000, 1.000000, 1.000000}
\pgfsetfillcolor{dialinecolor}
\definecolor{dialinecolor}{rgb}{1.000000, 1.000000, 1.000000}
\pgfsetfillcolor{dialinecolor}
\fill (19.500000\du,13.250000\du)--(19.500000\du,15.150000\du)--(21.500000\du,15.150000\du)--(21.500000\du,13.250000\du)--cycle;
\pgfsetlinewidth{0.100000\du}
\pgfsetdash{}{0pt}
\pgfsetdash{}{0pt}
\pgfsetmiterjoin
\definecolor{dialinecolor}{rgb}{0.000000, 0.000000, 0.000000}
\pgfsetstrokecolor{dialinecolor}
\draw (19.500000\du,13.250000\du)--(19.500000\du,15.150000\du)--(21.500000\du,15.150000\du)--(21.500000\du,13.250000\du)--cycle;
% setfont left to latex
\definecolor{dialinecolor}{rgb}{0.000000, 0.000000, 0.000000}
\pgfsetstrokecolor{dialinecolor}
\node at (20.500000\du,14.395000\du){A};
\definecolor{dialinecolor}{rgb}{1.000000, 1.000000, 1.000000}
\pgfsetfillcolor{dialinecolor}
\pgfpathellipse{\pgfpoint{24.196664\du}{10.223322\du}}{\pgfpoint{1.053364\du}{0\du}}{\pgfpoint{0\du}{1.026682\du}}
\pgfusepath{fill}
\pgfsetlinewidth{0.100000\du}
\pgfsetdash{}{0pt}
\pgfsetdash{}{0pt}
\pgfsetmiterjoin
\definecolor{dialinecolor}{rgb}{0.000000, 0.000000, 0.000000}
\pgfsetstrokecolor{dialinecolor}
\pgfpathellipse{\pgfpoint{24.196664\du}{10.223322\du}}{\pgfpoint{1.053364\du}{0\du}}{\pgfpoint{0\du}{1.026682\du}}
\pgfusepath{stroke}
% setfont left to latex
\definecolor{dialinecolor}{rgb}{0.000000, 0.000000, 0.000000}
\pgfsetstrokecolor{dialinecolor}
\node at (24.196664\du,10.418322\du){C};
\definecolor{dialinecolor}{rgb}{1.000000, 1.000000, 1.000000}
\pgfsetfillcolor{dialinecolor}
\pgfpathellipse{\pgfpoint{20.453356\du}{8.846680\du}}{\pgfpoint{1.071256\du}{0\du}}{\pgfpoint{0\du}{1.044120\du}}
\pgfusepath{fill}
\pgfsetlinewidth{0.100000\du}
\pgfsetdash{}{0pt}
\pgfsetdash{}{0pt}
\pgfsetmiterjoin
\definecolor{dialinecolor}{rgb}{0.000000, 0.000000, 0.000000}
\pgfsetstrokecolor{dialinecolor}
\pgfpathellipse{\pgfpoint{20.453356\du}{8.846680\du}}{\pgfpoint{1.071256\du}{0\du}}{\pgfpoint{0\du}{1.044120\du}}
\pgfusepath{stroke}
% setfont left to latex
\definecolor{dialinecolor}{rgb}{0.000000, 0.000000, 0.000000}
\pgfsetstrokecolor{dialinecolor}
\node at (20.453356\du,9.041680\du){C'};
\definecolor{dialinecolor}{rgb}{1.000000, 1.000000, 1.000000}
\pgfsetfillcolor{dialinecolor}
\pgfpathellipse{\pgfpoint{26.803364\du}{16.366682\du}}{\pgfpoint{1.053364\du}{0\du}}{\pgfpoint{0\du}{1.026682\du}}
\pgfusepath{fill}
\pgfsetlinewidth{0.100000\du}
\pgfsetdash{}{0pt}
\pgfsetdash{}{0pt}
\pgfsetmiterjoin
\definecolor{dialinecolor}{rgb}{0.000000, 0.000000, 0.000000}
\pgfsetstrokecolor{dialinecolor}
\pgfpathellipse{\pgfpoint{26.803364\du}{16.366682\du}}{\pgfpoint{1.053364\du}{0\du}}{\pgfpoint{0\du}{1.026682\du}}
\pgfusepath{stroke}
% setfont left to latex
\definecolor{dialinecolor}{rgb}{0.000000, 0.000000, 0.000000}
\pgfsetstrokecolor{dialinecolor}
\node at (26.803364\du,16.561682\du){B};
\definecolor{dialinecolor}{rgb}{1.000000, 1.000000, 1.000000}
\pgfsetfillcolor{dialinecolor}
\pgfpathellipse{\pgfpoint{26.703395\du}{12.986725\du}}{\pgfpoint{1.068695\du}{0\du}}{\pgfpoint{0\du}{1.041625\du}}
\pgfusepath{fill}
\pgfsetlinewidth{0.100000\du}
\pgfsetdash{}{0pt}
\pgfsetdash{}{0pt}
\pgfsetmiterjoin
\definecolor{dialinecolor}{rgb}{0.000000, 0.000000, 0.000000}
\pgfsetstrokecolor{dialinecolor}
\pgfpathellipse{\pgfpoint{26.703395\du}{12.986725\du}}{\pgfpoint{1.068695\du}{0\du}}{\pgfpoint{0\du}{1.041625\du}}
\pgfusepath{stroke}
% setfont left to latex
\definecolor{dialinecolor}{rgb}{0.000000, 0.000000, 0.000000}
\pgfsetstrokecolor{dialinecolor}
\node at (26.703395\du,13.181725\du){B'};
\definecolor{dialinecolor}{rgb}{1.000000, 1.000000, 1.000000}
\pgfsetfillcolor{dialinecolor}
\pgfpathellipse{\pgfpoint{25.653364\du}{19.656682\du}}{\pgfpoint{1.053364\du}{0\du}}{\pgfpoint{0\du}{1.026682\du}}
\pgfusepath{fill}
\pgfsetlinewidth{0.100000\du}
\pgfsetdash{}{0pt}
\pgfsetdash{}{0pt}
\pgfsetmiterjoin
\definecolor{dialinecolor}{rgb}{0.000000, 0.000000, 0.000000}
\pgfsetstrokecolor{dialinecolor}
\pgfpathellipse{\pgfpoint{25.653364\du}{19.656682\du}}{\pgfpoint{1.053364\du}{0\du}}{\pgfpoint{0\du}{1.026682\du}}
\pgfusepath{stroke}
% setfont left to latex
\definecolor{dialinecolor}{rgb}{0.000000, 0.000000, 0.000000}
\pgfsetstrokecolor{dialinecolor}
\node at (25.653364\du,19.851682\du){D};
\definecolor{dialinecolor}{rgb}{1.000000, 1.000000, 1.000000}
\pgfsetfillcolor{dialinecolor}
\pgfpathellipse{\pgfpoint{19.803392\du}{19.226664\du}}{\pgfpoint{1.086792\du}{0\du}}{\pgfpoint{0\du}{1.059264\du}}
\pgfusepath{fill}
\pgfsetlinewidth{0.100000\du}
\pgfsetdash{}{0pt}
\pgfsetdash{}{0pt}
\pgfsetmiterjoin
\definecolor{dialinecolor}{rgb}{0.000000, 0.000000, 0.000000}
\pgfsetstrokecolor{dialinecolor}
\pgfpathellipse{\pgfpoint{19.803392\du}{19.226664\du}}{\pgfpoint{1.086792\du}{0\du}}{\pgfpoint{0\du}{1.059264\du}}
\pgfusepath{stroke}
% setfont left to latex
\definecolor{dialinecolor}{rgb}{0.000000, 0.000000, 0.000000}
\pgfsetstrokecolor{dialinecolor}
\node at (19.803392\du,19.421664\du){D'};
\pgfsetlinewidth{0.100000\du}
\pgfsetdash{}{0pt}
\pgfsetdash{}{0pt}
\pgfsetbuttcap
{
\definecolor{dialinecolor}{rgb}{0.000000, 0.000000, 0.000000}
\pgfsetfillcolor{dialinecolor}
% was here!!!
\pgfsetarrowsend{stealth}
\definecolor{dialinecolor}{rgb}{0.000000, 0.000000, 0.000000}
\pgfsetstrokecolor{dialinecolor}
\draw (21.408392\du,13.199631\du)--(23.451823\du,10.949296\du);
}
\pgfsetlinewidth{0.100000\du}
\pgfsetdash{}{0pt}
\pgfsetdash{}{0pt}
\pgfsetbuttcap
{
\definecolor{dialinecolor}{rgb}{0.000000, 0.000000, 0.000000}
\pgfsetfillcolor{dialinecolor}
% was here!!!
\pgfsetarrowsend{stealth}
\definecolor{dialinecolor}{rgb}{0.000000, 0.000000, 0.000000}
\pgfsetstrokecolor{dialinecolor}
\draw (21.549928\du,13.994653\du)--(25.606518\du,13.201255\du);
}
\pgfsetlinewidth{0.100000\du}
\pgfsetdash{}{0pt}
\pgfsetdash{}{0pt}
\pgfsetbuttcap
{
\definecolor{dialinecolor}{rgb}{0.000000, 0.000000, 0.000000}
\pgfsetfillcolor{dialinecolor}
% was here!!!
\pgfsetarrowsend{stealth}
\definecolor{dialinecolor}{rgb}{0.000000, 0.000000, 0.000000}
\pgfsetstrokecolor{dialinecolor}
\draw (21.549919\du,14.560893\du)--(25.762678\du,16.008963\du);
}
\pgfsetlinewidth{0.100000\du}
\pgfsetdash{}{0pt}
\pgfsetdash{}{0pt}
\pgfsetbuttcap
{
\definecolor{dialinecolor}{rgb}{0.000000, 0.000000, 0.000000}
\pgfsetfillcolor{dialinecolor}
% was here!!!
\pgfsetarrowsend{stealth}
\definecolor{dialinecolor}{rgb}{0.000000, 0.000000, 0.000000}
\pgfsetstrokecolor{dialinecolor}
\draw (21.444238\du,15.199814\du)--(24.905711\du,18.865023\du);
}
\pgfsetlinewidth{0.100000\du}
\pgfsetdash{}{0pt}
\pgfsetdash{}{0pt}
\pgfsetbuttcap
{
\definecolor{dialinecolor}{rgb}{0.000000, 0.000000, 0.000000}
\pgfsetfillcolor{dialinecolor}
% was here!!!
\pgfsetarrowsend{stealth}
\definecolor{dialinecolor}{rgb}{0.000000, 0.000000, 0.000000}
\pgfsetstrokecolor{dialinecolor}
\draw (20.361478\du,15.199565\du)--(19.944891\du,18.205623\du);
}
\pgfsetlinewidth{0.100000\du}
\pgfsetdash{}{0pt}
\pgfsetdash{}{0pt}
\pgfsetbuttcap
{
\definecolor{dialinecolor}{rgb}{0.000000, 0.000000, 0.000000}
\pgfsetfillcolor{dialinecolor}
% was here!!!
\pgfsetarrowsend{stealth}
\definecolor{dialinecolor}{rgb}{0.000000, 0.000000, 0.000000}
\pgfsetstrokecolor{dialinecolor}
\draw (20.491285\du,13.199847\du)--(20.462890\du,9.940935\du);
}
\pgfsetlinewidth{0.100000\du}
\pgfsetdash{{\pgflinewidth}{0.200000\du}}{0cm}
\pgfsetdash{{\pgflinewidth}{0.200000\du}}{0cm}
\pgfsetbuttcap
{
\definecolor{dialinecolor}{rgb}{0.000000, 0.000000, 0.000000}
\pgfsetfillcolor{dialinecolor}
% was here!!!
\pgfsetarrowsend{stealth}
\definecolor{dialinecolor}{rgb}{0.000000, 0.000000, 0.000000}
\pgfsetstrokecolor{dialinecolor}
\pgfpathmoveto{\pgfpoint{19.859953\du}{9.769223\du}}
\pgfpatharc{206}{156}{4.082526\du and 4.082526\du}
\pgfusepath{stroke}
}
\pgfsetlinewidth{0.100000\du}
\pgfsetdash{{\pgflinewidth}{0.200000\du}}{0cm}
\pgfsetdash{{\pgflinewidth}{0.200000\du}}{0cm}
\pgfsetbuttcap
{
\definecolor{dialinecolor}{rgb}{0.000000, 0.000000, 0.000000}
\pgfsetfillcolor{dialinecolor}
% was here!!!
\pgfsetarrowsend{stealth}
\definecolor{dialinecolor}{rgb}{0.000000, 0.000000, 0.000000}
\pgfsetstrokecolor{dialinecolor}
\pgfpathmoveto{\pgfpoint{25.649759\du}{12.639195\du}}
\pgfpatharc{283}{238}{5.494268\du and 5.494268\du}
\pgfusepath{stroke}
}
\pgfsetlinewidth{0.100000\du}
\pgfsetdash{{\pgflinewidth}{0.200000\du}}{0cm}
\pgfsetdash{{\pgflinewidth}{0.200000\du}}{0cm}
\pgfsetbuttcap
{
\definecolor{dialinecolor}{rgb}{0.000000, 0.000000, 0.000000}
\pgfsetfillcolor{dialinecolor}
% was here!!!
\pgfsetarrowsend{stealth}
\definecolor{dialinecolor}{rgb}{0.000000, 0.000000, 0.000000}
\pgfsetstrokecolor{dialinecolor}
\pgfpathmoveto{\pgfpoint{20.556576\du}{18.397407\du}}
\pgfpatharc{34}{-17}{3.719076\du and 3.719076\du}
\pgfusepath{stroke}
}
\pgfsetlinewidth{0.100000\du}
\pgfsetdash{}{0pt}
\pgfsetdash{}{0pt}
\pgfsetbuttcap
{
\definecolor{dialinecolor}{rgb}{0.000000, 0.000000, 0.000000}
\pgfsetfillcolor{dialinecolor}
% was here!!!
\pgfsetarrowsend{stealth}
\definecolor{dialinecolor}{rgb}{0.000000, 0.000000, 0.000000}
\pgfsetstrokecolor{dialinecolor}
\draw (19.825000\du,22.775000\du)--(19.810147\du,20.335952\du);
}
\pgfsetlinewidth{0.100000\du}
\pgfsetdash{}{0pt}
\pgfsetdash{}{0pt}
\pgfsetbuttcap
{
\definecolor{dialinecolor}{rgb}{0.000000, 0.000000, 0.000000}
\pgfsetfillcolor{dialinecolor}
% was here!!!
\definecolor{dialinecolor}{rgb}{0.000000, 0.000000, 0.000000}
\pgfsetstrokecolor{dialinecolor}
\draw (18.750000\du,22.800000\du)--(20.900000\du,22.750000\du);
}
\pgfsetlinewidth{0.100000\du}
\pgfsetdash{{\pgflinewidth}{0.200000\du}}{0cm}
\pgfsetdash{{\pgflinewidth}{0.200000\du}}{0cm}
\pgfsetbuttcap
{
\definecolor{dialinecolor}{rgb}{0.000000, 0.000000, 0.000000}
\pgfsetfillcolor{dialinecolor}
% was here!!!
\pgfsetarrowsend{stealth}
\definecolor{dialinecolor}{rgb}{0.000000, 0.000000, 0.000000}
\pgfsetstrokecolor{dialinecolor}
\draw (19.675000\du,5.625000\du)--(20.196247\du,7.782488\du);
}
\pgfsetlinewidth{0.100000\du}
\pgfsetdash{}{0pt}
\pgfsetdash{}{0pt}
\pgfsetbuttcap
{
\definecolor{dialinecolor}{rgb}{0.000000, 0.000000, 0.000000}
\pgfsetfillcolor{dialinecolor}
% was here!!!
\pgfsetarrowsend{stealth}
\definecolor{dialinecolor}{rgb}{0.000000, 0.000000, 0.000000}
\pgfsetstrokecolor{dialinecolor}
\draw (26.400550\du,22.235550\du)--(25.653364\du,20.683364\du);
}
\pgfsetlinewidth{0.100000\du}
\pgfsetdash{}{0pt}
\pgfsetdash{}{0pt}
\pgfsetbuttcap
{
\definecolor{dialinecolor}{rgb}{0.000000, 0.000000, 0.000000}
\pgfsetfillcolor{dialinecolor}
% was here!!!
\definecolor{dialinecolor}{rgb}{0.000000, 0.000000, 0.000000}
\pgfsetstrokecolor{dialinecolor}
\draw (25.351100\du,22.571100\du)--(27.450000\du,21.900000\du);
}
\pgfsetlinewidth{0.100000\du}
\pgfsetdash{{\pgflinewidth}{0.200000\du}}{0cm}
\pgfsetdash{{\pgflinewidth}{0.200000\du}}{0cm}
\pgfsetbuttcap
{
\definecolor{dialinecolor}{rgb}{0.000000, 0.000000, 0.000000}
\pgfsetfillcolor{dialinecolor}
% was here!!!
\pgfsetarrowsend{stealth}
\definecolor{dialinecolor}{rgb}{0.000000, 0.000000, 0.000000}
\pgfsetstrokecolor{dialinecolor}
\draw (25.850000\du,7.100000\du)--(24.703038\du,9.266728\du);
}
\pgfsetlinewidth{0.100000\du}
\pgfsetdash{}{0pt}
\pgfsetdash{}{0pt}
\pgfsetbuttcap
{
\definecolor{dialinecolor}{rgb}{0.000000, 0.000000, 0.000000}
\pgfsetfillcolor{dialinecolor}
% was here!!!
\definecolor{dialinecolor}{rgb}{0.000000, 0.000000, 0.000000}
\pgfsetstrokecolor{dialinecolor}
\draw (18.350000\du,5.900000\du)--(21.000000\du,5.350000\du);
}
\pgfsetlinewidth{0.100000\du}
\pgfsetdash{}{0pt}
\pgfsetdash{}{0pt}
\pgfsetbuttcap
{
\definecolor{dialinecolor}{rgb}{0.000000, 0.000000, 0.000000}
\pgfsetfillcolor{dialinecolor}
% was here!!!
\definecolor{dialinecolor}{rgb}{0.000000, 0.000000, 0.000000}
\pgfsetstrokecolor{dialinecolor}
\draw (24.850000\du,6.500000\du)--(26.850000\du,7.700000\du);
}
\definecolor{dialinecolor}{rgb}{1.000000, 1.000000, 1.000000}
\pgfsetfillcolor{dialinecolor}
\pgfpathellipse{\pgfpoint{14.443364\du}{14.681682\du}}{\pgfpoint{1.056857\du}{0\du}}{\pgfpoint{0\du}{1.030087\du}}
\pgfusepath{fill}
\pgfsetlinewidth{0.100000\du}
\pgfsetdash{}{0pt}
\pgfsetdash{}{0pt}
\pgfsetmiterjoin
\definecolor{dialinecolor}{rgb}{0.000000, 0.000000, 0.000000}
\pgfsetstrokecolor{dialinecolor}
\pgfpathellipse{\pgfpoint{14.443364\du}{14.681682\du}}{\pgfpoint{1.056857\du}{0\du}}{\pgfpoint{0\du}{1.030087\du}}
\pgfusepath{stroke}
% setfont left to latex
\definecolor{dialinecolor}{rgb}{0.000000, 0.000000, 0.000000}
\pgfsetstrokecolor{dialinecolor}
\node at (14.443364\du,14.876682\du){E'};
\pgfsetlinewidth{0.100000\du}
\pgfsetdash{}{0pt}
\pgfsetdash{}{0pt}
\pgfsetbuttcap
{
\definecolor{dialinecolor}{rgb}{0.000000, 0.000000, 0.000000}
\pgfsetfillcolor{dialinecolor}
% was here!!!
\pgfsetarrowsend{stealth}
\definecolor{dialinecolor}{rgb}{0.000000, 0.000000, 0.000000}
\pgfsetstrokecolor{dialinecolor}
\draw (11.820574\du,14.477234\du)--(13.340717\du,14.595730\du);
}
\pgfsetlinewidth{0.100000\du}
\pgfsetdash{}{0pt}
\pgfsetdash{}{0pt}
\pgfsetbuttcap
{
\definecolor{dialinecolor}{rgb}{0.000000, 0.000000, 0.000000}
\pgfsetfillcolor{dialinecolor}
% was here!!!
\definecolor{dialinecolor}{rgb}{0.000000, 0.000000, 0.000000}
\pgfsetstrokecolor{dialinecolor}
\draw (11.850000\du,15.700000\du)--(11.791149\du,13.254468\du);
}
\definecolor{dialinecolor}{rgb}{1.000000, 1.000000, 1.000000}
\pgfsetfillcolor{dialinecolor}
\pgfpathellipse{\pgfpoint{15.193403\du}{19.483402\du}}{\pgfpoint{1.053364\du}{0\du}}{\pgfpoint{0\du}{1.026682\du}}
\pgfusepath{fill}
\pgfsetlinewidth{0.100000\du}
\pgfsetdash{}{0pt}
\pgfsetdash{}{0pt}
\pgfsetmiterjoin
\definecolor{dialinecolor}{rgb}{0.000000, 0.000000, 0.000000}
\pgfsetstrokecolor{dialinecolor}
\pgfpathellipse{\pgfpoint{15.193403\du}{19.483402\du}}{\pgfpoint{1.053364\du}{0\du}}{\pgfpoint{0\du}{1.026682\du}}
\pgfusepath{stroke}
% setfont left to latex
\definecolor{dialinecolor}{rgb}{0.000000, 0.000000, 0.000000}
\pgfsetstrokecolor{dialinecolor}
\node at (15.193403\du,19.678402\du){E};
\pgfsetlinewidth{0.100000\du}
\pgfsetdash{}{0pt}
\pgfsetdash{}{0pt}
\pgfsetbuttcap
{
\definecolor{dialinecolor}{rgb}{0.000000, 0.000000, 0.000000}
\pgfsetfillcolor{dialinecolor}
% was here!!!
\pgfsetarrowsend{stealth}
\definecolor{dialinecolor}{rgb}{0.000000, 0.000000, 0.000000}
\pgfsetstrokecolor{dialinecolor}
\draw (13.675020\du,21.850860\du)--(14.608440\du,20.395475\du);
}
\pgfsetlinewidth{0.100000\du}
\pgfsetdash{}{0pt}
\pgfsetdash{}{0pt}
\pgfsetbuttcap
{
\definecolor{dialinecolor}{rgb}{0.000000, 0.000000, 0.000000}
\pgfsetfillcolor{dialinecolor}
% was here!!!
\definecolor{dialinecolor}{rgb}{0.000000, 0.000000, 0.000000}
\pgfsetstrokecolor{dialinecolor}
\draw (14.500039\du,22.701720\du)--(12.850000\du,21.000000\du);
}
\pgfsetlinewidth{0.100000\du}
\pgfsetdash{{\pgflinewidth}{0.200000\du}}{0cm}
\pgfsetdash{{\pgflinewidth}{0.200000\du}}{0cm}
\pgfsetbuttcap
{
\definecolor{dialinecolor}{rgb}{0.000000, 0.000000, 0.000000}
\pgfsetfillcolor{dialinecolor}
% was here!!!
\pgfsetarrowsend{stealth}
\definecolor{dialinecolor}{rgb}{0.000000, 0.000000, 0.000000}
\pgfsetstrokecolor{dialinecolor}
\draw (19.449774\du,14.283524\du)--(15.546822\du,14.593924\du);
}
\pgfsetlinewidth{0.100000\du}
\pgfsetdash{{\pgflinewidth}{0.200000\du}}{0cm}
\pgfsetdash{{\pgflinewidth}{0.200000\du}}{0cm}
\pgfsetbuttcap
{
\definecolor{dialinecolor}{rgb}{0.000000, 0.000000, 0.000000}
\pgfsetfillcolor{dialinecolor}
% was here!!!
\pgfsetarrowsend{stealth}
\definecolor{dialinecolor}{rgb}{0.000000, 0.000000, 0.000000}
\pgfsetstrokecolor{dialinecolor}
\pgfpathmoveto{\pgfpoint{15.434900\du}{15.159237\du}}
\pgfpatharc{110}{64}{5.114357\du and 5.114357\du}
\pgfusepath{stroke}
}
\pgfsetlinewidth{0.100000\du}
\pgfsetdash{{\pgflinewidth}{0.200000\du}}{0cm}
\pgfsetdash{{\pgflinewidth}{0.200000\du}}{0cm}
\pgfsetbuttcap
{
\definecolor{dialinecolor}{rgb}{0.000000, 0.000000, 0.000000}
\pgfsetfillcolor{dialinecolor}
% was here!!!
\pgfsetarrowsend{stealth}
\definecolor{dialinecolor}{rgb}{0.000000, 0.000000, 0.000000}
\pgfsetstrokecolor{dialinecolor}
\draw (19.500000\du,15.150000\du)--(15.961462\du,18.710563\du);
}
\definecolor{dialinecolor}{rgb}{1.000000, 1.000000, 1.000000}
\pgfsetfillcolor{dialinecolor}
\pgfpathellipse{\pgfpoint{15.443403\du}{11.233402\du}}{\pgfpoint{1.056857\du}{0\du}}{\pgfpoint{0\du}{1.030087\du}}
\pgfusepath{fill}
\pgfsetlinewidth{0.100000\du}
\pgfsetdash{}{0pt}
\pgfsetdash{}{0pt}
\pgfsetmiterjoin
\definecolor{dialinecolor}{rgb}{0.000000, 0.000000, 0.000000}
\pgfsetstrokecolor{dialinecolor}
\pgfpathellipse{\pgfpoint{15.443403\du}{11.233402\du}}{\pgfpoint{1.056857\du}{0\du}}{\pgfpoint{0\du}{1.030087\du}}
\pgfusepath{stroke}
% setfont left to latex
\definecolor{dialinecolor}{rgb}{0.000000, 0.000000, 0.000000}
\pgfsetstrokecolor{dialinecolor}
\node at (15.443403\du,11.428402\du){F};
\pgfsetlinewidth{0.100000\du}
\pgfsetdash{}{0pt}
\pgfsetdash{}{0pt}
\pgfsetbuttcap
{
\definecolor{dialinecolor}{rgb}{0.000000, 0.000000, 0.000000}
\pgfsetfillcolor{dialinecolor}
% was here!!!
\pgfsetarrowsend{stealth}
\definecolor{dialinecolor}{rgb}{0.000000, 0.000000, 0.000000}
\pgfsetstrokecolor{dialinecolor}
\draw (13.520594\du,8.703094\du)--(14.784081\du,10.365771\du);
}
\pgfsetlinewidth{0.100000\du}
\pgfsetdash{}{0pt}
\pgfsetdash{}{0pt}
\pgfsetbuttcap
{
\definecolor{dialinecolor}{rgb}{0.000000, 0.000000, 0.000000}
\pgfsetfillcolor{dialinecolor}
% was here!!!
\definecolor{dialinecolor}{rgb}{0.000000, 0.000000, 0.000000}
\pgfsetstrokecolor{dialinecolor}
\draw (12.450000\du,9.600000\du)--(14.591188\du,7.806188\du);
}
\pgfsetlinewidth{0.100000\du}
\pgfsetdash{{\pgflinewidth}{0.200000\du}}{0cm}
\pgfsetdash{{\pgflinewidth}{0.200000\du}}{0cm}
\pgfsetbuttcap
{
\definecolor{dialinecolor}{rgb}{0.000000, 0.000000, 0.000000}
\pgfsetfillcolor{dialinecolor}
% was here!!!
\pgfsetarrowsend{stealth}
\definecolor{dialinecolor}{rgb}{0.000000, 0.000000, 0.000000}
\pgfsetstrokecolor{dialinecolor}
\draw (16.392132\du,11.790001\du)--(19.450040\du,13.584011\du);
}
\end{tikzpicture}
}
\caption{We depict all the ways in which an initiator node, $A$, can
be connected to the graph. Circles 
represent sets of nodes.
Dotted lines represent one or more non-strong paths. Solid lines represent one or more
strong paths.
A T-shaped end-point indicates the root, R.
If C${}^\prime$, D${}^\prime$, E${}^\prime$ and F are empty sets, A is garbage, otherwise 
it is not. }
\label{fig:completeabstract}
%\end{center}
\end{figure}

\begin{comment}
\begin{figure}[h!]
\centering
\begin{subfigure}[t]{0.45\linewidth}
%\begin{center}
\scalebox{0.6}[0.6]{% Graphic for TeX using PGF
% Title: /home/hkrish/podcpaper/distgc/completeabstract.dia
% Creator: Dia v0.97.2
% CreationDate: Wed Apr 27 12:51:38 2016
% For: hkrish
% \usepackage{tikz}
% The following commands are not supported in PSTricks at present
% We define them conditionally, so when they are implemented,
% this pgf file will use them.
\ifx\du\undefined
  \newlength{\du}
\fi
\setlength{\du}{15\unitlength}
\begin{tikzpicture}
\pgftransformxscale{1.000000}
\pgftransformyscale{-1.000000}
\definecolor{dialinecolor}{rgb}{0.000000, 0.000000, 0.000000}
\pgfsetstrokecolor{dialinecolor}
\definecolor{dialinecolor}{rgb}{1.000000, 1.000000, 1.000000}
\pgfsetfillcolor{dialinecolor}
\definecolor{dialinecolor}{rgb}{1.000000, 1.000000, 1.000000}
\pgfsetfillcolor{dialinecolor}
\fill (19.500000\du,13.250000\du)--(19.500000\du,15.150000\du)--(21.500000\du,15.150000\du)--(21.500000\du,13.250000\du)--cycle;
\pgfsetlinewidth{0.100000\du}
\pgfsetdash{}{0pt}
\pgfsetdash{}{0pt}
\pgfsetmiterjoin
\definecolor{dialinecolor}{rgb}{0.000000, 0.000000, 0.000000}
\pgfsetstrokecolor{dialinecolor}
\draw (19.500000\du,13.250000\du)--(19.500000\du,15.150000\du)--(21.500000\du,15.150000\du)--(21.500000\du,13.250000\du)--cycle;
% setfont left to latex
\definecolor{dialinecolor}{rgb}{0.000000, 0.000000, 0.000000}
\pgfsetstrokecolor{dialinecolor}
\node at (20.500000\du,14.395000\du){A};
\definecolor{dialinecolor}{rgb}{1.000000, 1.000000, 1.000000}
\pgfsetfillcolor{dialinecolor}
\pgfpathellipse{\pgfpoint{24.196664\du}{10.223322\du}}{\pgfpoint{1.053364\du}{0\du}}{\pgfpoint{0\du}{1.026682\du}}
\pgfusepath{fill}
\pgfsetlinewidth{0.100000\du}
\pgfsetdash{}{0pt}
\pgfsetdash{}{0pt}
\pgfsetmiterjoin
\definecolor{dialinecolor}{rgb}{0.000000, 0.000000, 0.000000}
\pgfsetstrokecolor{dialinecolor}
\pgfpathellipse{\pgfpoint{24.196664\du}{10.223322\du}}{\pgfpoint{1.053364\du}{0\du}}{\pgfpoint{0\du}{1.026682\du}}
\pgfusepath{stroke}
% setfont left to latex
\definecolor{dialinecolor}{rgb}{0.000000, 0.000000, 0.000000}
\pgfsetstrokecolor{dialinecolor}
\node at (24.196664\du,10.418322\du){C};
\definecolor{dialinecolor}{rgb}{1.000000, 1.000000, 1.000000}
\pgfsetfillcolor{dialinecolor}
\pgfpathellipse{\pgfpoint{20.453356\du}{8.846680\du}}{\pgfpoint{1.071256\du}{0\du}}{\pgfpoint{0\du}{1.044120\du}}
\pgfusepath{fill}
\pgfsetlinewidth{0.100000\du}
\pgfsetdash{}{0pt}
\pgfsetdash{}{0pt}
\pgfsetmiterjoin
\definecolor{dialinecolor}{rgb}{0.000000, 0.000000, 0.000000}
\pgfsetstrokecolor{dialinecolor}
\pgfpathellipse{\pgfpoint{20.453356\du}{8.846680\du}}{\pgfpoint{1.071256\du}{0\du}}{\pgfpoint{0\du}{1.044120\du}}
\pgfusepath{stroke}
% setfont left to latex
\definecolor{dialinecolor}{rgb}{0.000000, 0.000000, 0.000000}
\pgfsetstrokecolor{dialinecolor}
\node at (20.453356\du,9.041680\du){C'};
\definecolor{dialinecolor}{rgb}{1.000000, 1.000000, 1.000000}
\pgfsetfillcolor{dialinecolor}
\pgfpathellipse{\pgfpoint{26.803364\du}{16.366682\du}}{\pgfpoint{1.053364\du}{0\du}}{\pgfpoint{0\du}{1.026682\du}}
\pgfusepath{fill}
\pgfsetlinewidth{0.100000\du}
\pgfsetdash{}{0pt}
\pgfsetdash{}{0pt}
\pgfsetmiterjoin
\definecolor{dialinecolor}{rgb}{0.000000, 0.000000, 0.000000}
\pgfsetstrokecolor{dialinecolor}
\pgfpathellipse{\pgfpoint{26.803364\du}{16.366682\du}}{\pgfpoint{1.053364\du}{0\du}}{\pgfpoint{0\du}{1.026682\du}}
\pgfusepath{stroke}
% setfont left to latex
\definecolor{dialinecolor}{rgb}{0.000000, 0.000000, 0.000000}
\pgfsetstrokecolor{dialinecolor}
\node at (26.803364\du,16.561682\du){B};
\definecolor{dialinecolor}{rgb}{1.000000, 1.000000, 1.000000}
\pgfsetfillcolor{dialinecolor}
\pgfpathellipse{\pgfpoint{26.703395\du}{12.986725\du}}{\pgfpoint{1.068695\du}{0\du}}{\pgfpoint{0\du}{1.041625\du}}
\pgfusepath{fill}
\pgfsetlinewidth{0.100000\du}
\pgfsetdash{}{0pt}
\pgfsetdash{}{0pt}
\pgfsetmiterjoin
\definecolor{dialinecolor}{rgb}{0.000000, 0.000000, 0.000000}
\pgfsetstrokecolor{dialinecolor}
\pgfpathellipse{\pgfpoint{26.703395\du}{12.986725\du}}{\pgfpoint{1.068695\du}{0\du}}{\pgfpoint{0\du}{1.041625\du}}
\pgfusepath{stroke}
% setfont left to latex
\definecolor{dialinecolor}{rgb}{0.000000, 0.000000, 0.000000}
\pgfsetstrokecolor{dialinecolor}
\node at (26.703395\du,13.181725\du){B'};
\definecolor{dialinecolor}{rgb}{1.000000, 1.000000, 1.000000}
\pgfsetfillcolor{dialinecolor}
\pgfpathellipse{\pgfpoint{25.653364\du}{19.656682\du}}{\pgfpoint{1.053364\du}{0\du}}{\pgfpoint{0\du}{1.026682\du}}
\pgfusepath{fill}
\pgfsetlinewidth{0.100000\du}
\pgfsetdash{}{0pt}
\pgfsetdash{}{0pt}
\pgfsetmiterjoin
\definecolor{dialinecolor}{rgb}{0.000000, 0.000000, 0.000000}
\pgfsetstrokecolor{dialinecolor}
\pgfpathellipse{\pgfpoint{25.653364\du}{19.656682\du}}{\pgfpoint{1.053364\du}{0\du}}{\pgfpoint{0\du}{1.026682\du}}
\pgfusepath{stroke}
% setfont left to latex
\definecolor{dialinecolor}{rgb}{0.000000, 0.000000, 0.000000}
\pgfsetstrokecolor{dialinecolor}
\node at (25.653364\du,19.851682\du){D};
\definecolor{dialinecolor}{rgb}{1.000000, 1.000000, 1.000000}
\pgfsetfillcolor{dialinecolor}
\pgfpathellipse{\pgfpoint{19.803392\du}{19.226664\du}}{\pgfpoint{1.086792\du}{0\du}}{\pgfpoint{0\du}{1.059264\du}}
\pgfusepath{fill}
\pgfsetlinewidth{0.100000\du}
\pgfsetdash{}{0pt}
\pgfsetdash{}{0pt}
\pgfsetmiterjoin
\definecolor{dialinecolor}{rgb}{0.000000, 0.000000, 0.000000}
\pgfsetstrokecolor{dialinecolor}
\pgfpathellipse{\pgfpoint{19.803392\du}{19.226664\du}}{\pgfpoint{1.086792\du}{0\du}}{\pgfpoint{0\du}{1.059264\du}}
\pgfusepath{stroke}
% setfont left to latex
\definecolor{dialinecolor}{rgb}{0.000000, 0.000000, 0.000000}
\pgfsetstrokecolor{dialinecolor}
\node at (19.803392\du,19.421664\du){D'};
\pgfsetlinewidth{0.100000\du}
\pgfsetdash{}{0pt}
\pgfsetdash{}{0pt}
\pgfsetbuttcap
{
\definecolor{dialinecolor}{rgb}{0.000000, 0.000000, 0.000000}
\pgfsetfillcolor{dialinecolor}
% was here!!!
\pgfsetarrowsend{stealth}
\definecolor{dialinecolor}{rgb}{0.000000, 0.000000, 0.000000}
\pgfsetstrokecolor{dialinecolor}
\draw (21.408392\du,13.199631\du)--(23.451823\du,10.949296\du);
}
\pgfsetlinewidth{0.100000\du}
\pgfsetdash{}{0pt}
\pgfsetdash{}{0pt}
\pgfsetbuttcap
{
\definecolor{dialinecolor}{rgb}{0.000000, 0.000000, 0.000000}
\pgfsetfillcolor{dialinecolor}
% was here!!!
\pgfsetarrowsend{stealth}
\definecolor{dialinecolor}{rgb}{0.000000, 0.000000, 0.000000}
\pgfsetstrokecolor{dialinecolor}
\draw (21.549928\du,13.994653\du)--(25.606518\du,13.201255\du);
}
\pgfsetlinewidth{0.100000\du}
\pgfsetdash{}{0pt}
\pgfsetdash{}{0pt}
\pgfsetbuttcap
{
\definecolor{dialinecolor}{rgb}{0.000000, 0.000000, 0.000000}
\pgfsetfillcolor{dialinecolor}
% was here!!!
\pgfsetarrowsend{stealth}
\definecolor{dialinecolor}{rgb}{0.000000, 0.000000, 0.000000}
\pgfsetstrokecolor{dialinecolor}
\draw (21.549919\du,14.560893\du)--(25.762678\du,16.008963\du);
}
\pgfsetlinewidth{0.100000\du}
\pgfsetdash{}{0pt}
\pgfsetdash{}{0pt}
\pgfsetbuttcap
{
\definecolor{dialinecolor}{rgb}{0.000000, 0.000000, 0.000000}
\pgfsetfillcolor{dialinecolor}
% was here!!!
\pgfsetarrowsend{stealth}
\definecolor{dialinecolor}{rgb}{0.000000, 0.000000, 0.000000}
\pgfsetstrokecolor{dialinecolor}
\draw (21.444238\du,15.199814\du)--(24.905711\du,18.865023\du);
}
\pgfsetlinewidth{0.100000\du}
\pgfsetdash{}{0pt}
\pgfsetdash{}{0pt}
\pgfsetbuttcap
{
\definecolor{dialinecolor}{rgb}{0.000000, 0.000000, 0.000000}
\pgfsetfillcolor{dialinecolor}
% was here!!!
\pgfsetarrowsend{stealth}
\definecolor{dialinecolor}{rgb}{0.000000, 0.000000, 0.000000}
\pgfsetstrokecolor{dialinecolor}
\draw (20.361478\du,15.199565\du)--(19.944891\du,18.205623\du);
}
\pgfsetlinewidth{0.100000\du}
\pgfsetdash{}{0pt}
\pgfsetdash{}{0pt}
\pgfsetbuttcap
{
\definecolor{dialinecolor}{rgb}{0.000000, 0.000000, 0.000000}
\pgfsetfillcolor{dialinecolor}
% was here!!!
\pgfsetarrowsend{stealth}
\definecolor{dialinecolor}{rgb}{0.000000, 0.000000, 0.000000}
\pgfsetstrokecolor{dialinecolor}
\draw (20.491285\du,13.199847\du)--(20.462890\du,9.940935\du);
}
\pgfsetlinewidth{0.100000\du}
\pgfsetdash{{\pgflinewidth}{0.200000\du}}{0cm}
\pgfsetdash{{\pgflinewidth}{0.200000\du}}{0cm}
\pgfsetbuttcap
{
\definecolor{dialinecolor}{rgb}{0.000000, 0.000000, 0.000000}
\pgfsetfillcolor{dialinecolor}
% was here!!!
\pgfsetarrowsend{stealth}
\definecolor{dialinecolor}{rgb}{0.000000, 0.000000, 0.000000}
\pgfsetstrokecolor{dialinecolor}
\pgfpathmoveto{\pgfpoint{19.859953\du}{9.769223\du}}
\pgfpatharc{206}{156}{4.082526\du and 4.082526\du}
\pgfusepath{stroke}
}
\pgfsetlinewidth{0.100000\du}
\pgfsetdash{{\pgflinewidth}{0.200000\du}}{0cm}
\pgfsetdash{{\pgflinewidth}{0.200000\du}}{0cm}
\pgfsetbuttcap
{
\definecolor{dialinecolor}{rgb}{0.000000, 0.000000, 0.000000}
\pgfsetfillcolor{dialinecolor}
% was here!!!
\pgfsetarrowsend{stealth}
\definecolor{dialinecolor}{rgb}{0.000000, 0.000000, 0.000000}
\pgfsetstrokecolor{dialinecolor}
\pgfpathmoveto{\pgfpoint{25.649759\du}{12.639195\du}}
\pgfpatharc{283}{238}{5.494268\du and 5.494268\du}
\pgfusepath{stroke}
}
\pgfsetlinewidth{0.100000\du}
\pgfsetdash{{\pgflinewidth}{0.200000\du}}{0cm}
\pgfsetdash{{\pgflinewidth}{0.200000\du}}{0cm}
\pgfsetbuttcap
{
\definecolor{dialinecolor}{rgb}{0.000000, 0.000000, 0.000000}
\pgfsetfillcolor{dialinecolor}
% was here!!!
\pgfsetarrowsend{stealth}
\definecolor{dialinecolor}{rgb}{0.000000, 0.000000, 0.000000}
\pgfsetstrokecolor{dialinecolor}
\pgfpathmoveto{\pgfpoint{20.556576\du}{18.397407\du}}
\pgfpatharc{34}{-17}{3.719076\du and 3.719076\du}
\pgfusepath{stroke}
}
\pgfsetlinewidth{0.100000\du}
\pgfsetdash{}{0pt}
\pgfsetdash{}{0pt}
\pgfsetbuttcap
{
\definecolor{dialinecolor}{rgb}{0.000000, 0.000000, 0.000000}
\pgfsetfillcolor{dialinecolor}
% was here!!!
\pgfsetarrowsend{stealth}
\definecolor{dialinecolor}{rgb}{0.000000, 0.000000, 0.000000}
\pgfsetstrokecolor{dialinecolor}
\draw (19.825000\du,22.775000\du)--(19.810147\du,20.335952\du);
}
\pgfsetlinewidth{0.100000\du}
\pgfsetdash{}{0pt}
\pgfsetdash{}{0pt}
\pgfsetbuttcap
{
\definecolor{dialinecolor}{rgb}{0.000000, 0.000000, 0.000000}
\pgfsetfillcolor{dialinecolor}
% was here!!!
\definecolor{dialinecolor}{rgb}{0.000000, 0.000000, 0.000000}
\pgfsetstrokecolor{dialinecolor}
\draw (18.750000\du,22.800000\du)--(20.900000\du,22.750000\du);
}
\pgfsetlinewidth{0.100000\du}
\pgfsetdash{{\pgflinewidth}{0.200000\du}}{0cm}
\pgfsetdash{{\pgflinewidth}{0.200000\du}}{0cm}
\pgfsetbuttcap
{
\definecolor{dialinecolor}{rgb}{0.000000, 0.000000, 0.000000}
\pgfsetfillcolor{dialinecolor}
% was here!!!
\pgfsetarrowsend{stealth}
\definecolor{dialinecolor}{rgb}{0.000000, 0.000000, 0.000000}
\pgfsetstrokecolor{dialinecolor}
\draw (19.675000\du,5.625000\du)--(20.196247\du,7.782488\du);
}
\pgfsetlinewidth{0.100000\du}
\pgfsetdash{}{0pt}
\pgfsetdash{}{0pt}
\pgfsetbuttcap
{
\definecolor{dialinecolor}{rgb}{0.000000, 0.000000, 0.000000}
\pgfsetfillcolor{dialinecolor}
% was here!!!
\pgfsetarrowsend{stealth}
\definecolor{dialinecolor}{rgb}{0.000000, 0.000000, 0.000000}
\pgfsetstrokecolor{dialinecolor}
\draw (26.400550\du,22.235550\du)--(25.653364\du,20.683364\du);
}
\pgfsetlinewidth{0.100000\du}
\pgfsetdash{}{0pt}
\pgfsetdash{}{0pt}
\pgfsetbuttcap
{
\definecolor{dialinecolor}{rgb}{0.000000, 0.000000, 0.000000}
\pgfsetfillcolor{dialinecolor}
% was here!!!
\definecolor{dialinecolor}{rgb}{0.000000, 0.000000, 0.000000}
\pgfsetstrokecolor{dialinecolor}
\draw (25.351100\du,22.571100\du)--(27.450000\du,21.900000\du);
}
\pgfsetlinewidth{0.100000\du}
\pgfsetdash{{\pgflinewidth}{0.200000\du}}{0cm}
\pgfsetdash{{\pgflinewidth}{0.200000\du}}{0cm}
\pgfsetbuttcap
{
\definecolor{dialinecolor}{rgb}{0.000000, 0.000000, 0.000000}
\pgfsetfillcolor{dialinecolor}
% was here!!!
\pgfsetarrowsend{stealth}
\definecolor{dialinecolor}{rgb}{0.000000, 0.000000, 0.000000}
\pgfsetstrokecolor{dialinecolor}
\draw (25.850000\du,7.100000\du)--(24.703038\du,9.266728\du);
}
\pgfsetlinewidth{0.100000\du}
\pgfsetdash{}{0pt}
\pgfsetdash{}{0pt}
\pgfsetbuttcap
{
\definecolor{dialinecolor}{rgb}{0.000000, 0.000000, 0.000000}
\pgfsetfillcolor{dialinecolor}
% was here!!!
\definecolor{dialinecolor}{rgb}{0.000000, 0.000000, 0.000000}
\pgfsetstrokecolor{dialinecolor}
\draw (18.350000\du,5.900000\du)--(21.000000\du,5.350000\du);
}
\pgfsetlinewidth{0.100000\du}
\pgfsetdash{}{0pt}
\pgfsetdash{}{0pt}
\pgfsetbuttcap
{
\definecolor{dialinecolor}{rgb}{0.000000, 0.000000, 0.000000}
\pgfsetfillcolor{dialinecolor}
% was here!!!
\definecolor{dialinecolor}{rgb}{0.000000, 0.000000, 0.000000}
\pgfsetstrokecolor{dialinecolor}
\draw (24.850000\du,6.500000\du)--(26.850000\du,7.700000\du);
}
\definecolor{dialinecolor}{rgb}{1.000000, 1.000000, 1.000000}
\pgfsetfillcolor{dialinecolor}
\pgfpathellipse{\pgfpoint{14.443364\du}{14.681682\du}}{\pgfpoint{1.056857\du}{0\du}}{\pgfpoint{0\du}{1.030087\du}}
\pgfusepath{fill}
\pgfsetlinewidth{0.100000\du}
\pgfsetdash{}{0pt}
\pgfsetdash{}{0pt}
\pgfsetmiterjoin
\definecolor{dialinecolor}{rgb}{0.000000, 0.000000, 0.000000}
\pgfsetstrokecolor{dialinecolor}
\pgfpathellipse{\pgfpoint{14.443364\du}{14.681682\du}}{\pgfpoint{1.056857\du}{0\du}}{\pgfpoint{0\du}{1.030087\du}}
\pgfusepath{stroke}
% setfont left to latex
\definecolor{dialinecolor}{rgb}{0.000000, 0.000000, 0.000000}
\pgfsetstrokecolor{dialinecolor}
\node at (14.443364\du,14.876682\du){E'};
\pgfsetlinewidth{0.100000\du}
\pgfsetdash{}{0pt}
\pgfsetdash{}{0pt}
\pgfsetbuttcap
{
\definecolor{dialinecolor}{rgb}{0.000000, 0.000000, 0.000000}
\pgfsetfillcolor{dialinecolor}
% was here!!!
\pgfsetarrowsend{stealth}
\definecolor{dialinecolor}{rgb}{0.000000, 0.000000, 0.000000}
\pgfsetstrokecolor{dialinecolor}
\draw (11.820574\du,14.477234\du)--(13.340717\du,14.595730\du);
}
\pgfsetlinewidth{0.100000\du}
\pgfsetdash{}{0pt}
\pgfsetdash{}{0pt}
\pgfsetbuttcap
{
\definecolor{dialinecolor}{rgb}{0.000000, 0.000000, 0.000000}
\pgfsetfillcolor{dialinecolor}
% was here!!!
\definecolor{dialinecolor}{rgb}{0.000000, 0.000000, 0.000000}
\pgfsetstrokecolor{dialinecolor}
\draw (11.850000\du,15.700000\du)--(11.791149\du,13.254468\du);
}
\definecolor{dialinecolor}{rgb}{1.000000, 1.000000, 1.000000}
\pgfsetfillcolor{dialinecolor}
\pgfpathellipse{\pgfpoint{15.193403\du}{19.483402\du}}{\pgfpoint{1.053364\du}{0\du}}{\pgfpoint{0\du}{1.026682\du}}
\pgfusepath{fill}
\pgfsetlinewidth{0.100000\du}
\pgfsetdash{}{0pt}
\pgfsetdash{}{0pt}
\pgfsetmiterjoin
\definecolor{dialinecolor}{rgb}{0.000000, 0.000000, 0.000000}
\pgfsetstrokecolor{dialinecolor}
\pgfpathellipse{\pgfpoint{15.193403\du}{19.483402\du}}{\pgfpoint{1.053364\du}{0\du}}{\pgfpoint{0\du}{1.026682\du}}
\pgfusepath{stroke}
% setfont left to latex
\definecolor{dialinecolor}{rgb}{0.000000, 0.000000, 0.000000}
\pgfsetstrokecolor{dialinecolor}
\node at (15.193403\du,19.678402\du){E};
\pgfsetlinewidth{0.100000\du}
\pgfsetdash{}{0pt}
\pgfsetdash{}{0pt}
\pgfsetbuttcap
{
\definecolor{dialinecolor}{rgb}{0.000000, 0.000000, 0.000000}
\pgfsetfillcolor{dialinecolor}
% was here!!!
\pgfsetarrowsend{stealth}
\definecolor{dialinecolor}{rgb}{0.000000, 0.000000, 0.000000}
\pgfsetstrokecolor{dialinecolor}
\draw (13.675020\du,21.850860\du)--(14.608440\du,20.395475\du);
}
\pgfsetlinewidth{0.100000\du}
\pgfsetdash{}{0pt}
\pgfsetdash{}{0pt}
\pgfsetbuttcap
{
\definecolor{dialinecolor}{rgb}{0.000000, 0.000000, 0.000000}
\pgfsetfillcolor{dialinecolor}
% was here!!!
\definecolor{dialinecolor}{rgb}{0.000000, 0.000000, 0.000000}
\pgfsetstrokecolor{dialinecolor}
\draw (14.500039\du,22.701720\du)--(12.850000\du,21.000000\du);
}
\pgfsetlinewidth{0.100000\du}
\pgfsetdash{{\pgflinewidth}{0.200000\du}}{0cm}
\pgfsetdash{{\pgflinewidth}{0.200000\du}}{0cm}
\pgfsetbuttcap
{
\definecolor{dialinecolor}{rgb}{0.000000, 0.000000, 0.000000}
\pgfsetfillcolor{dialinecolor}
% was here!!!
\pgfsetarrowsend{stealth}
\definecolor{dialinecolor}{rgb}{0.000000, 0.000000, 0.000000}
\pgfsetstrokecolor{dialinecolor}
\draw (19.449774\du,14.283524\du)--(15.546822\du,14.593924\du);
}
\pgfsetlinewidth{0.100000\du}
\pgfsetdash{{\pgflinewidth}{0.200000\du}}{0cm}
\pgfsetdash{{\pgflinewidth}{0.200000\du}}{0cm}
\pgfsetbuttcap
{
\definecolor{dialinecolor}{rgb}{0.000000, 0.000000, 0.000000}
\pgfsetfillcolor{dialinecolor}
% was here!!!
\pgfsetarrowsend{stealth}
\definecolor{dialinecolor}{rgb}{0.000000, 0.000000, 0.000000}
\pgfsetstrokecolor{dialinecolor}
\pgfpathmoveto{\pgfpoint{15.434900\du}{15.159237\du}}
\pgfpatharc{110}{64}{5.114357\du and 5.114357\du}
\pgfusepath{stroke}
}
\pgfsetlinewidth{0.100000\du}
\pgfsetdash{{\pgflinewidth}{0.200000\du}}{0cm}
\pgfsetdash{{\pgflinewidth}{0.200000\du}}{0cm}
\pgfsetbuttcap
{
\definecolor{dialinecolor}{rgb}{0.000000, 0.000000, 0.000000}
\pgfsetfillcolor{dialinecolor}
% was here!!!
\pgfsetarrowsend{stealth}
\definecolor{dialinecolor}{rgb}{0.000000, 0.000000, 0.000000}
\pgfsetstrokecolor{dialinecolor}
\draw (19.500000\du,15.150000\du)--(15.961462\du,18.710563\du);
}
\definecolor{dialinecolor}{rgb}{1.000000, 1.000000, 1.000000}
\pgfsetfillcolor{dialinecolor}
\pgfpathellipse{\pgfpoint{15.443403\du}{11.233402\du}}{\pgfpoint{1.056857\du}{0\du}}{\pgfpoint{0\du}{1.030087\du}}
\pgfusepath{fill}
\pgfsetlinewidth{0.100000\du}
\pgfsetdash{}{0pt}
\pgfsetdash{}{0pt}
\pgfsetmiterjoin
\definecolor{dialinecolor}{rgb}{0.000000, 0.000000, 0.000000}
\pgfsetstrokecolor{dialinecolor}
\pgfpathellipse{\pgfpoint{15.443403\du}{11.233402\du}}{\pgfpoint{1.056857\du}{0\du}}{\pgfpoint{0\du}{1.030087\du}}
\pgfusepath{stroke}
% setfont left to latex
\definecolor{dialinecolor}{rgb}{0.000000, 0.000000, 0.000000}
\pgfsetstrokecolor{dialinecolor}
\node at (15.443403\du,11.428402\du){F};
\pgfsetlinewidth{0.100000\du}
\pgfsetdash{}{0pt}
\pgfsetdash{}{0pt}
\pgfsetbuttcap
{
\definecolor{dialinecolor}{rgb}{0.000000, 0.000000, 0.000000}
\pgfsetfillcolor{dialinecolor}
% was here!!!
\pgfsetarrowsend{stealth}
\definecolor{dialinecolor}{rgb}{0.000000, 0.000000, 0.000000}
\pgfsetstrokecolor{dialinecolor}
\draw (13.520594\du,8.703094\du)--(14.784081\du,10.365771\du);
}
\pgfsetlinewidth{0.100000\du}
\pgfsetdash{}{0pt}
\pgfsetdash{}{0pt}
\pgfsetbuttcap
{
\definecolor{dialinecolor}{rgb}{0.000000, 0.000000, 0.000000}
\pgfsetfillcolor{dialinecolor}
% was here!!!
\definecolor{dialinecolor}{rgb}{0.000000, 0.000000, 0.000000}
\pgfsetstrokecolor{dialinecolor}
\draw (12.450000\du,9.600000\du)--(14.591188\du,7.806188\du);
}
\pgfsetlinewidth{0.100000\du}
\pgfsetdash{{\pgflinewidth}{0.200000\du}}{0cm}
\pgfsetdash{{\pgflinewidth}{0.200000\du}}{0cm}
\pgfsetbuttcap
{
\definecolor{dialinecolor}{rgb}{0.000000, 0.000000, 0.000000}
\pgfsetfillcolor{dialinecolor}
% was here!!!
\pgfsetarrowsend{stealth}
\definecolor{dialinecolor}{rgb}{0.000000, 0.000000, 0.000000}
\pgfsetstrokecolor{dialinecolor}
\draw (16.392132\du,11.790001\du)--(19.450040\du,13.584011\du);
}
\end{tikzpicture}
}
\caption{We depict all the ways in which an initiator node, $A$, can
be connected to the graph. Circles 
represent sets of nodes.
Dotted lines represent one or more non-strong paths. Solid lines represent one or more
strong paths.
A T-shaped end-point indicates the root, R.
If C', D', E' and F are empty sets, A is garbage, otherwise 
it is not. }
\label{fig:completeabstract}
%\end{center}
\end{subfigure}
%\hfill
\begin{subfigure}[t]{0.45\linewidth}
%\begin{center}
\scalebox{0.6}[0.6]{% Graphic for TeX using PGF
% Title: C:\Users\Hari\Pictures\Diagram2.dia
% Creator: Dia v0.97.2
% CreationDate: Wed May 11 10:06:32 2016
% For: Hari
% \usepackage{tikz}
% The following commands are not supported in PSTricks at present
% We define them conditionally, so when they are implemented,
% this pgf file will use them.
\ifx\du\undefined
  \newlength{\du}
\fi
\setlength{\du}{15\unitlength}
\begin{tikzpicture}
\pgftransformxscale{1.000000}
\pgftransformyscale{-1.000000}
\definecolor{dialinecolor}{rgb}{0.000000, 0.000000, 0.000000}
\pgfsetstrokecolor{dialinecolor}
\definecolor{dialinecolor}{rgb}{1.000000, 1.000000, 1.000000}
\pgfsetfillcolor{dialinecolor}
\definecolor{dialinecolor}{rgb}{1.000000, 1.000000, 1.000000}
\pgfsetfillcolor{dialinecolor}
\pgfpathellipse{\pgfpoint{14.956636\du}{4.226682\du}}{\pgfpoint{3.143364\du}{0\du}}{\pgfpoint{0\du}{1.876682\du}}
\pgfusepath{fill}
\pgfsetlinewidth{0.100000\du}
\pgfsetdash{}{0pt}
\pgfsetdash{}{0pt}
\pgfsetmiterjoin
\definecolor{dialinecolor}{rgb}{0.000000, 0.000000, 0.000000}
\pgfsetstrokecolor{dialinecolor}
\pgfpathellipse{\pgfpoint{14.956636\du}{4.226682\du}}{\pgfpoint{3.143364\du}{0\du}}{\pgfpoint{0\du}{1.876682\du}}
\pgfusepath{stroke}
% setfont left to latex
\definecolor{dialinecolor}{rgb}{0.000000, 0.000000, 0.000000}
\pgfsetstrokecolor{dialinecolor}
\node at (14.956636\du,4.466682\du){};
% setfont left to latex
\definecolor{dialinecolor}{rgb}{0.000000, 0.000000, 0.000000}
\pgfsetstrokecolor{dialinecolor}
\node[anchor=west] at (12.441636\du,4.351682\du){\textbf{Phantomization}};
\definecolor{dialinecolor}{rgb}{1.000000, 1.000000, 1.000000}
\pgfsetfillcolor{dialinecolor}
\pgfpathellipse{\pgfpoint{18.668364\du}{10.980046\du}}{\pgfpoint{2.178364\du}{0\du}}{\pgfpoint{0\du}{1.576682\du}}
\pgfusepath{fill}
\pgfsetlinewidth{0.100000\du}
\pgfsetdash{}{0pt}
\pgfsetdash{}{0pt}
\pgfsetmiterjoin
\definecolor{dialinecolor}{rgb}{0.000000, 0.000000, 0.000000}
\pgfsetstrokecolor{dialinecolor}
\pgfpathellipse{\pgfpoint{18.668364\du}{10.980046\du}}{\pgfpoint{2.178364\du}{0\du}}{\pgfpoint{0\du}{1.576682\du}}
\pgfusepath{stroke}
% setfont left to latex
\definecolor{dialinecolor}{rgb}{0.000000, 0.000000, 0.000000}
\pgfsetstrokecolor{dialinecolor}
\node at (18.668364\du,11.220046\du){\textbf{Recovery}};
% setfont left to latex
\definecolor{dialinecolor}{rgb}{0.000000, 0.000000, 0.000000}
\pgfsetstrokecolor{dialinecolor}
\node[anchor=west] at (9.818364\du,11.005046\du){};
\definecolor{dialinecolor}{rgb}{1.000000, 1.000000, 1.000000}
\pgfsetfillcolor{dialinecolor}
\pgfpathellipse{\pgfpoint{10.268364\du}{18.130046\du}}{\pgfpoint{2.178364\du}{0\du}}{\pgfpoint{0\du}{1.576682\du}}
\pgfusepath{fill}
\pgfsetlinewidth{0.100000\du}
\pgfsetdash{}{0pt}
\pgfsetdash{}{0pt}
\pgfsetmiterjoin
\definecolor{dialinecolor}{rgb}{0.000000, 0.000000, 0.000000}
\pgfsetstrokecolor{dialinecolor}
\pgfpathellipse{\pgfpoint{10.268364\du}{18.130046\du}}{\pgfpoint{2.178364\du}{0\du}}{\pgfpoint{0\du}{1.576682\du}}
\pgfusepath{stroke}
% setfont left to latex
\definecolor{dialinecolor}{rgb}{0.000000, 0.000000, 0.000000}
\pgfsetstrokecolor{dialinecolor}
\node at (10.268364\du,18.370046\du){\textbf{Build}};
\definecolor{dialinecolor}{rgb}{1.000000, 1.000000, 1.000000}
\pgfsetfillcolor{dialinecolor}
\pgfpathellipse{\pgfpoint{25.768364\du}{18.380046\du}}{\pgfpoint{2.579644\du}{0\du}}{\pgfpoint{0\du}{1.867125\du}}
\pgfusepath{fill}
\pgfsetlinewidth{0.100000\du}
\pgfsetdash{}{0pt}
\pgfsetdash{}{0pt}
\pgfsetmiterjoin
\definecolor{dialinecolor}{rgb}{0.000000, 0.000000, 0.000000}
\pgfsetstrokecolor{dialinecolor}
\pgfpathellipse{\pgfpoint{25.768364\du}{18.380046\du}}{\pgfpoint{2.579644\du}{0\du}}{\pgfpoint{0\du}{1.867125\du}}
\pgfusepath{stroke}
% setfont left to latex
\definecolor{dialinecolor}{rgb}{0.000000, 0.000000, 0.000000}
\pgfsetstrokecolor{dialinecolor}
\node at (25.768364\du,18.620046\du){\textbf{Deletion}};
\pgfsetlinewidth{0.100000\du}
\pgfsetdash{}{0pt}
\pgfsetdash{}{0pt}
\pgfsetbuttcap
{
\definecolor{dialinecolor}{rgb}{0.000000, 0.000000, 0.000000}
\pgfsetfillcolor{dialinecolor}
% was here!!!
\pgfsetarrowsstart{stealth}
\definecolor{dialinecolor}{rgb}{0.000000, 0.000000, 0.000000}
\pgfsetstrokecolor{dialinecolor}
\pgfpathmoveto{\pgfpoint{19.896530\du}{9.625575\du}}
\pgfpatharc{29}{-64}{3.869885\du and 3.869885\du}
\pgfusepath{stroke}
}
\pgfsetlinewidth{0.100000\du}
\pgfsetdash{}{0pt}
\pgfsetdash{}{0pt}
\pgfsetbuttcap
{
\definecolor{dialinecolor}{rgb}{0.000000, 0.000000, 0.000000}
\pgfsetfillcolor{dialinecolor}
% was here!!!
\pgfsetarrowsend{stealth}
\definecolor{dialinecolor}{rgb}{0.000000, 0.000000, 0.000000}
\pgfsetstrokecolor{dialinecolor}
\pgfpathmoveto{\pgfpoint{13.834722\du}{6.026196\du}}
\pgfpatharc{210}{187}{27.410428\du and 27.410428\du}
\pgfusepath{stroke}
}
\pgfsetlinewidth{0.100000\du}
\pgfsetdash{}{0pt}
\pgfsetdash{}{0pt}
\pgfsetbuttcap
{
\definecolor{dialinecolor}{rgb}{0.000000, 0.000000, 0.000000}
\pgfsetfillcolor{dialinecolor}
% was here!!!
\pgfsetarrowsend{stealth}
\definecolor{dialinecolor}{rgb}{0.000000, 0.000000, 0.000000}
\pgfsetstrokecolor{dialinecolor}
\pgfpathmoveto{\pgfpoint{16.759366\du}{11.816219\du}}
\pgfpatharc{243}{216}{15.710313\du and 15.710313\du}
\pgfusepath{stroke}
}
\pgfsetlinewidth{0.100000\du}
\pgfsetdash{}{0pt}
\pgfsetdash{}{0pt}
\pgfsetbuttcap
{
\definecolor{dialinecolor}{rgb}{0.000000, 0.000000, 0.000000}
\pgfsetfillcolor{dialinecolor}
% was here!!!
\pgfsetarrowsstart{stealth}
\definecolor{dialinecolor}{rgb}{0.000000, 0.000000, 0.000000}
\pgfsetstrokecolor{dialinecolor}
\pgfpathmoveto{\pgfpoint{25.439190\du}{16.478320\du}}
\pgfpatharc{344}{292}{7.799925\du and 7.799925\du}
\pgfusepath{stroke}
}
% setfont left to latex
\definecolor{dialinecolor}{rgb}{0.000000, 0.000000, 0.000000}
\pgfsetstrokecolor{dialinecolor}
\node[anchor=west] at (4.370000\du,7.853364\du){\textbf{If Build Set is non-empty}};
% setfont left to latex
\definecolor{dialinecolor}{rgb}{0.000000, 0.000000, 0.000000}
\pgfsetstrokecolor{dialinecolor}
\node[anchor=west] at (19.920000\du,5.603364\du){\textbf{If Build Set is empty}};
% setfont left to latex
\definecolor{dialinecolor}{rgb}{0.000000, 0.000000, 0.000000}
\pgfsetstrokecolor{dialinecolor}
\node[anchor=west] at (13.170000\du,15.503364\du){\textbf{If Recovery set is non-empty}};
% setfont left to latex
\definecolor{dialinecolor}{rgb}{0.000000, 0.000000, 0.000000}
\pgfsetstrokecolor{dialinecolor}
\node[anchor=west] at (22.920000\du,12.053364\du){\textbf{If Recovery set is emtpy}};
\end{tikzpicture}
}
\caption{The above figure depicts the state/phase transitions performed by initiator in the algorithm.
}
\label{fig:state}
%\end{center}

\end{subfigure}
\caption{}
\end{figure}
\end{comment}
\begin{comment}

\begin{figure}
\begin{center}
\scalebox{0.6}[0.6]{% Graphic for TeX using PGF
% Title: /home/hkrish/podcpaper/distgc/completeabstract.dia
% Creator: Dia v0.97.2
% CreationDate: Wed Apr 27 12:51:38 2016
% For: hkrish
% \usepackage{tikz}
% The following commands are not supported in PSTricks at present
% We define them conditionally, so when they are implemented,
% this pgf file will use them.
\ifx\du\undefined
  \newlength{\du}
\fi
\setlength{\du}{15\unitlength}
\begin{tikzpicture}
\pgftransformxscale{1.000000}
\pgftransformyscale{-1.000000}
\definecolor{dialinecolor}{rgb}{0.000000, 0.000000, 0.000000}
\pgfsetstrokecolor{dialinecolor}
\definecolor{dialinecolor}{rgb}{1.000000, 1.000000, 1.000000}
\pgfsetfillcolor{dialinecolor}
\definecolor{dialinecolor}{rgb}{1.000000, 1.000000, 1.000000}
\pgfsetfillcolor{dialinecolor}
\fill (19.500000\du,13.250000\du)--(19.500000\du,15.150000\du)--(21.500000\du,15.150000\du)--(21.500000\du,13.250000\du)--cycle;
\pgfsetlinewidth{0.100000\du}
\pgfsetdash{}{0pt}
\pgfsetdash{}{0pt}
\pgfsetmiterjoin
\definecolor{dialinecolor}{rgb}{0.000000, 0.000000, 0.000000}
\pgfsetstrokecolor{dialinecolor}
\draw (19.500000\du,13.250000\du)--(19.500000\du,15.150000\du)--(21.500000\du,15.150000\du)--(21.500000\du,13.250000\du)--cycle;
% setfont left to latex
\definecolor{dialinecolor}{rgb}{0.000000, 0.000000, 0.000000}
\pgfsetstrokecolor{dialinecolor}
\node at (20.500000\du,14.395000\du){A};
\definecolor{dialinecolor}{rgb}{1.000000, 1.000000, 1.000000}
\pgfsetfillcolor{dialinecolor}
\pgfpathellipse{\pgfpoint{24.196664\du}{10.223322\du}}{\pgfpoint{1.053364\du}{0\du}}{\pgfpoint{0\du}{1.026682\du}}
\pgfusepath{fill}
\pgfsetlinewidth{0.100000\du}
\pgfsetdash{}{0pt}
\pgfsetdash{}{0pt}
\pgfsetmiterjoin
\definecolor{dialinecolor}{rgb}{0.000000, 0.000000, 0.000000}
\pgfsetstrokecolor{dialinecolor}
\pgfpathellipse{\pgfpoint{24.196664\du}{10.223322\du}}{\pgfpoint{1.053364\du}{0\du}}{\pgfpoint{0\du}{1.026682\du}}
\pgfusepath{stroke}
% setfont left to latex
\definecolor{dialinecolor}{rgb}{0.000000, 0.000000, 0.000000}
\pgfsetstrokecolor{dialinecolor}
\node at (24.196664\du,10.418322\du){C};
\definecolor{dialinecolor}{rgb}{1.000000, 1.000000, 1.000000}
\pgfsetfillcolor{dialinecolor}
\pgfpathellipse{\pgfpoint{20.453356\du}{8.846680\du}}{\pgfpoint{1.071256\du}{0\du}}{\pgfpoint{0\du}{1.044120\du}}
\pgfusepath{fill}
\pgfsetlinewidth{0.100000\du}
\pgfsetdash{}{0pt}
\pgfsetdash{}{0pt}
\pgfsetmiterjoin
\definecolor{dialinecolor}{rgb}{0.000000, 0.000000, 0.000000}
\pgfsetstrokecolor{dialinecolor}
\pgfpathellipse{\pgfpoint{20.453356\du}{8.846680\du}}{\pgfpoint{1.071256\du}{0\du}}{\pgfpoint{0\du}{1.044120\du}}
\pgfusepath{stroke}
% setfont left to latex
\definecolor{dialinecolor}{rgb}{0.000000, 0.000000, 0.000000}
\pgfsetstrokecolor{dialinecolor}
\node at (20.453356\du,9.041680\du){C'};
\definecolor{dialinecolor}{rgb}{1.000000, 1.000000, 1.000000}
\pgfsetfillcolor{dialinecolor}
\pgfpathellipse{\pgfpoint{26.803364\du}{16.366682\du}}{\pgfpoint{1.053364\du}{0\du}}{\pgfpoint{0\du}{1.026682\du}}
\pgfusepath{fill}
\pgfsetlinewidth{0.100000\du}
\pgfsetdash{}{0pt}
\pgfsetdash{}{0pt}
\pgfsetmiterjoin
\definecolor{dialinecolor}{rgb}{0.000000, 0.000000, 0.000000}
\pgfsetstrokecolor{dialinecolor}
\pgfpathellipse{\pgfpoint{26.803364\du}{16.366682\du}}{\pgfpoint{1.053364\du}{0\du}}{\pgfpoint{0\du}{1.026682\du}}
\pgfusepath{stroke}
% setfont left to latex
\definecolor{dialinecolor}{rgb}{0.000000, 0.000000, 0.000000}
\pgfsetstrokecolor{dialinecolor}
\node at (26.803364\du,16.561682\du){B};
\definecolor{dialinecolor}{rgb}{1.000000, 1.000000, 1.000000}
\pgfsetfillcolor{dialinecolor}
\pgfpathellipse{\pgfpoint{26.703395\du}{12.986725\du}}{\pgfpoint{1.068695\du}{0\du}}{\pgfpoint{0\du}{1.041625\du}}
\pgfusepath{fill}
\pgfsetlinewidth{0.100000\du}
\pgfsetdash{}{0pt}
\pgfsetdash{}{0pt}
\pgfsetmiterjoin
\definecolor{dialinecolor}{rgb}{0.000000, 0.000000, 0.000000}
\pgfsetstrokecolor{dialinecolor}
\pgfpathellipse{\pgfpoint{26.703395\du}{12.986725\du}}{\pgfpoint{1.068695\du}{0\du}}{\pgfpoint{0\du}{1.041625\du}}
\pgfusepath{stroke}
% setfont left to latex
\definecolor{dialinecolor}{rgb}{0.000000, 0.000000, 0.000000}
\pgfsetstrokecolor{dialinecolor}
\node at (26.703395\du,13.181725\du){B'};
\definecolor{dialinecolor}{rgb}{1.000000, 1.000000, 1.000000}
\pgfsetfillcolor{dialinecolor}
\pgfpathellipse{\pgfpoint{25.653364\du}{19.656682\du}}{\pgfpoint{1.053364\du}{0\du}}{\pgfpoint{0\du}{1.026682\du}}
\pgfusepath{fill}
\pgfsetlinewidth{0.100000\du}
\pgfsetdash{}{0pt}
\pgfsetdash{}{0pt}
\pgfsetmiterjoin
\definecolor{dialinecolor}{rgb}{0.000000, 0.000000, 0.000000}
\pgfsetstrokecolor{dialinecolor}
\pgfpathellipse{\pgfpoint{25.653364\du}{19.656682\du}}{\pgfpoint{1.053364\du}{0\du}}{\pgfpoint{0\du}{1.026682\du}}
\pgfusepath{stroke}
% setfont left to latex
\definecolor{dialinecolor}{rgb}{0.000000, 0.000000, 0.000000}
\pgfsetstrokecolor{dialinecolor}
\node at (25.653364\du,19.851682\du){D};
\definecolor{dialinecolor}{rgb}{1.000000, 1.000000, 1.000000}
\pgfsetfillcolor{dialinecolor}
\pgfpathellipse{\pgfpoint{19.803392\du}{19.226664\du}}{\pgfpoint{1.086792\du}{0\du}}{\pgfpoint{0\du}{1.059264\du}}
\pgfusepath{fill}
\pgfsetlinewidth{0.100000\du}
\pgfsetdash{}{0pt}
\pgfsetdash{}{0pt}
\pgfsetmiterjoin
\definecolor{dialinecolor}{rgb}{0.000000, 0.000000, 0.000000}
\pgfsetstrokecolor{dialinecolor}
\pgfpathellipse{\pgfpoint{19.803392\du}{19.226664\du}}{\pgfpoint{1.086792\du}{0\du}}{\pgfpoint{0\du}{1.059264\du}}
\pgfusepath{stroke}
% setfont left to latex
\definecolor{dialinecolor}{rgb}{0.000000, 0.000000, 0.000000}
\pgfsetstrokecolor{dialinecolor}
\node at (19.803392\du,19.421664\du){D'};
\pgfsetlinewidth{0.100000\du}
\pgfsetdash{}{0pt}
\pgfsetdash{}{0pt}
\pgfsetbuttcap
{
\definecolor{dialinecolor}{rgb}{0.000000, 0.000000, 0.000000}
\pgfsetfillcolor{dialinecolor}
% was here!!!
\pgfsetarrowsend{stealth}
\definecolor{dialinecolor}{rgb}{0.000000, 0.000000, 0.000000}
\pgfsetstrokecolor{dialinecolor}
\draw (21.408392\du,13.199631\du)--(23.451823\du,10.949296\du);
}
\pgfsetlinewidth{0.100000\du}
\pgfsetdash{}{0pt}
\pgfsetdash{}{0pt}
\pgfsetbuttcap
{
\definecolor{dialinecolor}{rgb}{0.000000, 0.000000, 0.000000}
\pgfsetfillcolor{dialinecolor}
% was here!!!
\pgfsetarrowsend{stealth}
\definecolor{dialinecolor}{rgb}{0.000000, 0.000000, 0.000000}
\pgfsetstrokecolor{dialinecolor}
\draw (21.549928\du,13.994653\du)--(25.606518\du,13.201255\du);
}
\pgfsetlinewidth{0.100000\du}
\pgfsetdash{}{0pt}
\pgfsetdash{}{0pt}
\pgfsetbuttcap
{
\definecolor{dialinecolor}{rgb}{0.000000, 0.000000, 0.000000}
\pgfsetfillcolor{dialinecolor}
% was here!!!
\pgfsetarrowsend{stealth}
\definecolor{dialinecolor}{rgb}{0.000000, 0.000000, 0.000000}
\pgfsetstrokecolor{dialinecolor}
\draw (21.549919\du,14.560893\du)--(25.762678\du,16.008963\du);
}
\pgfsetlinewidth{0.100000\du}
\pgfsetdash{}{0pt}
\pgfsetdash{}{0pt}
\pgfsetbuttcap
{
\definecolor{dialinecolor}{rgb}{0.000000, 0.000000, 0.000000}
\pgfsetfillcolor{dialinecolor}
% was here!!!
\pgfsetarrowsend{stealth}
\definecolor{dialinecolor}{rgb}{0.000000, 0.000000, 0.000000}
\pgfsetstrokecolor{dialinecolor}
\draw (21.444238\du,15.199814\du)--(24.905711\du,18.865023\du);
}
\pgfsetlinewidth{0.100000\du}
\pgfsetdash{}{0pt}
\pgfsetdash{}{0pt}
\pgfsetbuttcap
{
\definecolor{dialinecolor}{rgb}{0.000000, 0.000000, 0.000000}
\pgfsetfillcolor{dialinecolor}
% was here!!!
\pgfsetarrowsend{stealth}
\definecolor{dialinecolor}{rgb}{0.000000, 0.000000, 0.000000}
\pgfsetstrokecolor{dialinecolor}
\draw (20.361478\du,15.199565\du)--(19.944891\du,18.205623\du);
}
\pgfsetlinewidth{0.100000\du}
\pgfsetdash{}{0pt}
\pgfsetdash{}{0pt}
\pgfsetbuttcap
{
\definecolor{dialinecolor}{rgb}{0.000000, 0.000000, 0.000000}
\pgfsetfillcolor{dialinecolor}
% was here!!!
\pgfsetarrowsend{stealth}
\definecolor{dialinecolor}{rgb}{0.000000, 0.000000, 0.000000}
\pgfsetstrokecolor{dialinecolor}
\draw (20.491285\du,13.199847\du)--(20.462890\du,9.940935\du);
}
\pgfsetlinewidth{0.100000\du}
\pgfsetdash{{\pgflinewidth}{0.200000\du}}{0cm}
\pgfsetdash{{\pgflinewidth}{0.200000\du}}{0cm}
\pgfsetbuttcap
{
\definecolor{dialinecolor}{rgb}{0.000000, 0.000000, 0.000000}
\pgfsetfillcolor{dialinecolor}
% was here!!!
\pgfsetarrowsend{stealth}
\definecolor{dialinecolor}{rgb}{0.000000, 0.000000, 0.000000}
\pgfsetstrokecolor{dialinecolor}
\pgfpathmoveto{\pgfpoint{19.859953\du}{9.769223\du}}
\pgfpatharc{206}{156}{4.082526\du and 4.082526\du}
\pgfusepath{stroke}
}
\pgfsetlinewidth{0.100000\du}
\pgfsetdash{{\pgflinewidth}{0.200000\du}}{0cm}
\pgfsetdash{{\pgflinewidth}{0.200000\du}}{0cm}
\pgfsetbuttcap
{
\definecolor{dialinecolor}{rgb}{0.000000, 0.000000, 0.000000}
\pgfsetfillcolor{dialinecolor}
% was here!!!
\pgfsetarrowsend{stealth}
\definecolor{dialinecolor}{rgb}{0.000000, 0.000000, 0.000000}
\pgfsetstrokecolor{dialinecolor}
\pgfpathmoveto{\pgfpoint{25.649759\du}{12.639195\du}}
\pgfpatharc{283}{238}{5.494268\du and 5.494268\du}
\pgfusepath{stroke}
}
\pgfsetlinewidth{0.100000\du}
\pgfsetdash{{\pgflinewidth}{0.200000\du}}{0cm}
\pgfsetdash{{\pgflinewidth}{0.200000\du}}{0cm}
\pgfsetbuttcap
{
\definecolor{dialinecolor}{rgb}{0.000000, 0.000000, 0.000000}
\pgfsetfillcolor{dialinecolor}
% was here!!!
\pgfsetarrowsend{stealth}
\definecolor{dialinecolor}{rgb}{0.000000, 0.000000, 0.000000}
\pgfsetstrokecolor{dialinecolor}
\pgfpathmoveto{\pgfpoint{20.556576\du}{18.397407\du}}
\pgfpatharc{34}{-17}{3.719076\du and 3.719076\du}
\pgfusepath{stroke}
}
\pgfsetlinewidth{0.100000\du}
\pgfsetdash{}{0pt}
\pgfsetdash{}{0pt}
\pgfsetbuttcap
{
\definecolor{dialinecolor}{rgb}{0.000000, 0.000000, 0.000000}
\pgfsetfillcolor{dialinecolor}
% was here!!!
\pgfsetarrowsend{stealth}
\definecolor{dialinecolor}{rgb}{0.000000, 0.000000, 0.000000}
\pgfsetstrokecolor{dialinecolor}
\draw (19.825000\du,22.775000\du)--(19.810147\du,20.335952\du);
}
\pgfsetlinewidth{0.100000\du}
\pgfsetdash{}{0pt}
\pgfsetdash{}{0pt}
\pgfsetbuttcap
{
\definecolor{dialinecolor}{rgb}{0.000000, 0.000000, 0.000000}
\pgfsetfillcolor{dialinecolor}
% was here!!!
\definecolor{dialinecolor}{rgb}{0.000000, 0.000000, 0.000000}
\pgfsetstrokecolor{dialinecolor}
\draw (18.750000\du,22.800000\du)--(20.900000\du,22.750000\du);
}
\pgfsetlinewidth{0.100000\du}
\pgfsetdash{{\pgflinewidth}{0.200000\du}}{0cm}
\pgfsetdash{{\pgflinewidth}{0.200000\du}}{0cm}
\pgfsetbuttcap
{
\definecolor{dialinecolor}{rgb}{0.000000, 0.000000, 0.000000}
\pgfsetfillcolor{dialinecolor}
% was here!!!
\pgfsetarrowsend{stealth}
\definecolor{dialinecolor}{rgb}{0.000000, 0.000000, 0.000000}
\pgfsetstrokecolor{dialinecolor}
\draw (19.675000\du,5.625000\du)--(20.196247\du,7.782488\du);
}
\pgfsetlinewidth{0.100000\du}
\pgfsetdash{}{0pt}
\pgfsetdash{}{0pt}
\pgfsetbuttcap
{
\definecolor{dialinecolor}{rgb}{0.000000, 0.000000, 0.000000}
\pgfsetfillcolor{dialinecolor}
% was here!!!
\pgfsetarrowsend{stealth}
\definecolor{dialinecolor}{rgb}{0.000000, 0.000000, 0.000000}
\pgfsetstrokecolor{dialinecolor}
\draw (26.400550\du,22.235550\du)--(25.653364\du,20.683364\du);
}
\pgfsetlinewidth{0.100000\du}
\pgfsetdash{}{0pt}
\pgfsetdash{}{0pt}
\pgfsetbuttcap
{
\definecolor{dialinecolor}{rgb}{0.000000, 0.000000, 0.000000}
\pgfsetfillcolor{dialinecolor}
% was here!!!
\definecolor{dialinecolor}{rgb}{0.000000, 0.000000, 0.000000}
\pgfsetstrokecolor{dialinecolor}
\draw (25.351100\du,22.571100\du)--(27.450000\du,21.900000\du);
}
\pgfsetlinewidth{0.100000\du}
\pgfsetdash{{\pgflinewidth}{0.200000\du}}{0cm}
\pgfsetdash{{\pgflinewidth}{0.200000\du}}{0cm}
\pgfsetbuttcap
{
\definecolor{dialinecolor}{rgb}{0.000000, 0.000000, 0.000000}
\pgfsetfillcolor{dialinecolor}
% was here!!!
\pgfsetarrowsend{stealth}
\definecolor{dialinecolor}{rgb}{0.000000, 0.000000, 0.000000}
\pgfsetstrokecolor{dialinecolor}
\draw (25.850000\du,7.100000\du)--(24.703038\du,9.266728\du);
}
\pgfsetlinewidth{0.100000\du}
\pgfsetdash{}{0pt}
\pgfsetdash{}{0pt}
\pgfsetbuttcap
{
\definecolor{dialinecolor}{rgb}{0.000000, 0.000000, 0.000000}
\pgfsetfillcolor{dialinecolor}
% was here!!!
\definecolor{dialinecolor}{rgb}{0.000000, 0.000000, 0.000000}
\pgfsetstrokecolor{dialinecolor}
\draw (18.350000\du,5.900000\du)--(21.000000\du,5.350000\du);
}
\pgfsetlinewidth{0.100000\du}
\pgfsetdash{}{0pt}
\pgfsetdash{}{0pt}
\pgfsetbuttcap
{
\definecolor{dialinecolor}{rgb}{0.000000, 0.000000, 0.000000}
\pgfsetfillcolor{dialinecolor}
% was here!!!
\definecolor{dialinecolor}{rgb}{0.000000, 0.000000, 0.000000}
\pgfsetstrokecolor{dialinecolor}
\draw (24.850000\du,6.500000\du)--(26.850000\du,7.700000\du);
}
\definecolor{dialinecolor}{rgb}{1.000000, 1.000000, 1.000000}
\pgfsetfillcolor{dialinecolor}
\pgfpathellipse{\pgfpoint{14.443364\du}{14.681682\du}}{\pgfpoint{1.056857\du}{0\du}}{\pgfpoint{0\du}{1.030087\du}}
\pgfusepath{fill}
\pgfsetlinewidth{0.100000\du}
\pgfsetdash{}{0pt}
\pgfsetdash{}{0pt}
\pgfsetmiterjoin
\definecolor{dialinecolor}{rgb}{0.000000, 0.000000, 0.000000}
\pgfsetstrokecolor{dialinecolor}
\pgfpathellipse{\pgfpoint{14.443364\du}{14.681682\du}}{\pgfpoint{1.056857\du}{0\du}}{\pgfpoint{0\du}{1.030087\du}}
\pgfusepath{stroke}
% setfont left to latex
\definecolor{dialinecolor}{rgb}{0.000000, 0.000000, 0.000000}
\pgfsetstrokecolor{dialinecolor}
\node at (14.443364\du,14.876682\du){E'};
\pgfsetlinewidth{0.100000\du}
\pgfsetdash{}{0pt}
\pgfsetdash{}{0pt}
\pgfsetbuttcap
{
\definecolor{dialinecolor}{rgb}{0.000000, 0.000000, 0.000000}
\pgfsetfillcolor{dialinecolor}
% was here!!!
\pgfsetarrowsend{stealth}
\definecolor{dialinecolor}{rgb}{0.000000, 0.000000, 0.000000}
\pgfsetstrokecolor{dialinecolor}
\draw (11.820574\du,14.477234\du)--(13.340717\du,14.595730\du);
}
\pgfsetlinewidth{0.100000\du}
\pgfsetdash{}{0pt}
\pgfsetdash{}{0pt}
\pgfsetbuttcap
{
\definecolor{dialinecolor}{rgb}{0.000000, 0.000000, 0.000000}
\pgfsetfillcolor{dialinecolor}
% was here!!!
\definecolor{dialinecolor}{rgb}{0.000000, 0.000000, 0.000000}
\pgfsetstrokecolor{dialinecolor}
\draw (11.850000\du,15.700000\du)--(11.791149\du,13.254468\du);
}
\definecolor{dialinecolor}{rgb}{1.000000, 1.000000, 1.000000}
\pgfsetfillcolor{dialinecolor}
\pgfpathellipse{\pgfpoint{15.193403\du}{19.483402\du}}{\pgfpoint{1.053364\du}{0\du}}{\pgfpoint{0\du}{1.026682\du}}
\pgfusepath{fill}
\pgfsetlinewidth{0.100000\du}
\pgfsetdash{}{0pt}
\pgfsetdash{}{0pt}
\pgfsetmiterjoin
\definecolor{dialinecolor}{rgb}{0.000000, 0.000000, 0.000000}
\pgfsetstrokecolor{dialinecolor}
\pgfpathellipse{\pgfpoint{15.193403\du}{19.483402\du}}{\pgfpoint{1.053364\du}{0\du}}{\pgfpoint{0\du}{1.026682\du}}
\pgfusepath{stroke}
% setfont left to latex
\definecolor{dialinecolor}{rgb}{0.000000, 0.000000, 0.000000}
\pgfsetstrokecolor{dialinecolor}
\node at (15.193403\du,19.678402\du){E};
\pgfsetlinewidth{0.100000\du}
\pgfsetdash{}{0pt}
\pgfsetdash{}{0pt}
\pgfsetbuttcap
{
\definecolor{dialinecolor}{rgb}{0.000000, 0.000000, 0.000000}
\pgfsetfillcolor{dialinecolor}
% was here!!!
\pgfsetarrowsend{stealth}
\definecolor{dialinecolor}{rgb}{0.000000, 0.000000, 0.000000}
\pgfsetstrokecolor{dialinecolor}
\draw (13.675020\du,21.850860\du)--(14.608440\du,20.395475\du);
}
\pgfsetlinewidth{0.100000\du}
\pgfsetdash{}{0pt}
\pgfsetdash{}{0pt}
\pgfsetbuttcap
{
\definecolor{dialinecolor}{rgb}{0.000000, 0.000000, 0.000000}
\pgfsetfillcolor{dialinecolor}
% was here!!!
\definecolor{dialinecolor}{rgb}{0.000000, 0.000000, 0.000000}
\pgfsetstrokecolor{dialinecolor}
\draw (14.500039\du,22.701720\du)--(12.850000\du,21.000000\du);
}
\pgfsetlinewidth{0.100000\du}
\pgfsetdash{{\pgflinewidth}{0.200000\du}}{0cm}
\pgfsetdash{{\pgflinewidth}{0.200000\du}}{0cm}
\pgfsetbuttcap
{
\definecolor{dialinecolor}{rgb}{0.000000, 0.000000, 0.000000}
\pgfsetfillcolor{dialinecolor}
% was here!!!
\pgfsetarrowsend{stealth}
\definecolor{dialinecolor}{rgb}{0.000000, 0.000000, 0.000000}
\pgfsetstrokecolor{dialinecolor}
\draw (19.449774\du,14.283524\du)--(15.546822\du,14.593924\du);
}
\pgfsetlinewidth{0.100000\du}
\pgfsetdash{{\pgflinewidth}{0.200000\du}}{0cm}
\pgfsetdash{{\pgflinewidth}{0.200000\du}}{0cm}
\pgfsetbuttcap
{
\definecolor{dialinecolor}{rgb}{0.000000, 0.000000, 0.000000}
\pgfsetfillcolor{dialinecolor}
% was here!!!
\pgfsetarrowsend{stealth}
\definecolor{dialinecolor}{rgb}{0.000000, 0.000000, 0.000000}
\pgfsetstrokecolor{dialinecolor}
\pgfpathmoveto{\pgfpoint{15.434900\du}{15.159237\du}}
\pgfpatharc{110}{64}{5.114357\du and 5.114357\du}
\pgfusepath{stroke}
}
\pgfsetlinewidth{0.100000\du}
\pgfsetdash{{\pgflinewidth}{0.200000\du}}{0cm}
\pgfsetdash{{\pgflinewidth}{0.200000\du}}{0cm}
\pgfsetbuttcap
{
\definecolor{dialinecolor}{rgb}{0.000000, 0.000000, 0.000000}
\pgfsetfillcolor{dialinecolor}
% was here!!!
\pgfsetarrowsend{stealth}
\definecolor{dialinecolor}{rgb}{0.000000, 0.000000, 0.000000}
\pgfsetstrokecolor{dialinecolor}
\draw (19.500000\du,15.150000\du)--(15.961462\du,18.710563\du);
}
\definecolor{dialinecolor}{rgb}{1.000000, 1.000000, 1.000000}
\pgfsetfillcolor{dialinecolor}
\pgfpathellipse{\pgfpoint{15.443403\du}{11.233402\du}}{\pgfpoint{1.056857\du}{0\du}}{\pgfpoint{0\du}{1.030087\du}}
\pgfusepath{fill}
\pgfsetlinewidth{0.100000\du}
\pgfsetdash{}{0pt}
\pgfsetdash{}{0pt}
\pgfsetmiterjoin
\definecolor{dialinecolor}{rgb}{0.000000, 0.000000, 0.000000}
\pgfsetstrokecolor{dialinecolor}
\pgfpathellipse{\pgfpoint{15.443403\du}{11.233402\du}}{\pgfpoint{1.056857\du}{0\du}}{\pgfpoint{0\du}{1.030087\du}}
\pgfusepath{stroke}
% setfont left to latex
\definecolor{dialinecolor}{rgb}{0.000000, 0.000000, 0.000000}
\pgfsetstrokecolor{dialinecolor}
\node at (15.443403\du,11.428402\du){F};
\pgfsetlinewidth{0.100000\du}
\pgfsetdash{}{0pt}
\pgfsetdash{}{0pt}
\pgfsetbuttcap
{
\definecolor{dialinecolor}{rgb}{0.000000, 0.000000, 0.000000}
\pgfsetfillcolor{dialinecolor}
% was here!!!
\pgfsetarrowsend{stealth}
\definecolor{dialinecolor}{rgb}{0.000000, 0.000000, 0.000000}
\pgfsetstrokecolor{dialinecolor}
\draw (13.520594\du,8.703094\du)--(14.784081\du,10.365771\du);
}
\pgfsetlinewidth{0.100000\du}
\pgfsetdash{}{0pt}
\pgfsetdash{}{0pt}
\pgfsetbuttcap
{
\definecolor{dialinecolor}{rgb}{0.000000, 0.000000, 0.000000}
\pgfsetfillcolor{dialinecolor}
% was here!!!
\definecolor{dialinecolor}{rgb}{0.000000, 0.000000, 0.000000}
\pgfsetstrokecolor{dialinecolor}
\draw (12.450000\du,9.600000\du)--(14.591188\du,7.806188\du);
}
\pgfsetlinewidth{0.100000\du}
\pgfsetdash{{\pgflinewidth}{0.200000\du}}{0cm}
\pgfsetdash{{\pgflinewidth}{0.200000\du}}{0cm}
\pgfsetbuttcap
{
\definecolor{dialinecolor}{rgb}{0.000000, 0.000000, 0.000000}
\pgfsetfillcolor{dialinecolor}
% was here!!!
\pgfsetarrowsend{stealth}
\definecolor{dialinecolor}{rgb}{0.000000, 0.000000, 0.000000}
\pgfsetstrokecolor{dialinecolor}
\draw (16.392132\du,11.790001\du)--(19.450040\du,13.584011\du);
}
\end{tikzpicture}
}
\caption{An illustration of all the ways in which an initiator node, $A$, can
be connected to the graph before a collection process begins. The circles in the
graph represent sets of nodes.
Dotted lines represent one or more non-strong paths, and solid lines represent one or more
strong paths (but non-strong paths may exist alongside them).
A T-shaped end-point indicates the root, $R$.
Other nodes and connections between nodes may exist,
but here we map only paths and nodes related to $A$ and $R$.
Thus, if $C'$, $D'$, $E'$ and $F$ are empty sets, $A$ is garbage, otherwise
it is not.}
\label{fig:completeabstract}
\end{center}
\vspace{-9mm}
\end{figure}
\end{comment}
%\paragraph{Modified Reference Graph $G$ using the Brownbridge Method:}
\paragraph{Classification of Edges in $G$:}
%Instead of using the reference graph $G$ defined above, we use the Brownbridge method \cite{Brownbridge1985} of classifying edges of $G$.
We classify the edges in $G$ as $weak$ or $strong$ according to the Brownbridge method~\cite{Brownbridge1985}.
This classification does not need preprocessing of $G$ and can be done directly at the time of creation of $G$.
%In the Brownbridge method, edges of $G$ are labeled as either $strong$ or $weak$. %Each node has two reference counts, one for each
%incoming edge type.
%The reason for the two types of edges are due to
%inability to detect cyclic structure by single edge type.
%Brownbridge idea is
%to avoid cycle of strong edges in the graph.
Chains of strong edges connect the root $R$ to all the nodes $G$ and contain
no cycles. The weak edges
are required for all cycle-creating edges in $G$. The advantage to the Brownbridge
method is that it permits less frequent
tracing of the outgoing edges. Without a distinction between strong and weak edges,
tracing is required after every deletion that does not bring the reference count to zero.
In the Brownbridge method, if a strong edge remains after the deletion of any edge,
the node is live and no tracing
is necessary.
%Our algorithm takes graph $G$ as this modified reference graph.
Fig.~\ref{fig:completeabstract} illustrates an example reference graph $G$ with strong (solid) and weak (dotted) edges.

\paragraph{Weight-based Edge Classification for the Reference Graph $G$:} We assign {\em weights} to the nodes as a means of classifying the edges in $G$.
Each node of $G$ maintains two attributes %related to the weight component of the algorithm.
$weight$ and $\maxweight$, as well as counts of strong and weak incoming edges. %The weight is an integer in each node that helps to
%label the edges in the graph.
The weight of the root, $R$, is zero at all times. For
an edge to
be strong, the weight of the source must be less than the weight of the
target.
The
$\maxweight$ of a node is defined as the maximum weight of the node's
$\inneighbors$. Messages sent by nodes contain
the $weight$ of the sender so that the $\outneighbors$ can keep track of their
strong count, weak count, and $\maxweight.$
When a
node is created, its initial weight is given by adding one to the weight of the
first node to reference it.
%edge to the node is a strong edge. An edge is considered strong only when the
%weight of the source node is less than weight of the target node. So a strong
%edge $e$ from $x$ to $y$ means $x.weight$ $<$ $y.weight$. Node creation followed by
%edge created to that node updates the $\maxweight$ of the node. If the node is
%just created, the weight is updated to the first increment of max weight. This
%satisfies the requirements to make the edge into strong edge.

%The advantage of
%weight based labeling is, along with atomic edge labeling, our algorithm
%requires
%the edge label to be changed
%without the source of the edge interacting with the conversion process. In this
%approach, a
%target node of an edge can change the label from strong to weak or vice-versa by
%adjusting
%its weight. When a node creates an edge, the source of the edge does not
%wait for the target to notify the label of the edge because the label of the
%edge is based on
%the weight of the source which is already sent in the message to create that
%edge.
%--% \begin{figure}
%--% \begin{center}
%--% \scalebox{0.6}[0.6]{\input{figs/fig2}}
%--% \caption{Examples of labeling edges when the edge is created}
%--% \label{fig:brownbridge}
%--% \end{center}
%--% \end{figure}

%\begin{figure}[h]
%\caption{Example of labeling edges when the edge is created using brownbridge
%technique}
%\centering
%\includegraphics[width=\textwidth]{figs/strongweak}
%\label{fig:brownbridge}
%\end{figure}

%Note that having a weak incoming edge does not necessarily mean the
%node is part of cycle.
% as depicted in the \ref{fig:brownbridge}.
%In figure \ref{fig:brownbridge}, there are two examples shown explaining
%our technique to name edges. %$N_i$ can be considered to be created
%before $N_{i-1}$.
%In the image on the left, a weak edge is labeled to mark
%the cycle closing edge. On the right side, however, an edge is marked weak although
%node $N_4$, which is not part of any cycle.


%\todo{remove?}
\begin{lemma}[Strong Cycle Invariant]
No cycle of strong edges will be formed by weight-based edge classification.
\end{lemma}

\begin{proof}
By construction, for any node $y$ which is reachable from a node $x$ by
strong edges, $y$ must have
a larger weight than $x$. Therefore, if $x$ is reachable from itself through a cycle,
and if that cycle is strong, $x$ must have a greater
weight than itself, which is impossible.
%To create a cycle of strong edges, the weight of source of cycle closing edge
%must be lesser than the weight of the target node. Let's assume the source of
%cycle closing edge for the graph is $z$ and target of the cycle closing edge
%is $a$. As we mentioned above, the cycle closing edge can be strong only if
%the $z.weigh$ $<$ $a.weight$. For this to be a cycle closing edge, there must
%be at least an edge from $a$ to series of nodes ultimately reaching $z$ or
%$a$ contains a edge to $z$ directly. In the later case, it is clear that the
%edge from $a$ to $z$ will be weak since the weights are not satisfying the
%condition to label them as strong. In the former case, for all the edges
%between $a$ to $z$ to be strong, all the weight of nodes in the path from
%$a$ to $z$ must be in ascending order. This contradicts our assumption that
%$a.weight$ $>$ $z.weight$. So a cycle of strong edges cannot be formed in weightbased edge labeling.
\end{proof}

\paragraph{Terminology:}
A $path$ is an ordered set of edges such that the destination
of each edge is the origin of the next.
A path is called a $strong$ path if it consists
exclusively of strong edges, otherwise it is called
$\nonstrong$ path. We say two nodes are $related$ if
a path exists between them.
A node, $x$, is a member of the $dependent$ $set$ of A if all strong paths leading to $x$ originate on $A$
(i.e. $B$, $B'$, $C$, and $C'$ in Fig.~\ref{fig:completeabstract}).

Consider a node, A, that has just lost its last strong edge. We model this situation
with an induced subgraph with $A$ removed (i.e. $G-A$), then the $purely$ $dependent$ $set$ of A
consists of the nodes that have no path from $R$
(i.e. $B$, and $B'$  in Fig.~\ref{fig:completeabstract}).
A node, $y$, is a member of the $supporting$ $set$ of A if
there is a path from $R$ to $y$, and from $y$ to $A$ (i.e.
$C'$, $D'$, $E'$, and $F$ in Fig.~\ref{fig:completeabstract} ).
A node, $z$, is a member of the $independent$ $set$ of A if $z$ is related to $A$ and
there is at least one strong path from $R$ to $z$
(i.e.
$D$, $D'$, $E$, $E'$, and $F$ in Fig.~\ref{fig:completeabstract}).
A node, $x$, is a member of the $build$ $set$ of A if it is both a member of the supporting set and
the independent set.
Alternatively, A node, $x$ is said to be in the $build$ $set$ if a strong path from $R$ to $x$ exists,
but also a path from $x$ to $A$ exists (i.e. $D'$, $E'$ and $F$ in Fig.~\ref{fig:completeabstract}).
A node, $x$, is a member of the $recovery$ $set$ of A if it is both a member of the supporting set
and the dependent set
(i.e. $C'$ in Fig.~\ref{fig:completeabstract}).
A node, $x$, is said to be in the $\mathit{affected\ set}$ if there exists a path, $p$, from $A$ to $x$, such that no
node found along $p$ (other than $x$ itself) has a strong path from $R$ (i.e.
everything except $F$ in Fig.~\ref{fig:completeabstract}).


\section{ Single Collector Algorithm (SCA)} 
\label{single}
We discuss here the single collection version of our algorithm; the multi-collection version will be discussed in Section \ref{multi}.
For the single collector version, we assume that every mutation in $G$ is serialized
and that the Adversary does not create or delete edges in $G$ during a \emph{collection process} (i.e. the sequence of messages and
state changes needed to determine the liveness of a node).
When the last strong edge to a node, $x$, in $G$ is deleted, but $\Gamma_{\rm in}(x)$ is not empty,
$x$ becomes an {\em initiator}. % A node is
%considered to be an initiator when it is uncertain about it's liveness, i.e. when
%it still has incoming edges, but loses its last strong edge.
The initiator node starts a set of graph traversals which we call phases:
{\em phantomization}, {\em recovery}, {\em building}, and {\em deletion}. We classify the latter three
phases as {\em correction} phases.
%All phases will run in the order listed on any given node, except that
%recovery and deletion will only occur in certain situations.
Fig.~\ref{fig:state} provides a flow diagram of the phases of an initiator. 
\begin{figure}
\centering
%\begin{center}
\scalebox{0.6}[0.6]{% Graphic for TeX using PGF
% Title: C:\Users\Hari\Pictures\Diagram2.dia
% Creator: Dia v0.97.2
% CreationDate: Wed May 11 10:06:32 2016
% For: Hari
% \usepackage{tikz}
% The following commands are not supported in PSTricks at present
% We define them conditionally, so when they are implemented,
% this pgf file will use them.
\ifx\du\undefined
  \newlength{\du}
\fi
\setlength{\du}{15\unitlength}
\begin{tikzpicture}
\pgftransformxscale{1.000000}
\pgftransformyscale{-1.000000}
\definecolor{dialinecolor}{rgb}{0.000000, 0.000000, 0.000000}
\pgfsetstrokecolor{dialinecolor}
\definecolor{dialinecolor}{rgb}{1.000000, 1.000000, 1.000000}
\pgfsetfillcolor{dialinecolor}
\definecolor{dialinecolor}{rgb}{1.000000, 1.000000, 1.000000}
\pgfsetfillcolor{dialinecolor}
\pgfpathellipse{\pgfpoint{14.956636\du}{4.226682\du}}{\pgfpoint{3.143364\du}{0\du}}{\pgfpoint{0\du}{1.876682\du}}
\pgfusepath{fill}
\pgfsetlinewidth{0.100000\du}
\pgfsetdash{}{0pt}
\pgfsetdash{}{0pt}
\pgfsetmiterjoin
\definecolor{dialinecolor}{rgb}{0.000000, 0.000000, 0.000000}
\pgfsetstrokecolor{dialinecolor}
\pgfpathellipse{\pgfpoint{14.956636\du}{4.226682\du}}{\pgfpoint{3.143364\du}{0\du}}{\pgfpoint{0\du}{1.876682\du}}
\pgfusepath{stroke}
% setfont left to latex
\definecolor{dialinecolor}{rgb}{0.000000, 0.000000, 0.000000}
\pgfsetstrokecolor{dialinecolor}
\node at (14.956636\du,4.466682\du){};
% setfont left to latex
\definecolor{dialinecolor}{rgb}{0.000000, 0.000000, 0.000000}
\pgfsetstrokecolor{dialinecolor}
\node[anchor=west] at (12.441636\du,4.351682\du){\textbf{Phantomization}};
\definecolor{dialinecolor}{rgb}{1.000000, 1.000000, 1.000000}
\pgfsetfillcolor{dialinecolor}
\pgfpathellipse{\pgfpoint{18.668364\du}{10.980046\du}}{\pgfpoint{2.178364\du}{0\du}}{\pgfpoint{0\du}{1.576682\du}}
\pgfusepath{fill}
\pgfsetlinewidth{0.100000\du}
\pgfsetdash{}{0pt}
\pgfsetdash{}{0pt}
\pgfsetmiterjoin
\definecolor{dialinecolor}{rgb}{0.000000, 0.000000, 0.000000}
\pgfsetstrokecolor{dialinecolor}
\pgfpathellipse{\pgfpoint{18.668364\du}{10.980046\du}}{\pgfpoint{2.178364\du}{0\du}}{\pgfpoint{0\du}{1.576682\du}}
\pgfusepath{stroke}
% setfont left to latex
\definecolor{dialinecolor}{rgb}{0.000000, 0.000000, 0.000000}
\pgfsetstrokecolor{dialinecolor}
\node at (18.668364\du,11.220046\du){\textbf{Recovery}};
% setfont left to latex
\definecolor{dialinecolor}{rgb}{0.000000, 0.000000, 0.000000}
\pgfsetstrokecolor{dialinecolor}
\node[anchor=west] at (9.818364\du,11.005046\du){};
\definecolor{dialinecolor}{rgb}{1.000000, 1.000000, 1.000000}
\pgfsetfillcolor{dialinecolor}
\pgfpathellipse{\pgfpoint{10.268364\du}{18.130046\du}}{\pgfpoint{2.178364\du}{0\du}}{\pgfpoint{0\du}{1.576682\du}}
\pgfusepath{fill}
\pgfsetlinewidth{0.100000\du}
\pgfsetdash{}{0pt}
\pgfsetdash{}{0pt}
\pgfsetmiterjoin
\definecolor{dialinecolor}{rgb}{0.000000, 0.000000, 0.000000}
\pgfsetstrokecolor{dialinecolor}
\pgfpathellipse{\pgfpoint{10.268364\du}{18.130046\du}}{\pgfpoint{2.178364\du}{0\du}}{\pgfpoint{0\du}{1.576682\du}}
\pgfusepath{stroke}
% setfont left to latex
\definecolor{dialinecolor}{rgb}{0.000000, 0.000000, 0.000000}
\pgfsetstrokecolor{dialinecolor}
\node at (10.268364\du,18.370046\du){\textbf{Build}};
\definecolor{dialinecolor}{rgb}{1.000000, 1.000000, 1.000000}
\pgfsetfillcolor{dialinecolor}
\pgfpathellipse{\pgfpoint{25.768364\du}{18.380046\du}}{\pgfpoint{2.579644\du}{0\du}}{\pgfpoint{0\du}{1.867125\du}}
\pgfusepath{fill}
\pgfsetlinewidth{0.100000\du}
\pgfsetdash{}{0pt}
\pgfsetdash{}{0pt}
\pgfsetmiterjoin
\definecolor{dialinecolor}{rgb}{0.000000, 0.000000, 0.000000}
\pgfsetstrokecolor{dialinecolor}
\pgfpathellipse{\pgfpoint{25.768364\du}{18.380046\du}}{\pgfpoint{2.579644\du}{0\du}}{\pgfpoint{0\du}{1.867125\du}}
\pgfusepath{stroke}
% setfont left to latex
\definecolor{dialinecolor}{rgb}{0.000000, 0.000000, 0.000000}
\pgfsetstrokecolor{dialinecolor}
\node at (25.768364\du,18.620046\du){\textbf{Deletion}};
\pgfsetlinewidth{0.100000\du}
\pgfsetdash{}{0pt}
\pgfsetdash{}{0pt}
\pgfsetbuttcap
{
\definecolor{dialinecolor}{rgb}{0.000000, 0.000000, 0.000000}
\pgfsetfillcolor{dialinecolor}
% was here!!!
\pgfsetarrowsstart{stealth}
\definecolor{dialinecolor}{rgb}{0.000000, 0.000000, 0.000000}
\pgfsetstrokecolor{dialinecolor}
\pgfpathmoveto{\pgfpoint{19.896530\du}{9.625575\du}}
\pgfpatharc{29}{-64}{3.869885\du and 3.869885\du}
\pgfusepath{stroke}
}
\pgfsetlinewidth{0.100000\du}
\pgfsetdash{}{0pt}
\pgfsetdash{}{0pt}
\pgfsetbuttcap
{
\definecolor{dialinecolor}{rgb}{0.000000, 0.000000, 0.000000}
\pgfsetfillcolor{dialinecolor}
% was here!!!
\pgfsetarrowsend{stealth}
\definecolor{dialinecolor}{rgb}{0.000000, 0.000000, 0.000000}
\pgfsetstrokecolor{dialinecolor}
\pgfpathmoveto{\pgfpoint{13.834722\du}{6.026196\du}}
\pgfpatharc{210}{187}{27.410428\du and 27.410428\du}
\pgfusepath{stroke}
}
\pgfsetlinewidth{0.100000\du}
\pgfsetdash{}{0pt}
\pgfsetdash{}{0pt}
\pgfsetbuttcap
{
\definecolor{dialinecolor}{rgb}{0.000000, 0.000000, 0.000000}
\pgfsetfillcolor{dialinecolor}
% was here!!!
\pgfsetarrowsend{stealth}
\definecolor{dialinecolor}{rgb}{0.000000, 0.000000, 0.000000}
\pgfsetstrokecolor{dialinecolor}
\pgfpathmoveto{\pgfpoint{16.759366\du}{11.816219\du}}
\pgfpatharc{243}{216}{15.710313\du and 15.710313\du}
\pgfusepath{stroke}
}
\pgfsetlinewidth{0.100000\du}
\pgfsetdash{}{0pt}
\pgfsetdash{}{0pt}
\pgfsetbuttcap
{
\definecolor{dialinecolor}{rgb}{0.000000, 0.000000, 0.000000}
\pgfsetfillcolor{dialinecolor}
% was here!!!
\pgfsetarrowsstart{stealth}
\definecolor{dialinecolor}{rgb}{0.000000, 0.000000, 0.000000}
\pgfsetstrokecolor{dialinecolor}
\pgfpathmoveto{\pgfpoint{25.439190\du}{16.478320\du}}
\pgfpatharc{344}{292}{7.799925\du and 7.799925\du}
\pgfusepath{stroke}
}
% setfont left to latex
\definecolor{dialinecolor}{rgb}{0.000000, 0.000000, 0.000000}
\pgfsetstrokecolor{dialinecolor}
\node[anchor=west] at (4.370000\du,7.853364\du){\textbf{If Build Set is non-empty}};
% setfont left to latex
\definecolor{dialinecolor}{rgb}{0.000000, 0.000000, 0.000000}
\pgfsetstrokecolor{dialinecolor}
\node[anchor=west] at (19.920000\du,5.603364\du){\textbf{If Build Set is empty}};
% setfont left to latex
\definecolor{dialinecolor}{rgb}{0.000000, 0.000000, 0.000000}
\pgfsetstrokecolor{dialinecolor}
\node[anchor=west] at (13.170000\du,15.503364\du){\textbf{If Recovery set is non-empty}};
% setfont left to latex
\definecolor{dialinecolor}{rgb}{0.000000, 0.000000, 0.000000}
\pgfsetstrokecolor{dialinecolor}
\node[anchor=west] at (22.920000\du,12.053364\du){\textbf{If Recovery set is emtpy}};
\end{tikzpicture}
}
\caption{The above figure depicts the phase transitions performed by initiator in the algorithm.
}
\label{fig:state}
%\end{center}

\end{figure}



\paragraph{}

%According to %

%\cref{def:path,def:strongpath,def:nonstrongpath,def:related,def:dep,def:puredep,def:sup,def:indep,def:build,def:rec,def:aff},
As illustrated in Fig.~\ref{fig:completeabstract}, a node $A\in G$ is dead if and only if its supporting set
is empty. If $A$ is discovered to be dead, then its purely dependent set is also
dead. % Our algorithm searches for $C'$ (which contains nodes that have paths
%leading to and coming from $A$), and detmines whether $D'$ (which only contains nodes with paths
%leading to $A$) is present.
%Because of our algorithm only traverses edges from the initiator, the nodes in
%will not be known to $A$, but $A$ will still be able to determine if build set is empty.
%Note that some nodes in the build set are outside the affected set (i.e. the nodes in $F$).
Even when the supporting set does
not intersect the affected set (i.e. when $C'$, $D'$ and $E'$ are empty), and thus
nodes in the build set do not receive any messages, our algorithm will still detect whether the
supporting set is empty. Appendix ~\ref{singlealgo} contains the detailed pseudocode of the single collector algorithm. We describe the phantomization and correction (recovery, building, and deletion) phases separately below.

\subsection{Phantomization Phase}
For a node of $G$ to become a phantom node, the node must
(1) have just lost its last strong edge, and
(2) not already be a phantom node.
Each node has a flag that identifies it as a phantom node.
The initiator node is, therefore, always a phantom node, and
phantomization always begins with the initiator.
When a node $w$ becomes a phantom node, it notifies
$\Gamma_{\rm out}(w)$ of this fact with {\em phantomize} messages. From that point on, all the outgoing edges from $w$ will
be considered to be {\em phantom edges}, i.e. neither strong nor weak but a transient and
indeterminate state. If a node, $u$, in $\Gamma_{\rm out}(w)$ loses its last strong edge
as a result of receiving a {\em phantomize} message, $u$ also becomes phantom node, %(i.e. if $u$ has any weak
%edges, it sets its $weight$ to $max_weight+1$, sends the phantomize message to $\Gamma_{\rm out}(u)$,
%and marks itself phantomized)
but not an initiator.
Phantomization will thus
mark all nodes in the dependent set.
%The set of nodes marked by the phantomization are called phantomized
All the nodes that receive the phantomize message are called {\em affected nodes}.
%The set $D$ will not but phantomized, but it is affected since nodes in this
%set will have some of their edges phantomized.
%Phantomization adjusts the edge labels of the affected nodes.

%\begin{lemma}[No Phantomization]
%If a strong path from the root to a node, $x$, exists, then $x$ will not
%phantomize.
%\label{lem:nophantomization}
%\end{lemma}
%\begin{proofs}
%A node will not phantomize unless it loses its last strong reference.
%\end{proofs}

%\begin{lemma}[Phantomization]
%If no strong path from the root to a node, $x$, exists, and $x$ is in the
%dependent set of an initiator node, then $x$ will phantomize.
%\label{lem:nophantomization}
%\end{lemma}
%\begin{proofs}
%\end{proofs}

Since the algorithm proceeds in phases, we need to wait for phantomization to complete before recovery or building can begin. For this reason,
for every \emph{phantomize} message sent from a node, $x$, to a node, $y$, a {\em return} message must be received by $x$ from $y$ . If
$y$ does not phantomize as a result of receiving the message from $x$, the {\em return}
is sent back to $x$ immediately. If the {\em phantomize} message causes $y$ to phantomize,
then $y$ waits until it has received {\em return} back from all nodes in $\Gamma_{\rm out}(y)$ before
replying to $x$ with {\em return}.
For this purpose, each phantom node maintains %an information in memory about
a single backward-pointing edge and a counter to track the number of {\em return} messages it is
waiting for. While $y$ is waiting for {\em return} messages, it is said to be {\em phantomizing}. Once
all \emph{return} messages are received, the phantom node is said to be {\em phantomized}.

%During the correction
%phase, our algorithm detects whether the sets $C'$ or $D'$ are non-empty. If they non-empty,
%it readjusts the edge labels (weights) in the graph and clears the phantomization bit.
% If no such set exist, then
%our algorithm tries to readjust the labels in $C$ and $C'$ and delete $A$ , $B$
%and $B'$.
%Correction is a three step phase. It involves searching
%for $C'$ and $D'$, readjusting the labels if necessary, and also deleting $A$,
%$B$, and $B'$ nodes in the graph if $C' \cap D'$ is empty.

%\subsubsection{Phantomization}

Phantomization is, therefore, essentially, a breadth-first search
rooted at the initiator.
%It will identify the affected nodes, i.e. the subset of the dependent and independent sets of nodes
%that are reachable from initiator.
%The process uses a message
%called \emph{phantomize} to find affected nodes.
The traversal contains two steps. In the forward
step, messages originate from the initiator and propagate to the affected nodes.
After they reach all the affected nodes,
\emph{return} messages will propagate backward to the initiator.
The reverse step is
a simple termination detection process which uses
a spanning tree in the subgraph comprised of the affected nodes (i.e., the single
backward-pointing edge forms the spanning tree).
%Each node will
%contain a parent pointer that is set when the node receives the first
%\emph{phantomize} message. During reverse step, each node will send \emph{return}
%message to the parent.

The edges of any phantom node are said to be \emph{phantom edges}.
%To distinguish the nodes that are part of the
%dependent set, our algorithm uses a flag for each node called \emph{phantomized}
%that identifies all outgoing edges of phantomized nodes as phantom edges. % A node and its
%outgoing edges are said to be
%phantomized if the \emph{phantomized flag} is set to true.
%On receiving a \emph{phantomize message}, based on the node's state, a node may send
%phantomize message to outneighbors. When a node sends phantomize message to
%outneighbors, it is said to be \emph{phantomized} and marks the propogation of
%phantomize message by setting the \emph{phantomized} flag. When a \emph{phantomize}
%message is sent from a node to another, the edge between them is said to converted to
%phantom. In addition to
Each node has a counter for its incoming phantom edges
(in addition to its strong and weak counter). Every node in the affected subgraph
will have a positive phantom count at the end of the phantomization phase. The initiator then enters to the correction phase. 
%This counter counts the number of incoming
%phantom edges and also, node updates the loss of strong and weak edges in their
%appropriate counter when the edges are converted into phantom.

%We mentioned that a node will send phantomize message to outneighbors based on
%certain conditions.

In addition to sending phantomize messages to its $\outneighbors$, when a node satisfies the
above conditions and contains only weak incoming edges, the node converts all the weak
incoming edges into strong incoming edges. %then sends the \emph{phantomize message} to its $\outneighbors$.
The process of converting all the weak incoming edges into strong
edges is called \emph{toggling}. Toggling is achieved by updating the weight
of the node to the maximum weight of its in-neighbors plus one ($weight \leftarrow \maxweight+1$).
%Every message sent
%in our algorithm always contain the weight of the node sending the message. At any
%point in time, all the nodes are aware of the maximum in-neighbor's weight.

%As we discussed above, the forward step
%along with finding the affected nodes, it also creates a spanning tree with parent
%pointer set for all the affected nodes to send the \emph{return} message back to
%initiator eventually. Return messages are also send to non-parent nodes /
%\emph{phantomize} message sender in certain cases as mentioned below:
%\begin{enumerate}
%\item If a node receives a phatonmize message and  is already phantomized after
%updating counters, it sends \emph{return} message to sender.
%\item If a node receives a phantomize message and  contains strong incoming edges
%after updating counters, it sends \emph{return} message to sender.
%\end{enumerate}

%When a node receives return messages from all the outneighbors,
%the node sends the return message to parent. When initiator receives return
%messages from children, the phantomization is said to be completed.
%\begin{definition}{Round:}
%We define rounds in the manner used by the $\mathcal{CONGEST}$ asynchronous model. A
%computation initiated by a node, $x$, is said to be in the $rth$ round if
%all $r$-neighborhoods of $x$ received the message.
%\end{definition}
\begin{comment}
\begin{lemma}[Time Complexity]
Phantomization finishes in O(Rad(i,$G_{a}$)) time, where $G_{a}$ is the graph
induced by affected nodes, i is the initiator, and Rad(i, $G_{a}$) is the radius
of the graph from from i.
\label{TimeP}
\end{lemma}
\begin{proofs}
%In r rounds, we assume at least r neighborhoods of i received the \emph{phantomize}
%message. The \emph{phantomize} message must reach all nodes in the $G_{a}$ from i.
In each time step, \emph{phantomization} spreads to the $\outneighbors$ of the previously
affected nodes, increasing the radius of the graph of phantomized nodes by 1.
In O(Rad(i,$G_{a}$)) time, all the nodes in $G_{a}$ receive
a \emph{phantomize} message, since all the nodes in $G_{a}$ are at distance less than
or equal to Rad(i,$G_{a}$) from i. In the reverse step, the same argument can be
applied backward.
%In r time, the $( Rad(i,G_{a}) - r) ^{th}$ neighborhood of i
%receives the \emph{return} messages.
So phantomization completes in O(Rad(i,$G_{a}$) time.
\end{proofs}

\begin{lemma}[Communication Complexity]
Phantomization uses O($E_{a}$) messages, where $E_{a}$ is the number of edges
in the graph induced by affected nodes.
\label{CCP}
\end{lemma}
\begin{proofs}
In the forward step of the algorithm, all the nodes in the dependent set send the
\emph{phantomize} message to their $\outneighbors$, and each node can do this at most
once (after which the \emph{phantomized} flag is set).
So the forward step of the algorithm uses only
O($E_{a}$) messages. In the reverse step, the \emph{return} messages are sent
backward along the edges of the spanning tree. So the reverse step sends O($V_{a}$)
messages, where $V_{a}$ is the number of nodes in the affected subgraph.
So Phantomization uses O($E_{a}$) messages.
\end{proofs}

\begin{lemma}[Message size Complexity]
Phantomization sends messages of O(log(n)) size, where n is the total number of nodes
in the graph.
\label{MSCP}
\end{lemma}
\begin{proofs}
The \emph{phantomize} messages have to hold a value at least as large as the count of nodes in the system
which are O(log(n)) size. Apart from the ids, the message also contains the weight of
node which is constant in size. In the return message, our algorithm only uses the id
of the sender and receiver. So all our messages in the phantomization are of
O(log(n)) size.
\end{proofs}

\end{comment}

\subsection{Correction Phase}

%Correction includes computing the
%supporting set, readjusting the labels and weights of the supporting set
%exists, and deleting a set of nodes if the supporting set is empty.
%After
%phantomization is complete,
The initiator starts either the \emph{recovery} or \emph{building} phase depending on the build set. 
%the initiator checks to see whether the build set is empty (i.e. if the initiator
%has strong edges).
If the build set is empty, the initiator enters the recovery phase; if it is
not empty, it enters the building phase.
In the building phase, the affected subgraph is traversed, phantom edges are
converted to strong and weak edges, and the weights of the nodes are adjusted.
%If the initiator finds out that the build
%set is empty (i.e. if the initiator has no incoming strong edges), then the direct building of the
%affected subgraph is not possible. The next phase is \emph{recovery}.
In the recovery phase, the affected subgraph is traced until the recovery set is found (i.e. phantom nodes
that have strong incoming edges). If and when this set is found, building phase begins. If not
the initiator deletes itself and all the purely
dependent nodes. We describe each phase in detail below.  

\paragraph{Building Phase:}
If the initiator has any strong incoming edges after the phantomization phase, then
the build set is not empty. In response, the initiator updates its phase to \emph{building}
and sends \emph{build} messages to
its $\outneighbors$, which convert phantom edges to strong or weak edges.

If a node, $y$, sends a \emph{build} message to a node, $x$, that is phantom node, the
$\maxweight$ of $x$ is set to the weight of $y$ ($x.\maxweight \leftarrow y.weight$),
and the weight of $x$ is set to the weight of $y$ plus one ($x.weight \leftarrow
y.weight+1$).  The node $x$ then
decrements the phantom count, increments the strong count, and
propagates the \emph{build} message to its $\outneighbors$.

If a node, $y$, sends a \emph{build} message to a node, $x$, that is not phantom node, it
updates the $\maxweight$ of $x$ if necessary, decrements
the phantom count and increments either the strong or the weak count, depending
on whether $y.weight < x.weight$. % of $y$ is less than the $weight$ of $x$ or not.

After a node, $x$, builds its last phantom edge and replies to its parent in the spanning tree with
\emph{return} (if it is the initiator node, it does not do this),
$x$ is then returned to a
\emph{healthy} state (i.e. all flags set to false,
etc.).

%(or
%possibly a phantom edge to a weak edge if other strong edges are already
%present on the receiving node).


%The phantomization, building, and recovery
%phases of the algorithm each contain two steps, forward and reverse.  In
%building, the forward step converts the phantom edges to strong or weak edges.
%In all cases, the reverse step detects the termination of the phase.

\begin{lemma}[Not Phantomized]
After phantomization, nodes in the build set are not phantom nodes.
\label{lem:nophan}
\end{lemma}
\begin{proof}
Appendix ~\ref{proofapp} contains the proof.
\end{proof}

\begin{lemma}[Build set]
After phantomization, if the initiator has strong incoming edges, then the
initiator is not dead.
\label{lem:buildset}
%nodes that are the source of those
%strong edges form the build set.
\end{lemma}
\begin{proof}
Appendix ~\ref{proofapp} contains the proof.
\end{proof}

%All nodes that are not already building that
%receive a build message change their status to building, and propagate the message along
%their outgoing edges.
%A node that sent a \emph{return} message in the building state will become a
%healthy node with no markings (i.e. no collection id, phantom count, etc.), thus removing itself from the phantomized subgraph and the collection process.

Like the phantomization phase, the building phase has a forward and reverse step and creates a spanning tree by resetting the parent attribute of the node in the same way as phantomization (i.e. when it
receives the first \emph{build} message). When all its $\outneighbors$ reply with \emph{return}, it sends \emph{return} to its parent.
A node that receives a \emph{build} message, but is already building, builds the phantom edge and sends the \emph{return} message to the sender.
%When a node receives a \emph{build} message in the
%forward step, the node converts the incoming phantom edge into strong or weak
%and  propagates under following condition:
%if the node finished the reverse step of phantomization or correction phase and it is not
%healthy node, then node converts the phantom edge to strong edge and
%propagates the \emph{build} message to out-neighbors. Otherwise, the node sends
%\emph{return} message to the sender.


%Forward step adjusts the weight of the node to have a strong path, if there is no
%strong path exist for the node. The idea behind creating the strong path is if a node
%receives a \emph{build} message from a node, the sender of \emph{build} message has a
%strong path already and to create a strong path, the node simply adjusts the weight of
%the node to higher than the weight of the \emph{build} message sender. Every message
%in the algorithm contains the weight of the node. If the node already has a strong path,
%then node simply updates the max-weight of the node and interprets the type of the edge
%from the sender's weight. After the build message is processed, the counters are updated
%to reflect the updated edges. Apart from the counters, the phatnomized flag is reset
%upon propagating the build message to out-neighbors. The node changes the state to
%building upon propagating the build message. A node sends the return message in the building
%step only if a node receives a build message when it is in healthy state or forward step in
%any part of the correction phase or no out-neighbors.

%At the end of the build process, the return messages are transmitted back and the initiator
%terminates the build process. The build process makes all the affected nodes become healthy
%and no node will have any marking or any non-zero counts in the phantom counter. If the
%build set is empty, the recovery process begins. Lemma \ref{TimeP},\ref{CCP}, and \ref{MSCP}
%holds true for the building process too as the building process is very similar to the
%phantomization except the algorithm removes the labels in the affected subgraph rather than
%the marking the nodes.

%\subsubsection*{Recovering}
\paragraph{Recovery Phase:}
%When the build set is empty, then the \emph{recovery phase} of the algorithm proceeds.
Recovery is the search for the recovery set. If found, a build of the subgraph of
nodes that are reachable from recovery set of nodes will begin. It is possible that the
recovery phase can build edges to the initiator and thereby rebuild the entire
affected set.


\begin{lemma}[Phantomized]
After phantomization, nodes in the recovery set are phantom nodes.
\label{lem:phant}
\end{lemma}
\begin{proof}
Appendix ~\ref{proofapp} contains the proof.
\end{proof}

\begin{lemma}[Recovery set]
After phantomization,
if a phantom node contains at least one strong incoming edge,
it belongs to the recovery set.
\label{lema:recoveryset}
\end{lemma}
\begin{proof}
Appendix ~\ref{proofapp} contains the proof.
\end{proof}

After the phantomization phase is complete, all phantom nodes that contain strong
incoming edges are members of the recovery set.  Once a set of phantomized nodes
with strong edges are detected, the building process described above can begin
from those nodes.

%Recovery uses the \emph{recovery} message to distinguish the process of
%searching for the recovery set.
When a phantom node with zero strong count %in the recovery set
%(i.e. a phantom node that has strong incoming edges)
receives a \emph{recovery}
message
it simply updates its state to \emph{recovering} and
propagates the \emph{recovery} message to $\outneighbors$. As in phantomization, when it receives all
\emph{return} messages back, it changes its state to \emph{recovered}. Also, as in phantomization
and building, a spanning tree and \emph{return} messages will be used to track
the progress of recovery.
If a node receives a \emph{recovery} message and that node has incoming
strong edges, it immediately starts building instead of recovering.

Note that it is possible for a build message to reach a node, $x$, that is still in
the forward step of the recovery process.  To accommodate this, when $x$
receives all its \emph{return} messages back from $\Gamma_{\rm out}(x)$,
$x$ checks to see if it now belongs to the recovery set (i.e. if it has positive strong
count). If it still does not belong to the recovery set, then the return
message is sent back to the parent as usual. If it is part of recovery set, the
node starts the build process. Any building node that sends the
\emph{return} message to its parent is not part of the phantomized subgraph and becomes
healthy, i.e. it is no longer part of any collection process.

%only if a node that is already in the recovering state
%received a recovery message or it is healthy node or no out-neighbors exist to
%propagate the message. If a node in recovering state received all the return
%messages and it does not have any strong incoming edge, it changes itself into
%recovered state and sends the return message to parent. The check to have
%strong incoming edge is to make sure that the node does not belong to recovery
%set as a result of any build process initiated by other nodes. If it belong to
%the recovery set, the node initiates the build process and at the end of the
%build process, it sends return message to the parent.





%\subsubsection*{Deletion}
\paragraph{Deletion Phase:}
%Deletion of edges by the Adversary is different from deletion of edges by a collector. Collectors
%use different type of message to delete the edges.
If the recovery phase finishes and the initiator, $x$, still has no incoming strong edge, it issues
\emph{plague delete} messages to $\Gamma_{\rm out}(x)$. %along its outgoing edges.
As the name suggests, this message is contagious and deadly. Any node receiving it decrements
its phantom count by one and (if the recipient has no incoming strong edges)
it sends the \emph{plague delete} message along its outgoing edges.
Once a node has no edges (i.e. phantom, strong, and weak count are zero), it is deleted.
Unlike the other phases, there is no \emph{return} message for the \emph{plague delete} message.
%traversal. A node will not be deleted until all its incoming edges are deleted.
%Plague delete
%does not have two step process. It is single step process. Once the initiator decided it cannot
%build on the return message from the out-neighbors, it initiates the plague delete message.
%The plague delete message deletes all the out-going edges which in turn decreases the counters of
%the out-neighbors. If the neighbors are part of the recovered nodes in the recovery, they
%propagate plague delete messages to out-neighbors. A node will stop plague delete only if it has
%any strong incoming edges. A node will delete itself when a node does not have any out-going
%neighbors and no incoming edges. Again as we reasoned the recovery part, plague delete is sent
%for each edge and each edge is being deleted as plague delete is sent.
\begin{comment}
\begin{lemma}[Plague Delete]
There are O($E_{a}$) plague delete messages sent and they require O(Rad(i,$G_{a}$)) time to finish the plague deletion.
\end{lemma}
\begin{proof}
In every forward step of the plague delete, all outgoing edges are consumed, and the message spreads outward by
a radius of 1. Therefore, in O(Rad(i,$G_{a}$)) time it will spread to the entire subgraph.
\end{proof}

\begin{lemma}
All nodes that have a strong count at the end of phantomization are live.
\end{lemma}
\begin{proof}
Assume there exists a node, $x$, that is dead, but has a strong count. If the
node is dead, it must either (1) be the initiator, or (2) belong to the purely dependent
set. If the $x$ is an initiator, then the strong count comes from the
build set, but there exists a path from $R$ to all nodes in the build set. So option (1)
contradicts our assumption. After
phantomization, all of the nodes in the purely dependent set will only have incoming
phantom edges. So option (2) also contradicts our
assumption.
\end{proof}
\end{comment}
\begin{lemma}[Time Complexity]
SCA finishes in O(Rad(i,$G_{a}$)) time, where $G_{a}$ is the graph
induced by affected nodes, i is the initiator, and Rad(i, $G_{a}$) is the radius
of the graph from from i.
\label{TimeC}
\end{lemma}
\begin{proof}
Appendix ~\ref{proofapp} contains the proof.
\end{proof}

\begin{observation}
SCA finishes in O($E_a$) time, since Rad(i, $G_{a}$) can be O($E_a$).
\end{observation}
\begin{lemma}[Communication Complexity]
SCA sends O($E_{a}$) messages, where $E_{a}$ is the number of edges
in the graph induced by affected nodes.
\label{CCC}
\end{lemma}
\begin{proof}
Appendix ~\ref{proofapp} contains the proof.
\end{proof}


\begin{lemma}[Message Size Complexity]
SCA sends messages of O(log(n)) size, where n is the total number of nodes
in the graph.
\label{MSCP}
\end{lemma}
\begin{proof}
Appendix ~\ref{proofapp} contains the proof.
\end{proof}


\begin{theorem}
All dead nodes will be deleted at the end of the correction phases.
\label{pro:livenesss}
\end{theorem}
\begin{proof}
Appendix ~\ref{proofapp} contains the proof.
\end{proof}

\begin{theorem}
No live node will be deleted at the end of the correction phases.
\label{pro:safetys}
\end{theorem}

\begin{proof}
Appendix ~\ref{proofapp} contains the proof.


\end{proof}

\begin{comment}
\begin{center}
\begin{figure}
\centering
\includegraphics[scale = 0.25]{figs/vennpng}
\caption{ Venn diagram of relationship among sets used by algorithm.
}
\label{fig:vennpng}
\end{figure}
\end{center}

\def \setA{ (0,0) circle (1cm) }
\def \setB{ (1.5,0) circle (1cm) }
\def \setC{ (3.0,0) circle (1cm) }
\def \setU{ (-2, -1.5) rectangle (5.5, 2.0) }
\begin{center}
\begin{tikzpicture}
\draw \setU node[below left]{$U$};
\begin{scope}
\clip \setA;
\fill[gray] \setB;
\end{scope}
\begin{scope}
\clip \setA;
\clip \setB;
\fill[white] \setC;
\end{scope}
\draw \setA node[left] {$A$};
\draw \setB node[right] {$B$};
\draw \setC node {$C$};
\end{tikzpicture}
\end{center}
\end{comment}

%\input{figs/Diagram1}
\section{Multi-Collector Algorithm (MCA)}
\label{multi}
We now discuss the case where mutations in $G$ are not serialized and the Adversary is allowed to create and delete edges in $G$ during the collection process.
As a consequence, multiple deletion events might create collection processes
with overlapping (induced) subgraphs, possibly trying to perform different phases of the
algorithm at the same time. 
We assign unique identifiers to collection processes to break the symmetry among multiple collectors and a transaction approach to ordering the computations of collectors if collection operations conflict. By doing this, we allow each collection process to operate in \emph{isolation}.
%We discuss in detail separately below about the symmetry breaking technique, our transaction approach of computation ordering, and the handling of the mutations of the Adversary during the collection processes.
%Our multi-collector algorithm also has four phases similar to the single collector algorithm,
Appendix \ref{multialgo} contains the detailed pseudocode of the multi-collector algorithm. 

\begin{definition}[Isolation]
A collection process is said to proceed in isolation if either (1) the affected subgraph does not mutate, or (2) mutations to the affected subgraph occur, but they do not affect the decisions of the initiator to delete or build, and do not affect the decisions of the recovery set to build by the correction phases.
\end{definition}

%\subsection{Collection identifiers}


\paragraph{Symmetry Breaking for Multiple Collectors:}
Each collection process has a unique id. If there are two or more (collection) processes acting on the same shared
subgraph, the collection process with the higher id (higher priority) will
acquire ownership of the nodes (of the shared subgraph).
%A node changes
%ownership by marking itself with the process id. It does this when it
%receives a message from a process with higher id.
When a node receives a message from a higher priority process, it marks itself
with that new higher id.

The phantomization phase of the multi-collector does not mark nodes with the
collection process id, however, as the simultaneous efforts of multiple
collection processes to phantomize do not interfere in a harmful way.
Moreover, when an initiator node, $I$, receives a \emph{phantomize} message from a higher
priority collection process, $I$ will unmark its collection id, and will,
therefore, loses its status as an initiator.

Another graph, which we call the \emph{collection process graph} and denote by  $C$, is formed
by treating each set of nodes with the same collection id as a single node.
Edges in $G$ are included in $C$ if and only if they connect nodes that are part
of different collection processes.
To avoid ambiguity,
we refer to nodes in $C$ as \emph{entities}.

\begin{lemma}
The collection process graph is eventually acyclic.
\label{lem:acyclic}
\end{lemma}
\begin{proof}
Appendix ~\ref{proofapp} contains the proof.
\end{proof}

\paragraph{Recovery Count:}
Consider the configuration of nodes in Fig.~\ref{fig:recoverycount}.
Initially, we have strong paths connected to both $N_6$ and $N_1$
and after these paths are deleted we have two collection processes. Assume that these two collection processes,
with priority $I_6$ and $I_1$, complete their phantomizations phases, and process $I_6$ begins its
recovery phase. Process $I_6$ should identify node $N_4$ as belonging to the recovery set,
but if process $I_1$ has not completed recovering and building,  it will not. Therefore, the recovery phase for
$I_6$ will complete and prematurely delete $N_6$.

We remedy this kind of conflict using {\em recovery count}. %, we introduce the recovery count. 
The recovery
count tracks the number of \emph{recovery} messages each node receives from its
own process. A recovering node can neither send the
\emph{return} message nor start deleting until
the recovery count is equal to the phantom count. The fourth edge type in 
the abstract refers to the recovery count. A recovery edge is also a phantom edge,
and a phantom edge will be converted to a recovery edge when a recover message is
sent for that edge with the same CID.

In the configuration discussed above, this would cause process $I_6$ to stall at
node $N_4$ while it waits for the recovery count and phantom count to become
equal. When process $I_1$ rebuilds the edge to $N_4$ as strong, this condition
is met and process $I_6$ stops waiting and rebuilds.

We assume in the discussion above that $I_6$ has higher priority than $I_1$. If the priority
were reversed, $I_1$ would take over building node $N_4$, but this
introduces additional complexity which we will now address.

\begin{lemma}
If an entity $A$ precedes an entity $B$ topologically in the collection process graph, and
$A$ has a lower priority than $B$, entity $A$ will complete before entity $B$ and both will
complete in isolation.
\label{lem:ordered}
\end{lemma}
\begin{proof}
Appendix ~\ref{proofapp} contains the proof.
\end{proof}

%Fig.~\ref{fig:recoverycount} describes the significance
%of the recovery count. The randomization involved in the collections will not
%guarantee the correctness. To regulate the algorithm based on the topology of
%the graph, recovery count helps to understand the topology of the related subgraph
%for a node. This recovery count helps to avoid premature deletion which is one of
%the most fundamental guarantee required by any garbage collectors.
\begin{figure}
\begin{center}
\scalebox{0.6}[0.6]{% Graphic for TeX using PGF
% Title: /home/hkrish/podcpaper/distgc/figs/recoverycount.dia
% Creator: Dia v0.97.2
% CreationDate: Wed Apr 27 14:23:44 2016
% For: hkrish
% \usepackage{tikz}
% The following commands are not supported in PSTricks at present
% We define them conditionally, so when they are implemented,
% this pgf file will use them.
\ifx\du\undefined
  \newlength{\du}
\fi
\setlength{\du}{15\unitlength}
\begin{tikzpicture}
\pgftransformxscale{1.000000}
\pgftransformyscale{-1.000000}
\definecolor{dialinecolor}{rgb}{0.000000, 0.000000, 0.000000}
\pgfsetstrokecolor{dialinecolor}
\definecolor{dialinecolor}{rgb}{1.000000, 1.000000, 1.000000}
\pgfsetfillcolor{dialinecolor}
\definecolor{dialinecolor}{rgb}{1.000000, 1.000000, 1.000000}
\pgfsetfillcolor{dialinecolor}
\pgfpathellipse{\pgfpoint{18.796636\du}{10.048318\du}}{\pgfpoint{2.591538\du}{0\du}}{\pgfpoint{0\du}{2.480237\du}}
\pgfusepath{fill}
\pgfsetlinewidth{0.300000\du}
\pgfsetdash{}{0pt}
\pgfsetdash{}{0pt}
\pgfsetmiterjoin
\definecolor{dialinecolor}{rgb}{0.000000, 0.000000, 0.000000}
\pgfsetstrokecolor{dialinecolor}
\pgfpathellipse{\pgfpoint{18.796636\du}{10.048318\du}}{\pgfpoint{2.591538\du}{0\du}}{\pgfpoint{0\du}{2.480237\du}}
\pgfusepath{stroke}
% setfont left to latex
\definecolor{dialinecolor}{rgb}{0.000000, 0.000000, 0.000000}
\pgfsetstrokecolor{dialinecolor}
\node at (18.796636\du,9.043318\du){ID:$N_1$};
% setfont left to latex
\definecolor{dialinecolor}{rgb}{0.000000, 0.000000, 0.000000}
\pgfsetstrokecolor{dialinecolor}
\node at (18.796636\du,9.843318\du){PC:1};
% setfont left to latex
\definecolor{dialinecolor}{rgb}{0.000000, 0.000000, 0.000000}
\pgfsetstrokecolor{dialinecolor}
\node at (18.796636\du,10.643318\du){RCC:0};
% setfont left to latex
\definecolor{dialinecolor}{rgb}{0.000000, 0.000000, 0.000000}
\pgfsetstrokecolor{dialinecolor}
\node at (18.796636\du,11.443318\du){CID:$I_1$};
\definecolor{dialinecolor}{rgb}{1.000000, 1.000000, 1.000000}
\pgfsetfillcolor{dialinecolor}
\pgfpathellipse{\pgfpoint{24.662738\du}{6.097308\du}}{\pgfpoint{2.483905\du}{0\du}}{\pgfpoint{0\du}{2.377226\du}}
\pgfusepath{fill}
\pgfsetlinewidth{0.100000\du}
\pgfsetdash{}{0pt}
\pgfsetdash{}{0pt}
\pgfsetmiterjoin
\definecolor{dialinecolor}{rgb}{0.000000, 0.000000, 0.000000}
\pgfsetstrokecolor{dialinecolor}
\pgfpathellipse{\pgfpoint{24.662738\du}{6.097308\du}}{\pgfpoint{2.483905\du}{0\du}}{\pgfpoint{0\du}{2.377226\du}}
\pgfusepath{stroke}
% setfont left to latex
\definecolor{dialinecolor}{rgb}{0.000000, 0.000000, 0.000000}
\pgfsetstrokecolor{dialinecolor}
\node at (24.662738\du,5.092308\du){ID:$N_2$};
% setfont left to latex
\definecolor{dialinecolor}{rgb}{0.000000, 0.000000, 0.000000}
\pgfsetstrokecolor{dialinecolor}
\node at (24.662738\du,5.892308\du){PC:1};
% setfont left to latex
\definecolor{dialinecolor}{rgb}{0.000000, 0.000000, 0.000000}
\pgfsetstrokecolor{dialinecolor}
\node at (24.662738\du,6.692308\du){RCC:1};
% setfont left to latex
\definecolor{dialinecolor}{rgb}{0.000000, 0.000000, 0.000000}
\pgfsetstrokecolor{dialinecolor}
\node at (24.662738\du,7.492308\du){CID:$I_1$};
\definecolor{dialinecolor}{rgb}{1.000000, 1.000000, 1.000000}
\pgfsetfillcolor{dialinecolor}
\pgfpathellipse{\pgfpoint{24.992738\du}{14.007308\du}}{\pgfpoint{2.483905\du}{0\du}}{\pgfpoint{0\du}{2.377226\du}}
\pgfusepath{fill}
\pgfsetlinewidth{0.100000\du}
\pgfsetdash{}{0pt}
\pgfsetdash{}{0pt}
\pgfsetmiterjoin
\definecolor{dialinecolor}{rgb}{0.000000, 0.000000, 0.000000}
\pgfsetstrokecolor{dialinecolor}
\pgfpathellipse{\pgfpoint{24.992738\du}{14.007308\du}}{\pgfpoint{2.483905\du}{0\du}}{\pgfpoint{0\du}{2.377226\du}}
\pgfusepath{stroke}
% setfont left to latex
\definecolor{dialinecolor}{rgb}{0.000000, 0.000000, 0.000000}
\pgfsetstrokecolor{dialinecolor}
\node at (24.992738\du,13.002308\du){ID:$N_3$};
% setfont left to latex
\definecolor{dialinecolor}{rgb}{0.000000, 0.000000, 0.000000}
\pgfsetstrokecolor{dialinecolor}
\node at (24.992738\du,13.802308\du){PC:1};
% setfont left to latex
\definecolor{dialinecolor}{rgb}{0.000000, 0.000000, 0.000000}
\pgfsetstrokecolor{dialinecolor}
\node at (24.992738\du,14.602308\du){RCC:1};
% setfont left to latex
\definecolor{dialinecolor}{rgb}{0.000000, 0.000000, 0.000000}
\pgfsetstrokecolor{dialinecolor}
\node at (24.992738\du,15.402308\du){CID:$I_1$};
\definecolor{dialinecolor}{rgb}{1.000000, 1.000000, 1.000000}
\pgfsetfillcolor{dialinecolor}
\pgfpathellipse{\pgfpoint{30.832738\du}{10.712308\du}}{\pgfpoint{2.483905\du}{0\du}}{\pgfpoint{0\du}{2.377226\du}}
\pgfusepath{fill}
\pgfsetlinewidth{0.100000\du}
\pgfsetdash{}{0pt}
\pgfsetdash{}{0pt}
\pgfsetmiterjoin
\definecolor{dialinecolor}{rgb}{0.000000, 0.000000, 0.000000}
\pgfsetstrokecolor{dialinecolor}
\pgfpathellipse{\pgfpoint{30.832738\du}{10.712308\du}}{\pgfpoint{2.483905\du}{0\du}}{\pgfpoint{0\du}{2.377226\du}}
\pgfusepath{stroke}
% setfont left to latex
\definecolor{dialinecolor}{rgb}{0.000000, 0.000000, 0.000000}
\pgfsetstrokecolor{dialinecolor}
\node at (30.832738\du,9.707308\du){ID:$N_4$};
% setfont left to latex
\definecolor{dialinecolor}{rgb}{0.000000, 0.000000, 0.000000}
\pgfsetstrokecolor{dialinecolor}
\node at (30.832738\du,10.507308\du){PC:2};
% setfont left to latex
\definecolor{dialinecolor}{rgb}{0.000000, 0.000000, 0.000000}
\pgfsetstrokecolor{dialinecolor}
\node at (30.832738\du,11.307308\du){RCC:1};
% setfont left to latex
\definecolor{dialinecolor}{rgb}{0.000000, 0.000000, 0.000000}
\pgfsetstrokecolor{dialinecolor}
\node at (30.832738\du,12.107308\du){CID:$I_6$};
\definecolor{dialinecolor}{rgb}{1.000000, 1.000000, 1.000000}
\pgfsetfillcolor{dialinecolor}
\pgfpathellipse{\pgfpoint{37.872738\du}{4.967308\du}}{\pgfpoint{2.483905\du}{0\du}}{\pgfpoint{0\du}{2.377226\du}}
\pgfusepath{fill}
\pgfsetlinewidth{0.100000\du}
\pgfsetdash{}{0pt}
\pgfsetdash{}{0pt}
\pgfsetmiterjoin
\definecolor{dialinecolor}{rgb}{0.000000, 0.000000, 0.000000}
\pgfsetstrokecolor{dialinecolor}
\pgfpathellipse{\pgfpoint{37.872738\du}{4.967308\du}}{\pgfpoint{2.483905\du}{0\du}}{\pgfpoint{0\du}{2.377226\du}}
\pgfusepath{stroke}
% setfont left to latex
\definecolor{dialinecolor}{rgb}{0.000000, 0.000000, 0.000000}
\pgfsetstrokecolor{dialinecolor}
\node at (37.872738\du,3.962308\du){ID:$N_5$};
% setfont left to latex
\definecolor{dialinecolor}{rgb}{0.000000, 0.000000, 0.000000}
\pgfsetstrokecolor{dialinecolor}
\node at (37.872738\du,4.762308\du){PC:1};
% setfont left to latex
\definecolor{dialinecolor}{rgb}{0.000000, 0.000000, 0.000000}
\pgfsetstrokecolor{dialinecolor}
\node at (37.872738\du,5.562308\du){RCC:1};
% setfont left to latex
\definecolor{dialinecolor}{rgb}{0.000000, 0.000000, 0.000000}
\pgfsetstrokecolor{dialinecolor}
\node at (37.872738\du,6.362308\du){CID:$I_6$};
\definecolor{dialinecolor}{rgb}{1.000000, 1.000000, 1.000000}
\pgfsetfillcolor{dialinecolor}
\pgfpathellipse{\pgfpoint{37.812738\du}{15.322308\du}}{\pgfpoint{2.591538\du}{0\du}}{\pgfpoint{0\du}{2.480237\du}}
\pgfusepath{fill}
\pgfsetlinewidth{0.300000\du}
\pgfsetdash{}{0pt}
\pgfsetdash{}{0pt}
\pgfsetmiterjoin
\definecolor{dialinecolor}{rgb}{0.000000, 0.000000, 0.000000}
\pgfsetstrokecolor{dialinecolor}
\pgfpathellipse{\pgfpoint{37.812738\du}{15.322308\du}}{\pgfpoint{2.591538\du}{0\du}}{\pgfpoint{0\du}{2.480237\du}}
\pgfusepath{stroke}
% setfont left to latex
\definecolor{dialinecolor}{rgb}{0.000000, 0.000000, 0.000000}
\pgfsetstrokecolor{dialinecolor}
\node at (37.812738\du,14.317308\du){ID:$N_6$};
% setfont left to latex
\definecolor{dialinecolor}{rgb}{0.000000, 0.000000, 0.000000}
\pgfsetstrokecolor{dialinecolor}
\node at (37.812738\du,15.117308\du){PC:1};
% setfont left to latex
\definecolor{dialinecolor}{rgb}{0.000000, 0.000000, 0.000000}
\pgfsetstrokecolor{dialinecolor}
\node at (37.812738\du,15.917308\du){RCC:1};
% setfont left to latex
\definecolor{dialinecolor}{rgb}{0.000000, 0.000000, 0.000000}
\pgfsetstrokecolor{dialinecolor}
\node at (37.812738\du,16.717308\du){CID:$I_6$};
\pgfsetlinewidth{0.100000\du}
\pgfsetdash{{1.000000\du}{0.400000\du}{0.200000\du}{0.400000\du}}{0cm}
\pgfsetdash{{0.500000\du}{0.200000\du}{0.100000\du}{0.200000\du}}{0cm}
\pgfsetbuttcap
{
\definecolor{dialinecolor}{rgb}{0.000000, 0.000000, 0.000000}
\pgfsetfillcolor{dialinecolor}
% was here!!!
\pgfsetarrowsstart{stealth}
\definecolor{dialinecolor}{rgb}{0.000000, 0.000000, 0.000000}
\pgfsetstrokecolor{dialinecolor}
\pgfpathmoveto{\pgfpoint{22.308389\du}{5.199560\du}}
\pgfpatharc{271}{202}{3.697283\du and 3.697283\du}
\pgfusepath{stroke}
}
\pgfsetlinewidth{0.100000\du}
\pgfsetdash{{0.500000\du}{0.200000\du}{0.100000\du}{0.200000\du}}{0cm}
\pgfsetdash{{0.500000\du}{0.200000\du}{0.100000\du}{0.200000\du}}{0cm}
\pgfsetbuttcap
{
\definecolor{dialinecolor}{rgb}{0.000000, 0.000000, 0.000000}
\pgfsetfillcolor{dialinecolor}
% was here!!!
\pgfsetarrowsstart{stealth}
\definecolor{dialinecolor}{rgb}{0.000000, 0.000000, 0.000000}
\pgfsetstrokecolor{dialinecolor}
\pgfpathmoveto{\pgfpoint{19.357642\du}{12.618799\du}}
\pgfpatharc{149}{95}{4.095811\du and 4.095811\du}
\pgfusepath{stroke}
}
\pgfsetlinewidth{0.100000\du}
\pgfsetdash{{0.500000\du}{0.200000\du}{0.100000\du}{0.200000\du}}{0cm}
\pgfsetdash{{0.500000\du}{0.200000\du}{0.100000\du}{0.200000\du}}{0cm}
\pgfsetbuttcap
{
\definecolor{dialinecolor}{rgb}{0.000000, 0.000000, 0.000000}
\pgfsetfillcolor{dialinecolor}
% was here!!!
\pgfsetarrowsstart{stealth}
\definecolor{dialinecolor}{rgb}{0.000000, 0.000000, 0.000000}
\pgfsetstrokecolor{dialinecolor}
\pgfpathmoveto{\pgfpoint{31.427702\du}{13.071727\du}}
\pgfpatharc{152}{99}{5.019674\du and 5.019674\du}
\pgfusepath{stroke}
}
\pgfsetlinewidth{0.100000\du}
\pgfsetdash{{0.500000\du}{0.200000\du}{0.100000\du}{0.200000\du}}{0cm}
\pgfsetdash{{0.500000\du}{0.200000\du}{0.100000\du}{0.200000\du}}{0cm}
\pgfsetbuttcap
{
\definecolor{dialinecolor}{rgb}{0.000000, 0.000000, 0.000000}
\pgfsetfillcolor{dialinecolor}
% was here!!!
\pgfsetarrowsstart{stealth}
\definecolor{dialinecolor}{rgb}{0.000000, 0.000000, 0.000000}
\pgfsetstrokecolor{dialinecolor}
\pgfpathmoveto{\pgfpoint{35.340870\du}{4.871367\du}}
\pgfpatharc{260}{202}{5.580074\du and 5.580074\du}
\pgfusepath{stroke}
}
\pgfsetlinewidth{0.100000\du}
\pgfsetdash{{0.500000\du}{0.200000\du}{0.100000\du}{0.200000\du}}{0cm}
\pgfsetdash{{0.500000\du}{0.200000\du}{0.100000\du}{0.200000\du}}{0cm}
\pgfsetbuttcap
{
\definecolor{dialinecolor}{rgb}{0.000000, 0.000000, 0.000000}
\pgfsetfillcolor{dialinecolor}
% was here!!!
\pgfsetarrowsstart{stealth}
\definecolor{dialinecolor}{rgb}{0.000000, 0.000000, 0.000000}
\pgfsetstrokecolor{dialinecolor}
\pgfpathmoveto{\pgfpoint{40.301136\du}{14.218733\du}}
\pgfpatharc{52}{-52}{5.245192\du and 5.245192\du}
\pgfusepath{stroke}
}
\pgfsetlinewidth{0.100000\du}
\pgfsetdash{{0.500000\du}{0.200000\du}{0.100000\du}{0.200000\du}}{0cm}
\pgfsetdash{{0.500000\du}{0.200000\du}{0.100000\du}{0.200000\du}}{0cm}
\pgfsetbuttcap
{
\definecolor{dialinecolor}{rgb}{0.000000, 0.000000, 0.000000}
\pgfsetfillcolor{dialinecolor}
% was here!!!
\pgfsetarrowsstart{stealth}
\definecolor{dialinecolor}{rgb}{0.000000, 0.000000, 0.000000}
\pgfsetstrokecolor{dialinecolor}
\pgfpathmoveto{\pgfpoint{25.444682\du}{11.620204\du}}
\pgfpatharc{6}{-9}{12.296238\du and 12.296238\du}
\pgfusepath{stroke}
}
\pgfsetlinewidth{0.100000\du}
\pgfsetdash{{0.500000\du}{0.200000\du}{0.100000\du}{0.200000\du}}{0cm}
\pgfsetdash{{0.500000\du}{0.200000\du}{0.100000\du}{0.200000\du}}{0cm}
\pgfsetbuttcap
{
\definecolor{dialinecolor}{rgb}{0.000000, 0.000000, 0.000000}
\pgfsetfillcolor{dialinecolor}
% was here!!!
\pgfsetarrowsstart{stealth}
\definecolor{dialinecolor}{rgb}{0.000000, 0.000000, 0.000000}
\pgfsetstrokecolor{dialinecolor}
\pgfpathmoveto{\pgfpoint{30.304033\du}{8.346620\du}}
\pgfpatharc{333}{283}{4.665507\du and 4.665507\du}
\pgfusepath{stroke}
}
\pgfsetlinewidth{0.100000\du}
\pgfsetdash{}{0pt}
\pgfsetdash{}{0pt}
\pgfsetbuttcap
{
\definecolor{dialinecolor}{rgb}{0.000000, 0.000000, 0.000000}
\pgfsetfillcolor{dialinecolor}
% was here!!!
\definecolor{dialinecolor}{rgb}{0.000000, 0.000000, 0.000000}
\pgfsetstrokecolor{dialinecolor}
\draw (24.150000\du,18.662500\du)--(26.600000\du,18.562500\du);
}
\pgfsetlinewidth{0.100000\du}
\pgfsetdash{}{0pt}
\pgfsetdash{}{0pt}
\pgfsetbuttcap
{
\definecolor{dialinecolor}{rgb}{0.000000, 0.000000, 0.000000}
\pgfsetfillcolor{dialinecolor}
% was here!!!
\pgfsetarrowsend{stealth}
\definecolor{dialinecolor}{rgb}{0.000000, 0.000000, 0.000000}
\pgfsetstrokecolor{dialinecolor}
\draw (25.150000\du,18.662500\du)--(25.074690\du,16.433213\du);
}
\end{tikzpicture}
}
\caption{We depict two collection processes in the recovery phase.
 Each circle is a node.
Node properties are: node id (ID),
phantom count (PC), recovery count (RCC), collection id (CID). Bold borders denote
initiators, and dot and dashed edges denote phantom edges.
%In the above
%scenario, $N_4$ belongs to collection process $N_6$. The RCC is 1 in $N_4$
%due to edge $N_6 \rightarrow N_4$. Edge $N_2 \rightarrow N_4$ does not increment
%RCC in $N_4$ due to its smaller collection id,
%preventing $N_4$ from sending a \emph{return} message to $N_6$. The RCC
%creates an ordering in the operations of the two collection processes and
%avoids premature deletion of $N_6$ and its affected subgraph.
}
\label{fig:recoverycount}
\end{center}
\vspace{-8mm}
\end{figure}

\paragraph{Transaction Approach:}
Consider a \emph{recover} message, $m$, arriving at node $N_4$ from $N_2$ (Fig.~\ref{fig:recoverycount}). This time, assume that the
collection id of $I_1$ is higher than $I_6$. % The higher
%priority collection process needs to take control.  
If $N_4$ is in the recovered
state, the collection id on $x$ is updated with the collection id in $m$
($x.CID \leftarrow m.CID$).  If, however, $x$ is recovering, i.e.
it is waiting for \emph{return} messages from its $\outneighbors$, a \emph{return} message, $r$, is immediately sent to $N_4$'s
parent, $N_4$'s parent is set to point to $N_2$, and the
\emph{re-recover flag} on $x$ is set. %, and the \emph{start recovery over (SRO)}
%flag on $r$.
The re-recover flag will cause recovery to restart on $x$ once the
current recovery operation finishes. This will allow the higher priority
collection to take over managing the subgraph.

In addition, a flag called \emph{start recovery over (SRO)} is set on the
\emph{return} message.
The SRO propagates back to $N_6$, where the recovery count will
be reset and the entire recovery phase will start over,
but with a slightly
higher collection priority, $I_6'$, which is still lower than $I_1$.  The slightly higher priority
(signified by the prime)
doesn't change the relative priority compared to another independent collection process, it merely
ensures that any
increments to the recovery count come from the new, restarted recovery
operation and not leftover messages from the old one ($I_6$).
After this, the new recovery phase
will not proceed until the phantom count is equal to the recovery count ($PC = RCC$), which ensures that the new recovery
does not start until the original \emph{recovery} phase is complete.
It is easy to see that the new recovery
triggered by the SRO flag is similar to the rollback of a transactional system.

%In dynamic system, when collectors share affected subgraphs, the collectors
%might be processing different phase of the algorithm. Synchronization among
%collectors is essential for correctness of the algorithm as the
%algorithm makes decision about liveness by assuming all the nodes in the
%affected subgraph are in the same state.

When a node changes ownership, the node has to join the phase of the new collection.
Each
node has three redo flags: rephantomize, rebuild,
and re-recover.
%Apart from all the redo flag, there is also  "start
%over" flag. when a initiator sees the start over flag set on the node, it
%simply redo the recently executed phase again to synchronize the affected
%subgraph. Every start over process is to let the initiator or the affected
%node understand the changes in the subgraph it is part of. This transaction
%approach helps nodes to change ownership when it becomes member of dependent
%set of multiple initiator.
These flags only take effect
after all \emph{return} messages % from the $\outneighbors$
have come back.
This way the messages of old collection processes will never be floating
in the system.
%All messages in flight are related to the current collections.
%This ensures when a node is dead, there will be no messages supposed to be
%received by the node.

%\subsection{Phantomization}

\begin{lemma}
If an entity $A$ precedes an entity $B$ in the collection process
graph $C$ with respect to a topological ordering, and $A$ has higher priority than $B$, entity $A$ will take ownership of
entity $B$ and $A$ will proceed in isolation.
\label{lem:takeover}
\end{lemma}
\begin{proof}
Because $A$ has higher priority, it takes ownership of every node it
touches and is thus unaware of $B$. So it proceeds in isolation. If $B$ loses
ownership of a recovered node, it will not affect isolation because a recovered
node has received \emph{recovery} messages on all its incoming edges ($RCC=PC$),
and has already sent its return.
%Thus, no further messages will be sent to it
%or received from it.
If $B$ loses a node that is building or recovering,
%this would violate isolation, so
%the recovery or build
$B$ is forced to start over. In
the new recovery or build phase, $B$ will precede $A$ in the  topological order, and both will
proceed in isolation. % by pointing to the
%nodes A has taken from B.
\end{proof}

\begin{theorem}
All collection processes will complete in isolation.
\label{thm:alliso}
\end{theorem}
\begin{proof}
By Lemma \ref{lem:acyclic} we know that the collection process graph will eventually
be topologically ordered, and the ordering will not violate
isolation given by Lemma \ref{lem:takeover}. Once ordered, the
collection processes will then complete in order proven by Lemma \ref{lem:ordered}.
\end{proof}

\begin{corollary}
In quiescent state (i.e. one in which the Adversary takes no further actions on $G_{a}$), with $p$ active collection processes, our algorithm finishes in $O(Rad_{all})$ time where $G_{a}$ is the
affected subgraphs of all connected collection process and
$Rad_{all}$ is the sum of the radii of the affected subgraphs of all $p$ initiators.
\end{corollary}

\paragraph{Handling the Mutations of The Adversary:}
Creation of edges and nodes by the Adversary poses no difficulty. If an edge is
created that originates on a phantomized node, the edge is created as a phantom
edge. If a new node, $x$, is created, and the first edge to the node comes from
a node, $y$, then $x$ is created in the same state as $y$ (e.g., phantomized,
recovered, etc.).

\begin{lemma}[Creation Isolation]
Creation of edges and nodes by the Adversary does not violate isolation.
\label{creationI}
\end{lemma}
\begin{proof}
The addition of edges cannot affect the isolation of a graph because (1) addition
of an edge cannot cause anything to become dead and (2) if an edge is created to
or from any edge in a collection process, then the process was already live by
Axiom~\ref{ax:immut}.
\end{proof}

Deletion is comparatively difficult, though.
If the Adversary deletes a phantom edge $x \rightarrow y$ from collection $x_1$, and $\Gamma_{\rm in}(y)$ is not empty,
%collection process $x_1$ and two nodes within it, $x$ and $y$, an edge exists between
%them, $x\rightarrow y$, and $x$ is phantomized,
it is not enough to reduce
the phantom count on $y$ by 1. In this case, $y$ is made the
initiator for a new collection process, $y_2$, such that $y_2$ has higher priority than $x_1$. If $y$ is in
the recovering state, it sends \emph{return} with $SRO = true$,
and re-recovers $y$.  If $x$ is not phantomized, the
procedure is similar to the single collector algorithm.

\begin{lemma}[Deletion Isolation]
Edge deletion by the Adversary does not violate isolation.
\label{deletionI}
\end{lemma}
\begin{proof}
Appendix ~\ref{proofapp} contains the proof.
\end{proof}


\begin{theorem}[Liveness]
Eventually, all dead nodes will be deleted.
\label{liveness}
\end{theorem}
\begin{proof}
By Axiom \ref{ax:immut}, we know the Adversary cannot create any
edges to dead nodes.
Theorem \ref{pro:livenesss} proves that if a  collection process
works in isolation, all dead nodes will be deleted.
%When the Adversary deletes an edge in the affected subgraph, a new initiator
%is created for the node that lost the edge regardless of the type
%of the edge.
Lemma \ref{deletionI} proves that the deletion of an edge in the
affected subgraph does not affect isolation property.
%So this new initiator proceeds the algorithm in
%an attempt to finish the collection process.
Lemma \ref{thm:alliso} proves that indeed all collection processes
complete in isolation. Theorem \ref{pro:livenesss} proves all collection
processes in isolation delete dead nodes. Thus, eventually, all dead nodes will be deleted.
%
%A collection process is finished in isolation if no other %collection
%process change the state of any node in the affected subgraph of
%the process. If the affected subgraph nodes experience a change
%in the state or encountered a different collection, based on %collection id the node decides the collection process it belongs %to. If the states change, the collection process initiator node is %notified. The collection id avoids the livelock among the %collection process using total ordering among the collection id. %So any collection process graph created by processes will become %acylic and totally ordered as proved in \ref{lem:acyclic}, %%%%%%\ref{lem:ordered} and \ref{lem:takeover}.
\end{proof}

\begin{theorem}[Safety]
 No live nodes will be deleted.
\label{safety}
\end{theorem}
\begin{proof}
Theorem \ref{pro:safetys} proves that if a  collection process
works in isolation, no live nodes of $G$ will be deleted. When the Adversary adds a strong edge in the affected subgraph, Lemma
\ref{creationI} proves that it does not violate the isolation
property. Theorems \ref{pro:safetys} and \ref{thm:alliso} prove that no live nodes will be deleted.
\end{proof}




%\begin{proofs}

%\end{proofs}

%==% \subsection{Recovery}
%==% Multi-collector recovery phase requires the recovery count to match with
%==% phantom count to send the return back to parent. This count being
%==% matched concludes the fact that all the possible path to the node
%==% is being explored. A node with one or more unexplored path means
%==% some of the incoming phantom edges are not recovery edges. Apart
%==% from the recovery count, the recovery part of the algorithm is
%==% similar to the phantomization.

%==% A healthy node never reacts to the recovery messages and shun the
%==% further recovery process by sending a return message to the sender
%==% with start over flag enabled to notify the initiator about change
%==% in the affected subgraph.
%==% A phantomizing node always joins the recovery process when a recovery
%==% hits the node. A non-initiator node responds to the ongoing phantomization
%==% by sending a return message to the parent and transfers the ownership to the
%==% recovery process. The ongoing phantomization is not interupted by setting
%==% the rerecover flag in the node. This transaction bit ensures the collection
%==% are not stalled and ownership changes seamlessly and the nodes are synced
%==% to the new collection. On the other hand, a phantomized node simply process
%==% the recovery message by incrementing the recovery message and moving into
%==% recovering/building state by assinging collection id, parent and sending recovery/
%==% build messages to out-neighbors accordingly.

%==% A recovering node interacting with a recovery message requires various
%==% parameters into decision. Second phase of algorithm uses collection id on
%==% each node. A lower collection id recovery message will be simply ignored by
%==% the higher collection id recovering node. This ensures that the lower collection
%==% id makes the decision about the collection before higher collection id. If the
%==% higher collection id can eventually make the lower collection id part of
%==% it, then it will eventually happen. So this ordering helps to solve the
%==% randomization involved in the collection id. A recovering node of the same
%==% collection simply increments the recovery count and returns the message to
%==% the sender. The node will only send return message to the parent if the
%==% recovery count matched with phantom count and it received all the return
%==% messages from the out-neighbors. If a recovering node received a
%==% recovery message from a higher collection id, then the ownership is changed
%==% with return message to old parent enabling start over. This ownership change
%==% is different from other ownership change because this requires to erase all the
%==% older recovery count. So the recovery count is reset to 1 ensuring only the recent
%==% recovery message is recorded as received. This also involves rerecovering if
%==% the return messages are pending and otherwise, sending recovery messages to
%==% out-neighbors.

%==% A recovered node receiving a recovery message is only possible when the
%==% predecessor of the node changes the ownership. So the node simply reassigns
%==% the ownership and states and recovery count. A building node with higher
%==% collection will return the message to sender with transactional flag set. For
%==% a building node, the edges are transformed from phantom to either or weak. To
%==% redo the transaction, rephantomize and rerecover flag is set to true. So that
%==% if the node can actually build, it will rebuild the edges.

%==%




%==% \subsection{Building}

%==% As far as the build message is concerned, when a node receives a build message,
%==% by default the edge is transformed into either strong or weak. The building
%==% process will continue to their out-neighbors based on the state. A healthy or
%==% building node will return the message since they do not have to propogate the
%==% message to their out-neighbors. A phantomizing node will change ownership
%==% like anyother phase if the higher collection message is received by the node.
%==% A phantomized node will spread the build process and will register itself to
%==% be part of the collection by saving the collection id. A recovering node
%==% that is waiting to get the return messages from out-neigbhors will react using
%==% general principle of changing ownership if the higher collection build message
%==% is received. Otherwise, it sends return message back to the source. Since the node
%==% is waiting, on the recovery return messages, the node might start a build again.
%==% If a recovering node with all the return message received receives a building
%==% message, it is evident the node is waiting for recovery count to be equal to
%==% phantom count, so the higher collection id build message changes the ownership
%==% and builds the edges. On the other hand, if the lower or same collection build
%==% message reaches the recovering node with unmatched counts, will
%==% simply start the building again. A recovered node will start updating rcc and build
%==% the edges of out-neighbors and propogate the build process.

%==% \subsection{Deletion}

%==% \begin{lemma}
%==% Liveness
%==% \end{lemma}

%==% \begin{lemma}
%==% Safety
%==% \end{lemma}

%==% \begin{lemma}
%==% Time Complexity
%==% \end{lemma}





\section{Conclusions}
\label{section:conclusions}

We have described a hybrid reference counting algorithm for garbage collection
in distributed systems. The algorithm improves on the previous
efforts by eliminating the need for centralization, 
global barriers, back references for every edge, and object migration. Moreover, it achieves efficient time and
space complexity. %; the previous algorithm provide no or inefficient
%time and space guarantees.
Furthermore, the algorithm is stable against
concurrent mutations in the reference graph. %(i.e., the reference graph can
%change during the algorithm's execution).
The main idea was to develop a technique
to identify the supporting set (nodes that prevent a given node from being
dead) and handle the synchronization of multiple collection processes.
We believe that our techniques
might be of independent interest in solving other fundamental problems that
arise in distributed computing. 

In future work, we hope to address how to provide an order in which the dead
nodes will be cleaned up, permitting some kind of ``destructor'' to be run, and
to address fault tolerance. In addition, we hope to implement the proposed
algorithm and compare its performance with previous algorithms using different
benchmarks.


\begin{comment}
\bibliographystyle{plain}
\bibliography{dgcbib}
\end{comment}
\appendix
\section{Appendices}
\label{section:appendix}
\subsection{Terminology}
\begin{description}
%\item[which(bit)] Each object has two reference counts in the header. which
%gives the index of the strong reference count.
\item[SC] Strong Reference Coung
% refers to strong count and WC refers to Weak count.
% refers to Reference count. Each object has strong and
%weak reference count.
\item[WC] Weak Reference Count

\item[PC] Phantom Reference Count 
%refers to Phantom reference count. This count is used by
%collector during collection.

%\item[PhantomNodeFlag] True when the node sent phantom %messages to all $\Gamma_{out}$.

 \item[RCC] Recovery Count
\item[CID] Collector Identifier 
%is a tuple of two 
%entities. $<$ Unique Id (Uid), Version Id(Vid) $>$ . \\
%   $CID_1 = CID_2 $ if $CID_1.Uid = CID_2.Uid \& CID_1.Vid = CID_2.Vid$\\
%   $CID_1 < CID_2$ if $CID_1.Uid < CID_2.Uid $ or $ 
%   CID_1.Uid = CID_2.Uid \& CID_1.Vid < CID_2.Vid  $\\

%\item[W] Return Message Waiting Count. 
%Algorithm uses termination detection algorithm technique
%to keep track of different phases of algorithm. Waiting %represents how many
%messages an node expects from out-neighbors.
\item[SRO] Start Recovery Over. 

%Tells the initiator if the recovery process
%need to start over with updated CID if set true.

\item[Parent] Parent Pointer.
% is used by termination detection algorithm to
%co-ordinate process/phase completion.

\item[$<$MSG TYPE,  Receiver, CID, SRO$>$] Sends a message of mentioned type to receiver.
Message types will one of the following: 
Phantomize(PH), Recovery(RC), Build(B), Return(R), Delete(D), Plague Delete(PD). All parameters are not necessary for some messages.

\item[State] State can be one of the following: Phantomizing, Phantomized, Recovering, Recovered, Building, Healthy. 
\end{description}
\begin{description}
\item[Healthy] SC $>$ 0 \& PC $=$ 0 \& RCC $=$ 0 \& PhantomNodeFlag $=$ false 
\& CID $=$ $\bot$ \& All return msgs received
\item[Weakly Supported] WC $>$ 0 \& SC $=$ 0 \& PC $=$ 0 \& RCC
$=$ 0 \& PhantomNodeFlag $=$ false \& CID $=$ $\bot$ \& waiting $=$ 0 
\item[Maybe Delete] PC $=$ RCC \& SC $=$ WC $=$ 0 \& waiting $=$ 0 \& CID $\neq$ $\bot$ \& PC $>$ 0
\item [Simply Dead] PC $=$ RCC \& SC $=$ WC $=$ 0 \& waiting $=$ 0 \& PC $=$ 0
  	\item [Garbage] Simply Dead or  ( Maybe Delete \& Initiator)
\end{description}

\subsection{Single Collector Algorithm}
\label{singlealgo}
\begin{algorithm}[H]
%\algsetup{linenosize=\tiny}
\scriptsize
\caption{Single Collector Algorithm}
\label{Node Automaton}
%\begin{multicols}{2}
\begin{algorithmic}[1]
\Procedure{OnMsgReceive}{}
\If{msg type = PH}
	\State Decrement SC/WC and Increment PC
	\If {state = Phantomizing}
		\State $<$R, msg.sender $>$
	\Else
		\State Parent = msg.sender
		\If {SC = 0}
			\State Toggle SC and WC
			\State $<$PH, $\Gamma_{out}$ $>$ 
			\State state = Phantomizing \& PhantomNodeFlag = true
		\EndIf
		\If {No Out-neighbors or SC $>$ 0}
				\State $<$R, Parent$>$
		\EndIf
	\EndIf
\ElsIf{msg type = RC}
	\If {state = Phantomized}
		\State Parent = msg.sender.
		\If { SC $>$ 0}
			\State state = Building
			\State $<$ B,  $\Gamma_{out}$ $>$ 
		\Else
			\State state = Recovering
			\State $<$RC, $\Gamma_{out}$ $>$ 
		\EndIf
		\If {No Out-neighbors}
			\State $<$R,  msg.sender$>$
		\EndIf
	\Else
		\State $<$ R, msg.sender $>$
	\EndIf
\ElsIf {msg type = B}
	\State Increment SC/WC and decrement PC.
	\If {state = Phantomized or Recovered}
		\State state = Building \& Parent = msg.sender
		\State  $<$B,  $\Gamma_{out}$ $>$ 
	\Else
		\State  $<$R,  msg.sender$>$
	\EndIf
\ElsIf {msg type = PD}
	\State Decrement PC.
	\If {SC= 0 \& WC = 0 \& PC = 0}
	  \State $<$PD , $\Gamma_{out}$ $>$ 
    $\Gamma_{out} = \bot$
  \EndIf
	\If {SC= 0 \& WC = 0 \& PC = 0}
		\State Delete the Node
	\EndIf
\ElsIf {msg type = D}
	\State Decrement SC/WC.
	\If { SC = 0 \& WC $>$ 0}
		\State Toggle WC and SC.
		\State state = Phantomizing.
		\State $<$PH , $\Gamma_{out}$ $>$
	\ElsIf { SC=0 \& WC = 0}
		\State $<$D , $\Gamma_{out}$ $>$	
		\State Delete the node.
	\EndIf
\ElsIf {msg type = R \& All return msgs received} 
	\If{state = Phantomizing}
		\State state = Phantomized
		\If {Not Initiator}
			\State $<$R , Parent $>$
		\Else
			\If {SC $>$ 0 }
				\State state = Building
				\State $<$B , $\Gamma_{out}$ $>$
			\Else
				\State state = Recovering
				\State $<$RC , $\Gamma_{out}$ $>$
			\EndIf
		\EndIf
	\ElsIf{state = Recovering}
		\State state = Recovered
		\If {SC $>$ 0} 
			\State state = Building
			\State $<$B , $\Gamma_{out}$ $>$
		\Else
			\If {Initiator}
				\State $<$PD , $\Gamma_{out}$ $>$
			\Else	
				\State  $<$R , Parent $>$
			\EndIf
		\EndIf
	\ElsIf {state = Building}
		\State state = Healthy
		\If { Not Initiator}
			\State  $<$R , Parent $>$
		\EndIf
	\EndIf
\EndIf
\EndProcedure
\end{algorithmic}
%\end{multicols}
\end{algorithm}	


\subsection{Multi-collector Algorithm}
\label{multialgo}
\begin{algorithm}[H]
\caption{Edge Deletion}
\label{Link Deletion}
\scriptsize
%\begin{multicols}{2}
\begin{algorithmic}[1]
\Procedure{OnEdgeDeletion}{}
\State Decrement SC, WC, or PC
  \If {msg.CID = CID}
    \State Decrement RCC
  \EndIf
	\If {Garbage} 
		\If {state = Recovered}
			\State $<$PD , $\Gamma_{out}$ $>$
		\ElsIf {Simply Dead}
			\State $<$D , $\Gamma_{out}$ $>$
			\State Delete the node.
		\EndIf
	\ElsIf {Weakly Supported}
		\State Toggle SC and WC
    \If {$\Gamma_{out} = \bot$}
      \State return
    \EndIf
		\State CID = new Collector Id
		\State $<$PH , $\Gamma_{out}$ $>$
		\State PhantomNodeFlag $=$ true
	\ElsIf {PC $>$ 0 \& SC $=$ 0}
		\State Toggle SC and WC.
		\If { Not Initiator}
				\State $<$R , Parent, CID, True$>$
		\EndIf
		\If { Waiting for return msgs }
			\State Rerecover = true
			\State Rephantomize = true
		\EndIf
	  \State CID = new Higher Collector Id
	\EndIf
\EndProcedure
\end{algorithmic}
%\end{multicols}
\end{algorithm}	

	
\begin{algorithm}[H]
\caption{On receiving Phantomize messge}
\label{Phantom message received}
\scriptsize
%\begin{multicols}{2}
\begin{algorithmic}[1]
\Procedure{OnPhantomizeReceive}{}
\State Decrement SC/WC and Increment PC.
\If{state = Phantomizing}
	\If{Initiator}
		\If{CID $\geq$ msg.CID}
			\State $<$R , msg.sender, CID $>$
		\Else
			\State Parent $=$ msg.sender
			\State CID $= \bot$ 
		\EndIf
	\Else
		\State $<$R , msg.sender, CID $>$
	\EndIf
\ElsIf {state = Phantomized or Healthy}
	\State Parent = msg.sender
	\State $<$ R , msg.sender, CID $>$
\ElsIf {SC $=$ 0 }
	\State Toggle SC and WC
	\State Parent = msg.sender
	\State state $=$ Phantomizing and PhantomNodeFlag = true
	\State $<$ PH , $\Gamma_{out}$, CID$>$
	\If{$\Gamma_{out}$ is empty}
		\State state = Phantomized
		\State $<$R , Parent$>$
	\EndIf
\ElsIf {state = Building}
	\If {SC $>$ 0 }
		\State $<$R , Msg.sender$>$
	\Else
  	\State Rephantomize = true
		\If {Not an initiator}
			\State $<$R , Parent $>$
			\State Parent = msg.sender
		\Else
			\If {msg.CID $>$ CID}
				\State CID = $\bot$ and Parent = msg.sender
			\Else
				\State $<$R , Msg.sender, CID $>$
				\State Rerecover = true
			\EndIf
		\EndIf
	\EndIf
\EndIf
\EndProcedure
\end{algorithmic}
%\end{multicols}
\end{algorithm}	
	
	
\begin{algorithm}[H]
\caption{On receiving Return message}
\label{ Done message received}
\scriptsize
%\begin{multicols}{2}
\begin{algorithmic}[1]
\Procedure{OnReturnReceive}{}
\If{msg.SRO = true}
  \State SRO = true
\EndIf
\If{All return msgs received \& Phantomizing}
	\If {Initiator or Rerecover = true}
		\State Rerecover = false
		\If {SC$>$0 }
			\State state = Building			
			\State $<$ B , $\Gamma_{out}$, CID $>$
		\Else
			\State state = Recovering
			\State $<$ RC , $\Gamma_{out}$, CID $>$
		\EndIf
	\Else
		\State $<$R , Parent $>$
	\EndIf
\EndIf
\If{All return msgs recieved \& Building}
	\If{SC $>$ 0 }
		\If {not an Initiator}
			\State $<$ R , Parent $>$
		\EndIf
		\State state = Healthy.
	\Else
		\State Rephantomize = false 
		\State Rerecover = true
		\State state = phantomizing
		\State $<$PH , $\Gamma_{out}$, CID $>$
	\EndIf
\EndIf
\If{All return msgs received \& Recovering}
	\If{RCC = PC}
		\If{Initiator}
			\If{SRO = true}
				\State Rerecover = true
	  \State CID = new Slightly Higher Collector Id
				\State SRO = false
			\EndIf
			\If{Rerecover = true or SC $>$0}
				\State Rerecover = false
				\State RCC = 0
				\If{SC $=$ 0}
					\State state = Recovering
					\State $<$RC , $\Gamma_{out}$,CID $>$
				\Else
					\State state = Building
					\State $<$ B , $\Gamma_{out}$, CID $>$
				\EndIf
			\ElsIf{Garbage}
				\State $<$ PD , $\Gamma_{out}$, CID $>$
			\EndIf
		\Else
			\If{SRO = true}
				\State State = Recovered
				\State SRO = false
				\State $<$R , Parent, CID, True $>$
			\ElsIf{Rerecover = true or SC $>$0}
				\State Rerecover = false
				\If{SC $=$ 0}
					\State state = Recovering
					\State $<$ RC , $\Gamma_{out}$, CID$>$
				\ElsIf{SC $>$ 0}
					\State state = Building
					\State $<$ B , $\Gamma_{out}$, CID $>$
				\EndIf
			\ElsIf{Maybe Delete}
				\State state = Recovered
				\State $<$R , Parent$>$
			\EndIf
		\EndIf
	\ElsIf{RCC $\neq$ PC}
		\If {Rerecover = true}
			\State Rerecover = false
			\If{Initiator}
				\State RCC = 0
			\EndIf
			\If{SC $=$ 0}
				\State state = Recover
				\State $<$ RC , $\Gamma_{out}$, CID $>$
			\ElsIf{SC $>$ 0}
				\State state = Building
				\State $<$B , $\Gamma_{out}$, CID $>$
			\EndIf
		\EndIf
	\EndIf
\EndIf
\EndProcedure
\end{algorithmic}
%\end{multicols}
\end{algorithm}	


	
\begin{algorithm}[H]
\caption{On receiving Build message}
\label{Build message received}
\scriptsize
%\begin{multicols}{2}
\begin{algorithmic}[1]
\Procedure{OnBuildReceive}{}
\State Increment SC/WC, and Decrement PC.
  \If {msg.CID = CID}
    \State Decrement RCC
  \EndIf
\If{state = Building or SC $>$ 0}
	\State $<$ R , Msg.Sender, CID $>$
\ElsIf{state = Phantomizing}
	\If{CID $\geq$ msg.CID or CID = $\bot$}
		\State $<$R , msg.sender $>$
	\Else
		\If{Not Initiator}
			\State $<$R , Parent $>$
		\EndIf
		\State Parent = msg.sender
		\State CID = msg.CID and Rerecover = True
	\EndIf
\ElsIf{state = Phantomized}
	\State Parent $=$ msg.sender
	\State CID $=$ msg.CID
	\If{PhantomNodeFlag $=$ True}
		\State state = Building
		\State $<$ B , $\Gamma_{out}$, CID $>$
	\Else
		\State state = Healthy
		\State $<$ R , Parent $>$
	\EndIf
\ElsIf{state = Recovering \& Waiting for return msgs}
	\If{CID$\geq$msg.CID}
		\State $<$R , msg.sender, CID $>$
	\Else
		\If{Not Initiator}
			\State $<$ R , Parent, CID, True $>$
		\EndIf
		\State CID = msg.CID
		\State Parent = msg.Sender
	\EndIf
\ElsIf{state = Recovering \& \\ \hspace{0.35in} All return msgs received \&  RCC $<$ PC}
	\If{CID$\geq$msg.CID}
		\State $<$R , msg.sender, CID $>$
		%\State Rerecover = true
		%\State W = W + 1
		%\State $<$ R  , CID $>$
	\Else
		\If{Not Initiator}
			\State $<$R , Parent , True $>$
		\EndIf
		\State Parent = msg.sender
		\State CID = msg.CID
		%\State W = W + 1
		%\State $<$ R  , CID $>$
  %\State Rerecover = true
	\EndIf
  \State Start Building
\ElsIf{state = Recovered}
	\State Update RCC if required
	\State CID = msg.CID
	\State Parent = msg.sender
	\State state = Building
	\State $<$ B , $\Gamma_{out}$, CID $>$
\EndIf
\EndProcedure
\end{algorithmic}
%\end{multicols}
\end{algorithm}	


\begin{algorithm}[H]
\caption{On receiving Recovery message}
\label{Recovery message received}
\scriptsize
%\begin{multicols}{2}
\begin{algorithmic}[1]
\Procedure{OnRecoveryReceive}{}
\If{S $>$ 0}
	\State $<$R , msg.sender $>$
\ElsIf{state = Phantomizing}
	\If{Not Initiator}
		\State $<$R , Parent, CID, True $>$
	\EndIf
	\State Increment RCC 
	\State Rerecover = True
	\State CID = msg.CID
	\State Parent = msg.sender
\ElsIf{state = Phantomized}
	\State Increment RCC
	\State state = Recovering
	\State CID = msg.CID
  \If {SC $>$ 0}
	\State $<$B , $\Gamma_{out}$, CID $>$
  \Else
	\State $<$RC , $\Gamma_{out}$, CID $>$
  \EndIf
	\State Parent = msg.sender
\ElsIf{state = Building}
	\If{msg.CID $<$ CID}
		\State Rephantomize = true
		\State Rerecover = true
		\State $<$R , msg.sender $>$
	\ElsIf {msg.CID = CID}
		\State $<$R , msg.sender $>$
	\Else
		\If{Not Initiator}
			\State $<$R , Parent, CID, True $>$
		\EndIf
		\State CID = msg.CID
		\State Rerecover = true
		\State Parent = msg.sender
		\State Rephantomize = true
	\EndIf
\ElsIf{state = Recovering}
	\If{msg.CID $<$ CID}
		\State $<$R , msg.sender $>$
	\ElsIf{msg.CID = CID}
		\State Increment RCC
		\State $<$R , msg.sender $>$
		\If{Not Initiator \& RCC = PC \&  \\ \hspace{1.0in} All return msgs received}
			\State $<$R , Parent $>$
		\EndIf
	\Else
		\If{Not Initiator}
			\State $<$R , Parent, CID, True $>$
		\EndIf
		\State CID = msg.CID
		\State RCC = 1
		\State Parent = msg.sender
		\If {Waiting for return msgs}
			\State Rerecover = True	
		\Else 
			\State $<$RC , $\Gamma_{out}$, CID $>$
		\EndIf
	\EndIf
\ElsIf {state = Recovered}
	\State RCC = 1;
	\State CID = msg.CID
	\State state = Recovering
	\State Parent = msg.sender
	\State $<$RC,  $\Gamma_{out}$, CID, $>$
\EndIf
\EndProcedure
\end{algorithmic}
%\end{multicols}
\end{algorithm}	

\subsection{Appendix of proofs}
\label{proofapp}
\begin{replemma}{lem:nophan}
After phantomization, nodes in the build set are not phantom nodes.
%\label{Not Phantomized}
\end{replemma}
\begin{proof}
By definition, a strong path from $R$ to each node in the build set
exists, therefore phantomization spreading from the initiator cannot take
away the last strong edge, and so the build set will not phantomize.
\end{proof}

\begin{replemma}{lem:buildset}
After phantomization, if the initiator has strong incoming edges, then the
initiator is not dead.
%\label{lem:buildset}
%nodes that are the source of those
%strong edges form the build set.
\end{replemma}
\begin{proof}
Only nodes in the supporting set can keep the initiator, $I$, alive. % since
%$\Gamma_{\rm in}(I)$ is a subset of the supporting set.
When the initiator became a phantom node, it converted its weak incoming edges to strong.
However, since nodes in the recovery set all became phantom nodes by Lemma~\ref{lem:phant},
they will not contribute strong support to the initiator.
Since nodes in the build set do not phantomize by Lemma~\ref{lem:nophan}, the edges connecting them to the initiator
remain strong.
\end{proof}

\begin{replemma}{lem:phant}
After phantomization, nodes in the recovery set are phantom nodes.
%\label{lem:phant}
\end{replemma}
\begin{proof}
By definition, no strong path from $R$ to the nodes in the recovery set
exists. The strong edges they do have come from the initiator, and these will
phantomize after the initiator phantomizes. Once that happens, the recovery
set will phantomize.
\end{proof}

\begin{replemma}{lema:recoveryset}
After phantomization,
if a phantom node contains at least one strong incoming edge,
it belongs to the recovery set.
%\label{lem:recoveryset}
\end{replemma}
\begin{proof}

Every node in the recovery set will phantomize by Lemma~\ref{lem:phant}, and every node
in the recovery set will have a non-strong path from $R$ prior to
phantomization. Phantomization, however, will cause the nodes to toggle,
converting the non-strong path from $R$ to a strong path for at least one
node in the recovery set. Therefore, the
recovery set will contain at least one node with at least one strong incoming edge.

\end{proof}

\begin{replemma}{TimeC}
SCA finishes in O(Rad(i,$G_{a}$)) time, where $G_{a}$ is the graph
induced by affected nodes, i is the initiator, and Rad(i, $G_{a}$) is radius
of the graph from i.
%\label{TimeC}
\end{replemma}
\begin{proof}
In each time step, \emph{phantomization} spreads to the $\outneighbors$ of the previously
affected nodes, increasing the radius of the graph of phantom nodes by 1.
In O(Rad(i,$G_{a}$)) time, all the nodes in $G_{a}$ receive
a \emph{phantomize} message, since all the nodes in $G_{a}$ are at distance less than
or equal to Rad(i,$G_{a}$) from i. In the reverse step, the same argument can be
applied backward.
%In r time, the $( Rad(i,G_{a}) - r) ^{th}$ neighborhood of i
%receives the \emph{return} messages.
So phantomization completes in O(Rad(i,$G_{a}$) time.

In the correction phase, during the forward step, in r time, r neighborhoods of i
received \emph{recovery} or \emph{build} or \emph{plague delete} message, until
the affected subgraph is traversed.
%In the forward step
%of the algorithm, the message recovery or build has to received by all the
%nodes in the affected subgraph.
%So in O(Rad(i, $G_{a}$)) round, all nodes receive
%forward step message of the recovery step.
%Unlike other process,
%the reverse step of the recovery step is very complicated.
In the reverse step of \emph{build} or \emph{recovery}, however,
a \emph{return} message might initiate the build process.
%The build
%process will send build messages to nodes in the subgraph reachable from the build
%process initiator.
While the build process nodes send return messages, the nodes
will become healthy thereby reducing the Rad(i,$G_{a}$).
So in the worst case, all nodes that received recovery might build. Because
each node will have only one parent, the return
step cannot take more than O(Rad(i, $G_{a}$) time.
\end{proof}

\begin{replemma}{CCC}
SCA sends O($E_{a}$) messages, where $E_{a}$ is the number of edges
in the graph induced by affected nodes.
%\label{CCC}
\end{replemma}
\begin{proof}
%Appendix ~\ref{proofapp} contains the proof.

In the forward step of the phantomization, all the nodes in the dependent set send the
\emph{phantomize} message to their $\outneighbors$, and each node can do this at most
once (after which the phantom node flag is set).
So the forward step of the algorithm uses only
O($E_{a}$) messages. In the reverse step, the \emph{return} messages are sent
backward along the edges of the spanning tree. So the reverse step sends O($V_{a}$)
messages, where $V_{a}$ is the number of nodes in the affected subgraph.
So Phantomization uses O($E_{a}$) messages.

In the forward step of the recovery/building, either a {\em recovery} message, a {\em build} message,
or both
traverses every edge, so the forward step of the algorithm uses O($E_{a}$) messages.
In the
reverse step of the algorithm, the \emph{return} messages are sent back to the parent a maximum
of two times (once for recovery, once for build), traversing a subset of the edges in the
reverse direction. Thus, there is a maximum of four message traversals for any edge.
%A build process initiated by any node will send build message in the same edges that
%sent return earlier. But once the edges sent build return message, the nodes will never
%be part of the affected subgraph. So they never receive a message. So all edges send
%constant number of recovery, build and return messages.
In every forward step of the plague delete, all outgoing edges are consumed, and therefore it 
takes O($E_a$) messages.
%a radius of 1. Therefore, in O(Rad(i,$G_{a}$)) time it will spread to the entire subgraph.
\end{proof}

\begin{replemma}{MSCP}
SCA sends messages of O(log(n)) size, where n is the total number of nodes
in the graph.
%\label{MSCP}
\end{replemma}
\begin{proof}
The messages have to hold a value at least as large as the count of nodes in the system
which are O(log(n)) size. Apart from the ids, the message also contains the weight of
node which is constant in size. In the return message, our algorithm only uses the id
of the sender and receiver. So all the messages in the SCA are of
O(log(n)) size.
\end{proof}

\begin{reptheorem}{pro:livenesss}
All dead nodes will be deleted at the end of the correction phases.
%\label{pro:livenesss}
\end{reptheorem}
\begin{proof}
%A node can be dead node only by losing all the strong 
%incoming edges. If a node lost all strong incoming edges, by the 
%rules of phantomization phases, the node will become initiator and
%detect if it is garbage node. Initiator also detects all the purely
%dependent nodes in the phantomization phase. So when an initiator decides the node is dead, it deletes all purely dependent nodes
%too.
Assume a node $y$, is a dead node, but flagged as live
node by the correction phase. If $y$ becomes live,
then it must have done so because its edges were rebuilt during building.
If so, then either $y$ has a strong count, or there exists a node $x$ with a strong count
 which also has a phantom path to $y$. However, any node with a strong count at the end
 of phantomization is a live node by Lemma~\ref{lem:buildset} and Lemma~\ref{lema:recoveryset}, and because a path from $x$ to $y$ exists, $x$ is also live.
 %Node $x$ must be part of the supporting set to have path from R to it. If the initiator is dead node and flagged as live node at the 
 %end of the correction phases, then $x$ must not in supporting set. 
 %From figure \ref{fig:completeabstract}, it is clear then $x$ must
 %belong to $B'$. No nodes in the purely dependent set has non-zero
 %strong count at the end of the phantomization phases. So the subgraph will never execute /$Build /$ phase. Thus for initiator to
 %be live node, $x$ must belong to supporting.
 This contradicts our
 assumption.
\end{proof}


\begin{reptheorem}{pro:safetys}
No live node will be deleted at the end of the correction phases.
%\label{pro:safetys}
\end{reptheorem}

\begin{proof}
Assume a node, $x$ is live, but it is deleted. If $x$ is the initiator,
and is live then the supporting set is not empty. If the build set isn't
empty, then $x$ will have a strong count, so it won't be deleted. If
the build set is empty and the recovery set is non-empty, then correction
will build a strong edge to $x$, so it won't be deleted. This contradicts
our assumption. Now assume $x$ is dead, but it causes some other node to
be deleted. If $x$ is dead, then the supporting set is empty and the purely
dependent set is dead. The independent set has a strong count before the
collection process, so it won't delete. The nodes in the dependent set
that had a non-strong path from $R$ before phantomization, will have a strong
path from $R$ after toggling and recovery/build, so they will not delete. This
also contradicts our assumption.
%ermination detection algorithm embedded in our algorithm ensures
%hat the initiator does not make any decision until the phases are 
%omplete in all the affected subgraph. There are two cases where 
%nitiator is live node and it is deleted and a node in the affected
%ubgraph that is live and has no path to initiator exist. For the
%irst case, after the end of the phantomization, for the initiator
%o be live, there has to at least a node in the supporting set . If
%here is a build set available, then the initiator will not delete 
%ny nodes in the affected subgraph. If build set is empty and recovery set is not empty, then the recovery phase ensures that 
%he initiator incoming edge(s) are converted into strong. So at the
%nd of the recovery phase, the initiator will not delete any nodes in the affected subgraph. In the second case, where the initiator is dead, but not all the nodes in the affected subgraph is dead, at
%he end of the recovery phase, the initiator will not have any strong incoming edges. So the initatior starts the plague delete. But if there exist a subgraph that contains live node, then the subgraph of nodes does not reach the initiator back. So, according 
%o \ref{fig:completeabstract}, it must be $C$, $D$, $E$. As $D$ has
%trong path from R before phantomization and the Adversary stops 
%utating graph after the collection process, the strong path from R
%emains through out the process and the plague delete will not delete the node. Nodes in $C$ and $E$ will convert the non-strong
%ath from R into strong path at the end of phantomization. So during the recovery phase, the nodes build (converting phantom edges into strong or weak) the reachable affected 
%ubgraph. So when a plague delete message reaches the node that are
%ive inside the affected subgraph, the strong count of the nodes never drop to zero and never will be deleted at the end of correction process.
\end{proof}



\begin{replemma}{lem:acyclic}
The collection process graph is eventually acyclic.
%\label{lem:acyclic}
\end{replemma}
\begin{proof}
If a cycle exists between two process graphs, A and B, then recovery or
build messages must eventually propagate from one to the other. Without
loss of generality, assume A has higher priority than B. During the
correction phases, messages propagate along all phantom edges, and
in the process take ownership of any nodes they touch. Eventually, therefore,
there should be no edges from A to B.
\end{proof}

\begin{replemma}{lem:ordered}
If an entity $A$ precedes an entity $B$ topologically in the collection process graph, and
$A$ has a lower priority than $B$, entity $A$ will complete before entity $B$ and both will
complete in isolation.
%\label{lem:ordered}
\end{replemma}
\begin{proof}
Consider a node, $x$, that has incoming edges from both $A$ and $B$. Process $B$ will
have ownership of the node during the recovery or build phase, but until the
edges from $A$ are either built or deleted, $x$ will have an $RCC$ equal to the
number of edges from $B$, and $PC$ equal to the sum of the edges from $A + B$.
So the recovery or build phase of $B$ must take place after $A$, and so
$B$ will operate in isolation. Since there are no edges from $B$ to $A$, and since $B$ is not
making progress, $B$ does not affect $A$, and $A$ is in isolation.
\end{proof}

\begin{replemma}{deletionI}
Edge deletion by the Adversary does not violate isolation.
%\label{deletionI}
\end{replemma}
\begin{proof}
%Lemma: When an Adversary deletes an edge from a graph, it will not violate isolation.
%\begin{itemize}
If we simply remove an edge $x\rightarrow y$ from inside a collection process
graph, that might violate isolation because it removes $y$ from the graph.
However, we have already developed a method to remove a node from the graph
without violating isolation, and that is to give the node to a higher
collection process id. So when the edge is deleted, a new, higher collection
process is created and it is given ownership of $y$. By
Theorem~\ref{thm:alliso}, the old and new collection process will proceed in
isolation.
%
%  \item Consider that $y$ is recovered and there is no phantom path from $y$ to $x$
%    There can be no effect because no further messages can reach $x$ from $y$ (i.e. it has received recovery messages along $\Gamma_{\rm in}(x)$, and has replied with return), and $y$ is expecting no further messages from $x$ (since $x$ has already sent return). Therefore, collection $x_1$ proceeds in isolation.
%  \item Consider that $x$ is recovered and there is a phantom path from $y$ to $x$
%    In this case, collection process $x_2$ will take ownership of $x_1$, and any decision it makes to build will not matter. It is not possible that $x_1$ will delete, because, because at the time of the deletion $y$ was live. If it were not, the delete would not be possible because we cannot mutate a live graph.
%  \item Consider that $y$ is phantomized and not recovered. In this case, the recovery will not proceed to $y$, and the recovery will proceed as if the paths originating on $y$ never existed. The eventual recovery or build will, therefore, proceed in isolation.
%  \end{itemize}
\end{proof}




\begin{comment}
\end{document}
\end{comment}