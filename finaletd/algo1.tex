Note that in the following sections, locks are always obtained in canonical order that avoids deadlocks. Unlock methods unlock the last set of items that were locked.

The lists on the collectors are thread-safe.

The collector object is itself managed by a simple reference count. Code for incrementing and decrementing this count is not explicitly present in the code below.

A reference implementation is also provided ~\cite{url:refimpl}.

\begin{algorithm}[H]
		\scriptsize
			\caption{LinkSet creates a new link,
				and decides the type of the link to create.}
			\label{single:algorithm:linkset}
			
%				\begin{multicols}{2}
					\begin{algorithmic}[1]
		
%{\small
%\tcp{}
\Procedure{ LinkSet(link,node)}{}
\State lock (link.Source,node)
\If {node $==$ NULL}
\State LinkFree(link)
\State link.Target = NULL
\EndIf
\If {link.Target $==$ node }
\State Return
\EndIf
\State oldLink = copy(link) 
\State link.Target = node
\If {link.Source.Phantomized }
\State MergeCollectors(link.Source, link.Target)
\State link.PhantomCount++
\State link.Phantomized = True
\ElsIf {link.Source == node }
\State link.Which = 1 - node.Which
\State node.Count[link.Which]++
\ElsIf { node.Links not initialized }
\State node.Count[link.Which]++
\State link.Which = node.Which
\Else
\State link.which = 1 - node.Which
\State node.Count[link.Which]++
\EndIf
\If { oldLink != NULL}
\State LinkFree(oldLink)
\EndIf
\State unlock()
\EndProcedure
%}
%\caption{LinkSet}
%\label{}
\end{algorithmic}
%\end{multicols}
\end{algorithm}
\setlength{\textfloatsep}{0pt}

\begin{algorithm}[H]
	\scriptsize

%	\begin{multicols}{2}
		\begin{algorithmic}[1]

\Procedure{ LinkFree(link)}{}

% \tcp*[f]{Create a pointer from object $R$ to some free object}

\State lock(link.Source,link.Target)
\If { link.Target == NULL }
\State Return
\EndIf
\If { link.Phantomized }
\State DecPhantom(link.Target)
\Else
\State link.Target.Count[link.Which]-\,-
\If { link.Target.Count[link.Which] == 0 {\bf And} 
 link.Target.Which == link.Which }
\If { link.Target.Count[1-link.Target.Which] == 0 {\bf And} 
 link.Target.PhantomCount == 0 }
\State Delete(node)
\State link.Target = NULL
\Else
\If { link.Target.Collector == NULL }
\State link.Target.Collector = new Collector()
\EndIf
\State AddToCollector(link.Target)
\EndIf
\EndIf
\EndIf
\State unlock()

\EndProcedure

\caption{LinkFree}
\label{algorithm:linkfree}
\end{algorithmic}
%\end{multicols}
\end{algorithm}


\setlength{\textfloatsep}{0pt}
\begin{algorithm}[H]
		\scriptsize
		
%		\begin{multicols}{2}
			\begin{algorithmic}[1]
				
%{\small
%\begin{multicols}{2}
% \tcp{$R$ and $S$ are objects currently in use.}
% \tcp*[f]{Create a pointer from object $R$ to some free object}


%\tcp{Adding an object to the collector puts back the strong
%count, effectively transferring the source of the strong link
%to the collector. It also adds a phantom count, which helps
%prevent the clearing of the Collector field.}

\Procedure {AddToCollector(node)}{}
%{\Indp
\While { True }
\State lock(node,node.Collector)
\If {node.Collector.Forward != NULL }
\State node.Collector = node.Collector.Forward
\Else
\State node.Count[node.Which]++
\State node.PhantomCount++
\State node.Collector.CollectionList.append(node)
\State Break
\EndIf
\State{unlock()}
\EndWhile
%}
\EndProcedure
%}
\caption{AddToCollector}
\label{single:algorithm:addtocollector}
\end{algorithmic}
%\end{multicols}
\end{algorithm}
%\end{multicols}


\setlength{\textfloatsep}{0pt}
\begin{algorithm}[H]
	\scriptsize
	
%	\begin{multicols}{2}
		\begin{algorithmic}[1]
%{\small
%\begin{multicols}{2}
%\tcp{The collector takes away the strong link it made in
%AddToCollector().}
\Procedure { PhantomizeNode(node,collector)}{}
%{\Indp
\State {lock(node)}
\While { collector.Forward != NULL }
\State collector = collector.Forward
\EndWhile
\State node.Collector = collector
\State node.Count[node.Which]-\,-
%\State// Prevent deletion while the
%\State// node is managed by the Collector
\State phantomize = False
\If { node.Count[node.Which] $>$ 0 }
\State {\bf Return}
\Else
\If { node.Count[1-node.Which] $>$ 0 }
\State node.Which = 1-node.Which
\EndIf
\If { {\bf Not} node.Phantomized }
\State node.Phantomized = True
\State node.PhantomizationComplete = False
\State phantomize = True
\EndIf
\EndIf
\State links = NULL 
\If { phantomize }
\State links = copy(node.Links)
\EndIf
\State {unlock()}
\For { each outgoing link in links }
\State PhantomizeLink(link)
\EndFor
\State lock(node)
\State {node.PhantomizationComplete = True}
\State unlock()
%}
\EndProcedure
%\end{multicols}
%}
\caption{PhantomizeNode}
\label{algorithm:phantomizenode}
\end{algorithmic}
%\end{multicols}
\end{algorithm}



\setlength{\textfloatsep}{0pt}
\begin{algorithm}[H]
	\scriptsize
	
%	\begin{multicols}{2}
		\begin{algorithmic}[1]
%{\small
%\begin{multicols}{2}
%\tcp{This method describes the
%work to be carried out by a
%garbage collection thread. Live
%objects pointing to this collector, or
%Forward pointers from other collectors
%contribute to the RefCount field on
%the Collector.}
\Procedure { Collector.Main()}{}
%{\Indp
\While { True }
\State WaitFor(Collector.RefCount == 0 {\bf Or} Work to do)
\If { Collector.RefCount == 0 {\bf And} No work to do }
\State {\bf Break}
\EndIf
\While { Collector.MergedList.size() $>$ 0 }
\State node = Collector.MergedList.pop()
\State Collector.RecoveryList.append(node)
\EndWhile
\While { Collector.CollectionList.size() $>$ 0 }
\State node = Collector.CollectionList.pop()
\State PhantomizeNode(node,Collector)
\State Collector.RecoveryList.append(node)
\EndWhile
\While { Collector.RecoveryList.size() $>$ 0 }
\State node = Collector.RecoveryList.pop()
\State RecoverNode(node)
\State Collector.CleanList.append(node)
\EndWhile
\While {Collector.RebuildList.size() $>$ 0 }
\State node = Collector.RebuildList.pop()
\State RecoverNode(node)
\EndWhile
\While {Collector.CleanList.size() $>$ 0 }
\State node = Collector.CleanList.pop()
\State CleanNode(node)
\EndWhile
\EndWhile
%}
\EndProcedure
%}
\caption{Collector.Main}
\label{algorithm:main}
\end{algorithmic}
%\end{multicols}
\end{algorithm}
\setlength{\textfloatsep}{0pt}


\begin{algorithm}[H]
	\scriptsize
	
%	\begin{multicols}{2}
		\begin{algorithmic}[1]
%{\small
%\begin{multicols}{2}
\Procedure {PhantomizeLink(link)}{}
%{\Indp
\State {lock(link.Source,link.Target)}
\If { link.Target == NULL }
\State {unlock()}
\State {\bf Return}
\EndIf
\If { link.Phantomized }
\State {unlock()}
\State{\bf Return} 
\EndIf
\State link.Target.PhantomCount++
\State link.Phantomized = True
\State linkFree(link)
\State MergeCollectors(link.Source, link.Target)
\State  {unlock()}
%}
\EndProcedure
%}
\caption{PhantomizeLink}
\label{single:algorithm:phantomizelink}
\end{algorithmic}
%\end{multicols}
\end{algorithm}



\setlength{\textfloatsep}{0pt}
\begin{algorithm}[H]
	\scriptsize
	
%	\begin{multicols}{2}
		\begin{algorithmic}[1]
%{\small
%\begin{multicols}{2}
%\tcp{DecPhantom is responsible for removing any reference to
%the collector.}
\Procedure { DecPhantom(node)}{}
%{\Indp
\State  {lock(node)}
\State node.PhantomCount- -
\If { node.PhantomCount == 0 }
\If { node.Count[node.Which]== 0 {\bf And}
 node.Count[1-node.Which] == 0 }
\State Delete(node)
\Else
\State node.Collector = NULL
\EndIf
\EndIf
\State {unlock()}
%}
\EndProcedure
%}
\caption{DecPhantom}
\label{single:algorithm:decPhantom}
\end{algorithmic}
%\end{multicols}
\end{algorithm}
\setlength{\textfloatsep}{0pt}



\begin{algorithm}[H]
	\scriptsize
	
%	\begin{multicols}{2}
		\begin{algorithmic}[1]
%{\small
%\begin{multicols}{2}
\Procedure {RecoverNode(node)}{}
%{\Indp
\State  {lock(node)}
\State links = NULL
\If { node.Count[node.Which] $>$ 0 }
\State WaitFor(node.PhantomizationComplete == True)
\State node.Phantomized = False
\State links = copy(node.Links)
\EndIf
\State {unlock()}
\For {each  link in links }
\State Rebuild(link)
\EndFor
\EndProcedure
%}
%}
\caption{RecoverNode}
\label{algorithm:recover}
\end{algorithmic}
%\end{multicols}
\end{algorithm}
\setlength{\textfloatsep}{0pt}



\begin{algorithm}[H]
	\scriptsize
	
%	\begin{multicols}{2}
		\begin{algorithmic}[1]
%{\small
%\begin{multicols}{2}
\Procedure { Rebuild(link)}{}
%{\Indp
\State {lock(link.Source,link.Target)}
\If {link.Phantomized }
\If { link.Target == link.Source }
\State link.Which = 1- link.Target.Which
\ElsIf { link.Target.Phantomized } 
\State link.Which = link.Target.Which
\ElsIf { count(link.Target.Links) == 0 }
\State link.Which = link.Target.Which
\Else
\State link.Which = 1-link.Target.Which
\EndIf
\State link.Target.Count[link.Which]++
\State link.Target.PhantomCount- -
\If { link.Target.PhantomCount == 0 }
\State link.Target.Collector = NULL
\EndIf
\State link.Phantomized = False 
\State Add link.Target to Collector.RecoveryList
\EndIf
\State {unlock()}
 \EndProcedure
%}
%}
\caption{Rebuild}
\label{single:algorithm:rebuild}

\end{algorithmic}
%\end{multicols}
\end{algorithm}
\setlength{\textfloatsep}{0pt}


\begin{algorithm}[H]
	\scriptsize
	
%	\begin{multicols}{2}
		\begin{algorithmic}[1]
%{\small
%\begin{multicols}{2}
%\tcp{After deleting all the outgoing links, decrement
%the phantom count by one (i.e. the reference held by
%the collector itself). When the last phantom count
%is gone, the object is cleaned up.}
\Procedure {  CleanNode(node)}{}
%{\Indp
\State {lock(node)}
\State die = False
\If { node.Count[node.Which]== 0 {\bf And}
 node.Count[1-node.Which]== 0 }
\State die = True
\EndIf
\State {unlock()}
\If { die }
\For {each link in node }
\State LinkFree(link)
\EndFor
\EndIf
\State DecPhantom(node)
\EndProcedure
% }
%}
\caption{CleanNode}
\label{algorithm:cleanup}
\end{algorithmic}
%\end{multicols}
\end{algorithm}
\setlength{\textfloatsep}{0pt}


\begin{algorithm}[H]
	\scriptsize
	
%	\begin{multicols}{2}
		\begin{algorithmic}[1]
%{\small
\Procedure {  Delete(node)}{}
%{\Indp
\For {each link in node }
\State LinkFree(link)
\EndFor
\State freeMem(node)

 %}
 \EndProcedure 
%}
\caption{Delete}
\label{algorithm:Delete}
\end{algorithmic}
%\end{multicols}
\end{algorithm}
\setlength{\textfloatsep}{0pt}


\begin{algorithm}[H]
	\scriptsize
	
%	\begin{multicols}{2}
		\begin{algorithmic}[1]
			%{\small
%{\small
%\begin{multicols}{2}
%\tcp{When two collector threads
%realize they are managing a common
%subset of objects, one defers to
%the other. The arguments, source and
%target, are both nodes.}
\Procedure {MergeCollectors(source,target)}{}
%{\Indp
\State s = source.Collector
\State t = target.Collector
\State  done = False
\If { s == NULL {\bf And} t != NULL }
\State lock(source)
\State source.Collector = t
\State unlock()
\State {\bf Return}
\EndIf
\If { s != NULL {\bf And} t == NULL }
\State lock(target)
\State target.Collector = s
\State unlock()
\State {\bf Return}
\EndIf
\If { s == NULL {\bf Or} s == NULL }
\State {\bf Return}
\EndIf
\While {Not done }
\State {lock(s,t,target,source)}
\If { s.Forward == t and t.Forward == NULL }
\State target.Collector = s
\State source.Collector = s
\State done = True
\ElsIf { t.Forward == s and s.Forward == NULL }
\State target.Collector = t
\State source.Collector = t
\State done = True
\ElsIf { t.Forward != NULL }
\State t = t.Forward 
\ElsIf { s.Forward != NULL }
\State s = s.Forward 
\Else
\State Transfer s.CollectionList to t.CollectionList 
\State Transfer s.MergedList to t.MergedList 
\State Transfer s.RecoveryList to t.MergedList 
\State Transfer s.RebuildList to t.RebuildList 
\State Transfer s.CleanList to t.MergedList 
\State target.Collector = t
\State source.Collector = t
\State done = True
\EndIf
\State unlock()
\EndWhile
\EndProcedure
%}
%}
\caption{MergeCollectors}
\label{algorithm:mergecollectors}
\end{algorithmic}
%\end{multicols}
\end{algorithm}
