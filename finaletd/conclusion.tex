\section{Conclusions}
\label{section:conclusions}

We have described a hybrid reference counting algorithm for garbage collection
in distributed systems. The algorithm improves on the previous
efforts by eliminating the need for centralization, 
global barriers, back references for every edge, and object migration. Moreover, it achieves efficient time and
space complexity. %; the previous algorithm provide no or inefficient
%time and space guarantees.
Furthermore, the algorithm is stable against
concurrent mutations in the reference graph. %(i.e., the reference graph can
%change during the algorithm's execution).
The main idea was to develop a technique
to identify the supporting set (nodes that prevent a given node from being
dead) and handle the synchronization of multiple collection processes.
We believe that our techniques
might be of independent interest in solving other fundamental problems that
arise in distributed computing. 

In future work, we hope to address how to provide an order in which the dead
nodes will be cleaned up, permitting some kind of ``destructor'' to be run, and
to address fault tolerance. In addition, we hope to implement the proposed
algorithm and compare its performance with previous algorithms using different
benchmarks.
