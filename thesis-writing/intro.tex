\section{Introduction}
Garbage collection (GC) is very useful for software development. It helps the developers to avoid explicit deallocation and also avoid dangling pointers. There has been constant complaint that the GC reduces the performance of the application. But the recent research shows that there is in fact much benefit in using it. With the advent of multicore, the processor utilization is of the high concern and application post multi-core era has been focusing on building concurrent applications. Garbage Collections are designed to be concurrent and parallel. The distinction between concurrent and parallel is if the application runs simultaneously with the garbage collection it is termed as concurrent collector while there are more than one collector thread runs during the collection, it is called as parallel collector. There has been lot of focus on concurrent collectors lately. 

In HPC community, applications are mostly written in languages with no or less managed runtime systems. It is just time that all applications that are written for the huge clusters to change to managed runtime systems. The HPC applications can benefit all the advantages of the garbage collection. Predominantly HPC codes are written for the two kind of the memory architectures. The first being the shared memory systems, where the memory is common for all the processing element. Other being distributed memory system where each processing element has its own share of memory. Bacon et al described the theory of garbage collection in very lucid way. The proposed hypothesis is based on the Bacon et al observation on the different kinds of garbage collector. The two main GC techniques are mark-sweep (MS) and reference counting (RC). All the available GC algorithm follows one of these approach to identify garbage. In the abstract way, MS traces all the live heap objects and then deletes the untraced but allocated heap objects. On the other hand, RC traces all the dead/potential dead objects and then deletes the traced objects. 

The hypothesis of this research is from the above abstract idea. If MS collector can determine the garbage by traversing the live heap objects, then in a huge memory systems with more live objects the collector should perform worse than RC collectors. In reality, MS is most preferred GC than RC. It is constantly proved that RC performs worse than MS. RC implementations have write barriers which require additional instruction to update the reference count of the heap objects on every change in the object graph.
In a highly optimized GC system with huge memory, RC based systems will perform better than MS. The research steer towards RC based garbage collection.
