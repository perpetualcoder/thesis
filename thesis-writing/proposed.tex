\section{Future Work}

\iffalse
Local creation of links only allows the creation of strong references when no
cycle creation is possible. Consider the creation of a link from a source object $S$
to a target object $T$. The link will be created strong if (i) the only strong
links to $S$ are from roots i.e. there is no object $C$ with a strong link to $S$;
(ii) object $T$ has no outgoing links i.e. it is newly created and its outgoing links are
not initialized; and (iii) object $T$ is phantomized, and $S$ is not. All
self-references are weak. Any other link is created phantom or weak.
\fi


In HPC community, applications are written for both the shared memory and 
distributed memory systems. The preliminary work uses the centralized queue based 
solution. This work can be applied to shared memory systems and distributed memory 
systems. Although the solution is applicable to both of the systems, the distributed memory 
would benefit more from completely decentralized solution. So the work is focused more 
towards the decentralized/localized solution to the distributed garabge collection.


% \pagebreak
% \doublespacing


Fundamental thesis of the work is any tracing based algorithm traces the entire memory. 
But hybrid approaches like our algorithm only traces the affected subgraph. So in practical 
use, the graph traversed to collect garbage is just infinitesimal of the total graph. 
So the throughput of the garbage collector should be better in case of our approach. 
We want to extend the concept of strong and weak to the distributed garbage collection. 
Various approaches up until now available for the distributed garbage collectors are not complete.
They are not practical solution for various reason. Our proposed work is to build a reliable, 
fault-tolerant, linear garbage collector in the distributed memory system. Apart from 
the designing the algorithm, we would also like to evaluate the performance of the the 
previously described algorithm in a virtual machine and also  try to implement 
the solution to be discovered in a scientific framework to compare with the other algorithms.
