\section{Literature Review}

McCarthy invented the concept of Garbage Collection in 1959~\cite{mccarthy}. His works focused on indirect method of collecting 
garbage which is called Mark Sweep or tracing collector.Collins designed a new method called reference counting, which is more direct method to detect garbage~\cite{Collins1960}. 
At the time when garbage collectors are designed and several of them cannot be used in practice due to bugs in algorithm, Dijkstra et al~\cite{dijkstra} modeled a tricolor tracing garbage collector that provided correctness proofs. This work redefined the 
 way the collectors should be designed and how one can prove the properties of the collectors. Haddon et al ~\cite{haddon} realized 
 the fragmentation issues in the Mark Sweep algorithm and designed a Mark Compact algorithm where objects are moved when traced and arranged in order to accumulate all the free space together. Fenichel et al~\cite{feni} and Cheney~\cite{cheney}
 provided new kind of algorithm called Copying collector based on idea of live objects. Copying collector divides the memory into two equal halves and collector moves live object from one section to the other section and cleans the entire half of the section. This method uses only half of the heap at any given time and requires copying the object which is very expensive. Based on Foderaro and Fateman ~\cite{fode81} observation,  98\% of the objects collection has been allocated after the last collection. Several such observations were made and reported in ~\cite{zorn89,sans93}. Based on the above observations, Appel  designed a simple generation garbage collection~\cite{Appel89}. Generational garbage collection utilizes idea from the all classical method mentioned above and provides a good throughput collector. It divides the memory into at least two generation. Old generation occupies large portion of the heap and young generation takes less portion of heap. It exploits the weak generational hypothesis which says the young objects die soon. In spite of considerable overhead due to inter-generational pointers, promotions, and full heap collection once in a while, this garbage collection technique dominates alternatives in terms of performance and widely used in many run-time systems.
 
 While most of the method up until now considered are  indirect method of collecting garbage through tracing, the other form of collection is counting the references. McBeth's~\cite{McBeth1963} article on the inability to collect cyclic garbage  by 
 reference counting method made the method less practical. Bobrow's ~\cite{Bobrow1980} solution to collect cyclic garbage based on reference counting method was the first hybrid collector which uses both tracing and
counting the references. Hughes ~\cite{hugh83,hugh87} identified the practical issues in the Bobrow's algorithm.
several attempts to fix this problem appeared subsequently, e.g. \cite{Friedman1979,Bobrow1980,Lins2008}. 
In contrast to the approach followed in \cite{Friedman1979,Bobrow1980,Lins2008} and several others,
Brownbridge \cite{Brownbridge1985} proposed, in 1985, a strong/weak pointer algorithm to tackle the problem of reclaiming cyclic  structures using strong and weak pointers \cite{Jones1996}. 
This algorithm relied on maintaining two invariants: (a) there are no cycles in strong pointers and (b) all items in the Object Reference graph must be strongly reachable from the roots.

Couple of years after the publication, Salkild \cite{Salkild1987} showed that Brownbridge's algorithm
\cite{Brownbridge1985} could reclaim objects prematurely in some
configurations, e.g. a double cycle. If the last strong pointer (or link) to
an object in one cycle but not the other was lost, Brownbridge's
method would incorrectly claim nodes from the cycle.
Salkild \cite{Salkild1987} corrected this problem by proposing
that if the last strong link was removed from an object which still
had weak pointers, a collection process should re-start from that node.
While this approach eliminated the premature deletion problem, it introduced a
potential non-termination problem.

Subsequently, Pepels et al.~\cite{Pepels1988} proposed a new algorithm based on
Brownbridge-Salkild's algorithm and solved the problem of non-termination by
using a marking scheme. In their algorithm, they used two kinds of mark: one to
prevent an infinite number of searches, and the other to guarantee termination
of each search. Although correct and terminating, Pepels et al.'s algorithm is far more
complex than Brownbridge-Salkild's algorithm and in some cyclic structures the
cleanup cost complexity becomes at least
exponential in the worst-case \cite{Jones1996}. This is due to the fact that when
cycles occur, whole state space searches from
each node in the cyclic graph must be initiated, possibly many times. After Pepels et al.'s algorithm, we are not aware
of any other work on reducing the cleanup cost or complexity of the Brownbridge
algorithm. Moreover, there is no concurrent collection technique using this approach which can be applicable for the garbage collection in modern multiprocessors.

Typical hybrid reference count collection systems, e.g. \cite{Bacon2001,Levanoni2006,Bacon:2001:JWC,Barabash2005,Lins2008}, which use a reference counting collector combined with a tracing collector or cycle collector,
must perform nontrivial work whenever a reference count is decreased and does
not reach zero. Trial deletion approach was studied by Christopher \cite{Christopher1984} which tries to collect cycles by identifying groups of self-sustaining objects. 
Lins \cite{Lins:1992:CRC} used a cyclic buffer to reduce repeated scanning of the same nodes in their Mark-Scan algorithm for cyclic reference counting. Moreover, in \cite{Lins:2002:EAC}, Lins improved his algorithm  from \cite{Lins:1992:CRC} by eliminating the scan operation through the use of a Jump-stack data structure.

With the advancement of multiprocessor architectures,
reference counting garbage collectors have become popular because
they do not require all application threads to be stopped before the garbage collection algorithm can run~\cite{Levanoni2006}.
Recent work in reference counting algorithms, e.g. \cite{Barabash2005,Levanoni2006,Bacon2001,Bacon:2001:JWC}, try to
reduce concurrent operations and increase the efficiency of reference counting collectors.
However, as mentioned earlier, reference counting garbage collectors cannot collect cycles \cite{McBeth1963}. Therefore, concurrent reference counting collectors \cite{Barabash2005,Levanoni2006,Bacon2001,Bacon:2001:JWC,Paz2007,Lins2008} use other techniques, e.g. they supplement the reference counter with a tracing collector or a cycle detector, together with their concurrent reference counting algorithm. For example, the reference counting collector proposed in \cite{Paz2007} combines the sliding view reference counting concurrent collector of \cite{Levanoni2006} with the cycle collector of \cite{Bacon2001}. Recently, Frampton provides a detailed study of cycle collection in his PhD thesis \cite{Frampton2010}.


Apple's ARC memory management system makes a distinction between ``strong'' and ``weak'' pointers, similar to what Brownbridge describe. In the ARC memory system, however, the type of each pointer must be specifically designated by the programmer, and this type will not change during the program's execution. If the programmer gets the type wrong, it is possible for ARC to have strong cycles as well as prematurely deleted objects. With our system, the pointer type is automatic and can change during the execution. Our system protects against these possibilities, at the cost of lower efficiency.

There exist other concurrent techniques optimized for both uniprocessors as well as multiprocessors. Generational concurrent garbage collectors were also studied, e.g. \cite{Printezis:2000}. Huelsbergen and Winterbottom \cite{Huelsbergen1998} proposed an incremental algorithm for the concurrent garbage collection that is a variant of Mark Sweep collection scheme first proposed in \cite{McCarthy1960}.
Furthermore, garbage collection is also considered for several other systems, namely real-time systems %, mobile actor systems, 
and asynchronous distributed systems, e.g. \cite{Pizlo2008,Veiga2005}.
Concurrent collectors are gaining popularity. The concurrent collector described in Bacon and Rajan~\cite{Bacon2001} can be considered to be one of the more efficient reference counting concurrent collectors. The algorithm uses two counters per object, one for the actual reference count and other for the cyclic reference count. Apart from the number of the counters used, the cycle detection strategy requires a minimum of two traversals of cycle when the cycle is reachable and eleven cycle traversals when the cycle is garbage.

Distributed systems require garbage collectors to reclaim memory. Distributed Garbage Collector(DGC) require algorithm to be designed in different way than the other approaches. So most of the algorithm discussed above cannot be applicable to the distributed setting. Bevan\cite{Bevan87} proposed the use of Weighted Reference Counting as an efficient solution to DGC. Each reference has two weights: a partial weight and a total weight. Every reference creation halves the partial weight. The drawback of this method is an application cannot have more than 32 or 64 references based on the architecture. Piquer\cite{piquer91} suggests an original solution to previous approach. Both of the above mentioned method used straight reference counting, so they cannot reclaim cyclic structures and they are not resilient to message failures. Shapiro et al \cite{Shapiro92} provided a acyclic truly distributed garbage collector. Although the previous method is not cyclic distributed garbage collector, the work lays foundation of how to model a  distributed garbage collection. A standard approach to distributed tracing collector is to combine independent local, per-space collectors, with a global inter-space collector. The main problem with distributed tracing is to synchronize the distributed mark phase with independent sweep phase\cite{plain95}. Hughes designed a distributed tracing garbage collector with  time-stamp instead of mark bits\cite{hugh85}. The method requires the global GC to process nodes only when the application stops its execution or request to stop it when collecting. This is not a practical solution for lot of systems. Several impractical solution based on centralized collector and moving object to one space are proposed\cite{Maheshwari1997,Maheshwari1997b,Liskov,ladin,Veiga05}. Lately, mobile actor based distributed garbage collector was designed\cite{want}. Mobile actor DGC cannot be easily implemented as actor need access to remote site root set and actors move with huge memory.  All of the distributed garbage collection solution have one or more of the problems: inability to detect cycle, moving object to one site for cycle detection, weak heuristics to compute cycle, no fault-tolerance, centralized solution, stopping the world, not scalable, and failure to be resilient to message failures.
